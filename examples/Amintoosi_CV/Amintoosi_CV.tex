\documentclass[11pt,a4paper]{bidimoderncv}
% M.Amintoosi
% برای اجرا باید دنباله کارهای زیر را انجام دهید:
% xelatex Amintoosi_CV
% bibtex Amintoosi_CV
% xelatex Amintoosi_CV
% xelatex Amintoosi_CV

\cvtheme[blue]{bidiclassic}%casual} 
\usepackage{pifont}
\usepackage[scale=0.8]{geometry}
\usepackage{xepersian}
\settextfont[Scale=1]{XB Zar}%{XB Niloofar}%{B Nazanin}%
\setdigitfont{XB Zar}%{XB Niloofar}%{B Nazanin}%
\setlatintextfont[Scale=1]{Linux Libertine}%{Times New Roman}%
\defpersianfont\nastaliq[Scale=1.5]{IranNastaliq}

\AtBeginDocument{\recomputelengths} 
\firstname{محمود}
\familyname{امین‌طوسی}
\resumename{رزومه}
\title{شرح حال}               
\address{آدرس: تهران – نارمک - دانشگاه علم و صنعت ایران – دانشکده مهندسی کامپیوتر}   
\mobile{***۰۹۱۲۲۸۷۴}                   
\phone{***۰۵۳۲۸۲۴۶}  
%\fax{شماره فکس}                          
\email{m.amintoosi@gmail.com}                      
\extrainfo{\httplink[http://webpages.iust.ac.ir/mamintoosi]{webpages.iust.ac.ir/mamintoosi}} 
\photo[64pt]{mAmintoosi.PNG}                         
%\quote{نقل قول}  
\begin{document}
\maketitle
\section{تحصیلات}
\cventry{۱۳۷۰-۱۳۷۴}{کارشناسی}{دانشگاه فردوسی}{ مشهد}{}{ریاضی (کاربرد در کامپیوتر)}  
\cventry{۱۳۷۵-۱۳۷۸}{کارشناسی ارشد}{دانشگاه فردوسی}{ مشهد}{}{مهندسی کامپیوتر(نرم افزار)}  
\cventry{۱۳۸۴-۱۳۸۸}{دکترا(درحال تحصیل)}{دانشگاه علم و صنعت ایران}{ تهران}{}{مهندسی کامپیوتر (هوش مصنوعی) - تاریخ مورد انتظار اتمام: شهریور 88}  

\section{وضعیت فعلی}
\cventry{1380 تاکنون}{عضو هیأت علمی}{گروه ریاضی دانشگاه تربیت معلم سبزوار}{}{}{}
\cventry{۱۳۸۴ تاکنون}{دانشجوی دکترای کامپیوتر، گرایش هوش مصنوعی و رباتیک}{دانشگاه علم و صنعت ایران}{تحت راهنمایی آقایان دکتر محمود فتحی و دکتر ناصر مزینی}{}{}
 
\subsection{دروسی که در دورهٔ دکترا گذرانیده‌ام}
\cvlistdoubleitem[\Neutral]{شبکه‌های عصبی}{پردازش تصویر}
\cvlistdoubleitem[\Neutral]{مهندسی دانش و سیستم‌های خبره}{فرآیندهای اتفاقی}
\cvlistdoubleitem[\Neutral]{هوش مصنوعی توزیع شده}{شناسائی آماری الگو}
\cvlistdoubleitem[\Neutral]{پردازش سیگنالهای رقمی}{بینایی ماشین }
\cvlistdoubleitem[\Neutral]{شبکه‌های عصبی پیشرفته}{}

\subsection{موضوع رساله}
\cvline{}{%
 عنوان تصویب شدهٔ پروژهٔ اینجانب 
{\em «بهبود کیفیت ویدئو با استفاده از تکنیکهای وضوح برتر»} می‌باشد.%\RTLfootnote{Enhancing}
}

\section{کارهای فعلی}
\cvline{}{در حال حاضر بیشتر اوقات مشغول کار بر روی پروژهٔ خود می‌باشم.عمده مطالعات و کارهای فعلی من در موضوعات زیر می‌باشد:}
\cvlistdoubleitem[\ding{72}\ding{72}\ding{72}\ding{78}]{وضوح برتر \lr{\small(Super Resolution)}}{ثبت تصویر \lr{{\small (Image Registration)}}}
\cvlistdoubleitem[\ding{72}\ding{72}\ding{77}\ding{73}]{ترکیب تصاویر \lr{\small (Fusion)}}{تصاویر عریض \lr{\small (Panorama)}}
\cvlistdoubleitem[\ding{72}\ding{72}\ding{78}\ding{73}]{بازسازی سه‌بعدی \lr{\small (3D-Reconstruction)}}{پایدارسازی \lr{\small (Stabilization)}}
\cvlistdoubleitem[\ding{72}\ding{72}\ding{76}\ding{73}]{بهینه‌سازی}{زی‌پرشین}%\lr{\small Seam Carving}}
\cvlistdoubleitem[\ding{72}\ding{72}\ding{73}\ding{73}]{بازسازی تصاویر \lr{\small (Image Reconstruction)}}{شناسائی آماری الگو}
\cvlistdoubleitem[\ding{72}\ding{73}\ding{73}\ding{73}]{مسائل معکوس}{فیلترهای وفقی}


\section{طرح‌های پژوهشی}
\cvlistitem  {طراحی و پیاده سازی وب سایت دانشگاه تربیت معلم سبزوار\hfill {\small\em پایان یافته}}
\cvlistitem  {خودكارسازی برنامه ریزی هفتگی دروس دانشگاهی\hfill {\small\em پایان یافته}}
\cvlistitem  {طراحی و پیاده سازی برنامه مورد نیاز دفتر نظارت و ارزیابی دانشگاه تربیت معلم سبزوار\hfill {\small\em پایان یافته}}
\cvlistitem  {خوشه‌بندی بر مبنای رفتار گروه ماهی‌ها\hfill {\small\em در حال داوری}}
\cvlistitem  {رمزنگاری بصری\hfill {\small\em در حال داوری}}

 
\section{سابقه تدریس}
\cvlistitem	{شروع تدریس از سن 10 سالگی با تدریس ریاضی به دانش آموزان همکلاسی در مدرسه.}
\cvlistitem	{ادامه تدریس به صورت کلاس خصوصی در دوره راهنمایی و دوره کارشناسی.}
\cvlistitem	{تدریس دروس کامپیوتر از سال 1376 در دانشگاههای آزاد اسلامی واحد کاشمر و سبزوار – دانشگاه تربیت معلم و دانشگاه پیام نور سبزوار }
\subsection{دروس تدریس شده:}
\cvlistdoubleitem[\Neutral]{مبانی کامپیوتر و برنامه‌سازی}{برنامه‌سازی پیشرفته}
\cvlistdoubleitem[\Neutral]{ساختمان داده‌ها}{محیط‌های چندرسانه‌ای}
\cvlistdoubleitem[\Neutral]{طراحی الگوریتم‌ها}{آشنایی با نرم‌افزار \lr{MATLAB}}
\cvlistdoubleitem[\Neutral]{گرافیک کامپیوتری}{سیستم عامل }
\cvlistdoubleitem[\Neutral]{ساختمان و زبان ماشین}{پایگاه داده‌ها}


\section{سخنرانی‌ها و کارگاههای برگزار شده}
\cvline{۱۳۸۷}{عصر اطلاعات و اینترنت، {\small جلسهٔ دبیران ریاضی کاشمر}}
\cvline{۱۳۸۶}{مروری بر پروژ‌ه‌های انجام شده در حوزهٔ پردازش تصاویر و مرتبط با ترافیک، {\small هشتمین کنفرانس ترافیک و مهندسی حمل و نقل}}
\cvline{۱۳۸۳}{دوره آموزشی طراحی صفحات وب جهت اعضای هیات علمی، {\small دانشگاه تربیت معلم سبزوار}}
\cvline{۱۳۸۲}{بهینه‌سازی با روش اجتماع مورچگان، {\small دانشگاه تربیت معلم سبزوار}}
\cvline{۱۳۸۱}{محاسبه با \lr{DNA}، {\small دانشگاه تربیت معلم سبزوار}}
\cvline{۱۳۸۱}{دوره آموزشی اینترنت جهت اعضای هیات علمی، {\small دانشگاه تربیت معلم سبزوار}}
\cvline{۱۳۸۰}{مروری بر الگوریتم‌های ژنتیک، {\small دانشگاه تربیت معلم سبزوار}}
\cvline{1379}{دوره آموزشی ویندوز جهت کارکنان شرکت برق، {\small شهرستان سبزوار}}

\section{عضویت در گروه‌ها}
\begin{latin}
\begin{flushleft}
{\small • Member of the EURO Working group on Automated TimeTabling (WATT) \hfill {\scriptsize\em http://www.asap.cs.nott.ac.uk/ASAP/watt/}}\\
{\small • Member of the EU/ME the European chapter on metaheuristics. \hfill {\scriptsize\em http://143.129.203.3/eume/php/eume.main.php}}\\
{\small • Member of the EURO Special Interest Group on Cutting and Packing (ESICUP). \hfill {\scriptsize\em http://www.apdio.pt/sicup/}}\\
{\small • Student member of IEEE (Institute of Electrical and Electronics Engineers). \hfill {\scriptsize\em http://www.ieee.org/}}\\
{\small • Member of the XePersian  Development Team. \hfill {\scriptsize\em http://www.ctan.org/tex-archive/help/Catalogue/entries/xepersian.html}}\\
%{\small • Student member of ISIC (International Student Identification Card). \hfill {\scriptsize\em http://www.isic.ir/}}\\
%{\small • Member of the \LaTeX  Community. \hfill {\scriptsize\em http://www.latex-community.org/}}\\
%{\small • Member of the Parsi \LaTeX $\.$  Group. \hfill {\scriptsize\em http://groups.google.com/group/farsilatex}}\\
%{\small • Member of the Persian Computing  Group. \hfill {\scriptsize\em http://groups.google.com/group/persian-computing}}\\
%{\small • Member of the The Institute for Telecommunication Sciences. \hfill {\scriptsize\em http://www.its.bldrdoc.gov/}}\\
\end{flushleft}
\end{latin}

%%%%%%%%%%%%%%%%%{تألیفات}
\nocite{*}
\bibliographystyle{ieeetr-fa}
\bibliography{BibTeX_M_Amintoosi}


\section{زمینه‌های کاری مورد علاقه}
\cvline{بینائی ماشین}{\small این درس را در دورهٔ دکترا با نمرهٔ ۲۰ گذرانیده‌ام و به خاطر موضوع پروژه روی موارد مرتبط با آن کار می‌کنم.}
\cvline{پردازش تصویر}{\small موضوع پروژه مرتبط با این مقوله نیز می‌باشد.}
\cvline{بهینه‌سازی ترکیبیاتی}{\small پروژهٔ کارشناسی ارشد اینجانب در این حوزه بوده است.}
\cvline{زمانبندی کلاسی}{\small در این زمینه یک طرح پژوهشی و چندین مقاله داشته‌ام.}
\cvline{خط و زبان فارسی}{\small اینجانب داری مدرک سطح «{\nastaliq خوش}» در خوشنویسی از {\nastaliq انجمن خوشنویسان ایران} بوده و در زمینهٔ توسعهٔ 
نرم‌افزارهای متن باز مربوط به زبان فارسی در گروه فارسی‌لاتک فعالیت‌هایی دارم.}
\cvline{زی‌پرشین}{\small زی‌پرشین یک سیستم حروف‌چینی فارسی متن باز و رایگان مبتنی بر \lr{\LaTeXe} است که در سیستم‌عامل‌های ویندوز، لینوکس و مک قابلیت کارکرد داشته و گزینه‌ای بسیار مناسب برای آماده‌سازی مستندات علمی می‌باشد (\lr{\httplink[http://fa.parsilatex.wikia.com]{fa.parsilatex.wikia.com}}). نسخهٔ ۱ زی‌پرشین هم اکنون توسط توزیع‌های معروف لاتک به صورت رسمی منتشر می‌شود. \hfill {\small (متن حاضر توسط زی‌پرشین آماده شده است.)}}


\section{مهارت‌ها}
\cvline{}{در زمینهٔ زبانهای برنامه‌نویسی و نرم‌افزارهای زیر تجربیاتی دارم:}
\cvline{زبانهای برنامه‌نویسی}{}
\begin{flushleft}
\begin{latin}
{FORTRAN, BASIC, COBOL, FoxPro, C and C++, Assembly, Pascal, SQL, PHP, Java and\\ Java Script.}
\end{latin}
\end{flushleft}
\cvline{نرم‌افزارها و بسته‌ها}{}
\begin{flushleft}
\begin{latin}
 MATLAB, C++ Builder, Delphi, \LaTeX, Microsoft Office (Word, Excel, Access, OneNote, Visio,\\ 
 OutLook, PowerPoint), Visual SVN, TortoiseSVN, WinEdt, TeXMaker, Notepad++,\XePersian, \\
 Farsi\TeX, Bib\TeX, FireFox, NetBeans, MiKTeX, JACK, GPSS and Some Others.
\end{latin}
\end{flushleft}

\section{کارگاههایی که شرکت کرده‌ام}
\cvline{۱۳۸۶}{شبیه‌سازی مونت کارلو، {\small دانشگاه صنعتی شریف}}
\cvline{۱۳۸۳}{شبکه‌های عصبی، {\small دانشگاه تربیت معلم سبزوار}}
\cvline{۱۳۸۲}{دوره آموزشی روش تحقیق در علوم پایه، {\small دانشگاه تربیت معلم سبزوار}}

\section{سایر موارد}
\cvlistitem{	رتبه دوم در بین دانشجویان کارشناسی ارشد کامپیوتر دانشگاه فردوسی ورودی 1375 با معدل 18.}
\cvlistitem{	یکی از دانشجویان منتخب دکترا در دانشگاه علم و صنعت ایران در سال 1385 به عنوان دانشجوی نخبه با معدل 18.75.}
\cvlistitem{	بالاترین امتیاز در بین اعضای هیأت علمی گروه ریاضی دانشگاه تربیت معلم سبزوار در دو ترم بر طبق ارزشیابی اساتید از دانشجویان.}

\end{document}
