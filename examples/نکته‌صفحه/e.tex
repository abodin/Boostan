\documentclass{memoir}
% برای تنظیم حاشیه صفحات
\usepackage[top=3cm, bottom=2.5cm, left=2cm, right=2.5cm]{geometry}
% بسته ای برای رنگی کردن لینک ها و فعال سازی لینک ها در یک نوشتار، بسته hyperref باید جزو آخرین بسته‌هایی باشد که فراخوانی می‌شود. 
\usepackage{hyperref}
%% در این قسمت تنظیمات بسته hyperref را قرار می دهیم.
%% این تنظیمات شامل موارد زیر است.
\hypersetup{ 
% به جای استفاده از مربع قرمز دور موارد ارجاعی از لینک های رنگی استفاده کند.
colorlinks=true,
% رنگ برخی از لینک ها در زیر تعریف شده است. 
linkcolor=blue, anchorcolor=green, citecolor=magenta, urlcolor=cyan, filecolor=magenta, pdftoolbar=true
}

% نکته مهم این جا است که بسته Xepersian برای پشتیبانی از زبان فارسی آورده شده است، و 
% می بایست آخرین بسته ای باشد که شما وارد می کنید، دقت کنید: آخرین بسته 
\usepackage{xepersian}
%%  با دستور زیر می توانید فونتی مخصوص عبارات فارسی تعیین کنید:
\settextfont[Scale=1.2]{XB Niloofar}
%% شما با دستور زیر بعد از فراخوانی بسته xepersian می توانید فونت انگلیسی را تعیین کنید
%% دقت کنید که عبارات انگلیسی شما باید در دستور \lr{} قرار گیرد تا xepersian بتواند بفمهد که این عبارات انگلیسی است
\setlatintextfont[Scale=1]{Times New Roman}

\notepageref
\renewcommand*{\notesname}{نکات}
\renewcommand*{\notedivision}{\chapter{\notesname}}
\renewcommand*{\pagerefname}{صفحه}
\makepagenote

\begin{document}
\baselineskip = .85cm

روابط ایران و عربستان سعودی پس از پیروزی انقلاب همواره با تنش هایی روبرو بوده است. هرچند جمهوری اسلامی ایران در دوره هایی تلاش کرد که با تنش زدایی روابط ایران و عربستان دوستانه گردد اما این روابط دوستانه چندان پایدار نبوده است. حداقل در نگاه کلی به نظر می رسد که عربستان چندان تمایلی به گسترش روابط نزدیک با جمهوری اسلامی ایران نداشته است. شاهد مثال این موضوع می تواند سفر مقامات بلندپایه کشورمان در دوره های مختلف به عربستان سعودی و عدم پاسخ این سفرها از سوی مقامات سعودی باشد.
\pagenote{ایران و روابط}

روابط ایران و عربستان سعودی پس از پیروزی انقلاب همواره با تنش هایی روبرو بوده است. هرچند جمهوری اسلامی ایران در دوره هایی تلاش کرد که با تنش زدایی روابط ایران و عربستان دوستانه گردد اما این روابط دوستانه چندان پایدار نبوده است. حداقل در نگاه کلی به نظر می رسد که عربستان چندان تمایلی به گسترش روابط نزدیک با جمهوری اسلامی ایران نداشته است. شاهد مثال این موضوع می تواند سفر مقامات بلندپایه کشورمان در دوره های مختلف به عربستان سعودی و عدم پاسخ این سفرها از سوی مقامات سعودی باشد.
\pagenote{این هم یک نکته}

روابط ایران و عربستان سعودی پس از پیروزی انقلاب همواره با تنش هایی روبرو بوده است. هرچند جمهوری اسلامی ایران در دوره هایی تلاش کرد که با تنش زدایی روابط ایران و عربستان دوستانه گردد اما این روابط دوستانه چندان پایدار نبوده است. حداقل در نگاه کلی به نظر می رسد که عربستان چندان تمایلی به گسترش روابط نزدیک با جمهوری اسلامی ایران نداشته است. شاهد مثال این موضوع می تواند سفر مقامات بلندپایه کشورمان در دوره های مختلف به عربستان سعودی و عدم پاسخ این سفرها از سوی مقامات سعودی باشد.

روابط ایران و عربستان سعودی پس از پیروزی انقلاب همواره با تنش هایی روبرو بوده است. هرچند جمهوری اسلامی ایران در دوره هایی تلاش کرد که با تنش زدایی روابط ایران و عربستان دوستانه گردد اما این روابط دوستانه چندان پایدار نبوده است. حداقل در نگاه کلی به نظر می رسد که عربستان چندان تمایلی به گسترش روابط نزدیک با جمهوری اسلامی ایران نداشته است. شاهد مثال این موضوع می تواند سفر مقامات بلندپایه کشورمان در دوره های مختلف به عربستان سعودی و عدم پاسخ این سفرها از سوی مقامات سعودی باشد.

روابط ایران و عربستان سعودی پس از پیروزی انقلاب همواره با تنش هایی روبرو بوده است. هرچند جمهوری اسلامی ایران در دوره هایی تلاش کرد که با تنش زدایی روابط ایران و عربستان دوستانه گردد اما این روابط دوستانه چندان پایدار نبوده است. حداقل در نگاه کلی به نظر می رسد که عربستان چندان تمایلی به گسترش روابط نزدیک با جمهوری اسلامی ایران نداشته است. شاهد مثال این موضوع می تواند سفر مقامات بلندپایه کشورمان در دوره های مختلف به عربستان سعودی و عدم پاسخ این سفرها از سوی مقامات سعودی باشد.

روابط ایران و عربستان سعودی پس از پیروزی انقلاب همواره با تنش هایی روبرو بوده است. هرچند جمهوری اسلامی ایران در دوره هایی تلاش کرد که با تنش زدایی روابط ایران و عربستان دوستانه گردد اما این روابط دوستانه چندان پایدار نبوده است. حداقل در نگاه کلی به نظر می رسد که عربستان چندان تمایلی به گسترش روابط نزدیک با جمهوری اسلامی ایران نداشته است. شاهد مثال این موضوع می تواند سفر مقامات بلندپایه کشورمان در دوره های مختلف به عربستان سعودی و عدم پاسخ این سفرها از سوی مقامات سعودی باشد.
\pagenote{این هم یک نکته دیگر}


\printpagenotes

\end{document}

