\documentclass{report}

\usepackage{amsmath}

%زی‌پرشین (به انگلیسی: XePersian) یک بسته حروف‌چینی رایگان و متن‌باز برای نگارش مستندات پارسی/انگلیسی با زی‌لاتک است.
% در واقع، زی‌پرشین، کمک می‌کند تا به آسانی، مستندات را به پارسی، حروف‌چینی کرد. این بسته را وفا خلیقی نوشته است،
% و به طور منظم، آن را بروز‌رسانی کرده و باگ‌های آن را رفع می‌کند.
% نکته مهم این جا است که بسته Xepersian برای پشتیبانی از زبان فارسی آورده شده است، و 
% می بایست آخرین بسته ای باشد که شما وارد می کنید، دقت کنید: آخرین بسته
\usepackage[extrafootnotefeatures]{xepersian}

%%  با دستور زیر می توانید فونتی مخصوص عبارات فارسی تعیین کنید:
\settextfont[Scale=1.2]{XB Niloofar}
%% شما با دستور زیر بعد از فراخوانی بسته xepersian می توانید فونت انگلیسی را تعیین کنید
%% دقت کنید که عبارات انگلیسی شما باید در دستور \lr{} قرار گیرد تا xepersian بتواند بفمهد که این عبارات انگلیسی است
\setlatintextfont[Scale=1]{Times New Roman}


\begin{document}
این مثال\footnote{شبکه اقتضایی} و یا پاورقی انگلیسی\LTRfootnote{Salam}

یک مثال از نوشتن ماتریس
\begin{equation}
\Theta =\begin{bmatrix}
\lambda_{d}^{11} & \lambda_{d}^{12} & \ldots & \lambda_{d}^{1(N-1)} & \lambda_{d}^{1N} \\*[2mm]
0 & \lambda_{d}^{22}  & \ldots & \lambda_{d}^{2(N-1)} & \lambda_{d}^{2N} \\
\vdots & \vdots  & \ddots & \vdots & \vdots\\*[2mm]
0 & 0 &  \ldots & \lambda_{d}^{(N-1)(N-1)} & \lambda_{d}^{(N-1)N}\\*[2mm]
0 & 0 &  \ldots & 0 & \lambda_{d}^{NN}
\end{bmatrix},
\label{wqlsdskdksds}
\end{equation}
\begin{equation}
\Theta =\begin{pmatrix}
\lambda_{d}^{11} & \lambda_{d}^{12} & \ldots & \lambda_{d}^{1(N-1)} & \lambda_{d}^{1N} \\*[2mm]
0 & \lambda_{d}^{22}  & \ldots & \lambda_{d}^{2(N-1)} & \lambda_{d}^{2N} \\
\vdots & \vdots  & \ddots & \vdots & \vdots\\*[2mm]
0 & 0 &  \ldots & \lambda_{d}^{(N-1)(N-1)} & \lambda_{d}^{(N-1)N}\\*[2mm]
0 & 0 &  \ldots & 0 & \lambda_{d}^{NN}
\end{pmatrix},
\label{wqlsdskdksds}
\end{equation}
	
\end{document}