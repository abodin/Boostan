
\begin{questionblock}%1
\question{
دو ایده  آقای \lr{Baran} برای شبکه‌های جایگزین شبکه‌های کلیدزنی مداری چه بود؟ 
}
\answer
{ارسال اطلاعات در بسته‌های مجزا-قراردادن آدرس مقصد در بسته‌ها}
{قراردادن آدرس مقصد در بسته‌ها-تنظیم نرخ  ارسال بر اساس وضعیت گیرنده}
{ارسال اطلاعات در بسته‌های مجزا-ارسال هر بسته از  مسیرهای مختلف}
{ارسال هر بسته از مسیرهایی مختلف-تنظیم نرخ  ارسال بر اساس وضعیت گیرنده}
\end{questionblock}

\begin{questionblock}%2
\question{
دو هدف بوجود آمدن پروتکل
\lr{TCP}
در سال ۱۹۷۴ توسط 
\lr{Vint Cerf} و \lr{Bob Kahn}
را ذکر کنید؟
}
\answer
{شکستن پیام لایه کاربرد به چندین بسته-تقسیم بهینه منابع بین کاربران}
{ مدیریت خطای ایجاد شده بر روی بسته‌ها-کاهش تاخیر مابین دو ارسال}
{اتصال ساده و با کمترین تغییرات بین دو شبکه - مدیریت خطای ایجاد شده بر روی بسته‌ها}
{اتصال ساده و با کمترین تغییرات بین دو شبکه-تقسیم بهینه منابع بین کاربران}
\end{questionblock}


\begin{questionblock}%3
\question{
جای خالی را پر کنید:

..... مجموعه‌ای از یادداشت‌های تکنیکی پیرامون شبکه و اینترنت است، که از سال 1969 تا امروز در حال تنظیم و توسعه است. این یادداشت‌ها شامل پروتکل‌ها بوده. مرکز رسمی و نظارت بر این پروتکل‌ها و یادداشتها سازمان \lr{IETF} است. به طور کلی روال تکاملی ..... از \lr{Proposed Standard} شروع می‌شود، یعنی شرکتی و یا اشخاصی برای ارایه یک استاندارد پیشنهاد داده، و بعد از بررسی‌های مختلف توسط سازمان \lr{IETF}، بعنوان یک \lr{Draft Standard} ارایه می‌شود؛ و بعد از انجام بررسی‌ها و آزمایشات مختلف توسط \lr{IETF} و سایر سازمان‌ها بعنوان \lr{Standard} انتخاب می‌شود؛ و در دنیای شیرین شبکه و اینترنت قابل استفاده می‌باشد. ......  بصورت فایل‌های متنی هستند که بیشتر در قالب موضوعات تخصصی مانند تشریح پروتکل‌های مختلف منتشر می شوند. 
}
\answer
{\lr{\lr{RFC} (Request For Comments)}}
{\lr{\lr{TS} (Technical Specifications)}}
{\lr{\lr{TR} (Technical Reports)}}
{\lr{ITU Q. (International Telecommunication Union Q.)}}
\end{questionblock}


\begin{questionblock} %4
	\question{
دو مزیت کلیدزنی بسته‌ای را نسبت به کلیدزنی مداری به شرح زیر است؟}
\answer
	{تضمین
\lr{QoS}-
		تاخیر کمتر}
	{تاخیر کمتر-تعداد کاربران بیشتر}
	{پشتیبانی از تعداد کاربران بیشتر-بهره‌وری بالا در استفاده از منابع}
	{بهره‌وری بالا در استفاده از منابع-تضمین
\lr{QoS}}
\end{questionblock}


\begin{questionblock} %5
	\question{
شبکه‌ای متشکل از $n$ کاربر را در نظر بگیرید. فرض کنید که هر کاربر به احتمال $p$ فعال و به احتمال $q=1-p$ فعال نیست. احتمال این‌که در این شبکه در یک زمان  بیشتر از 20 کاربر فعال  (20 نفر یا بیشتر) باشد، کدام یک از گزینه‌های زیر است؟
	}
	\answer
	{$\sum_{i=1}^{20}\binom{n}{i}p^{i}q^{n-i}$}
	{$\sum_{i=1}^{19}\binom{n}{i}p^{i}q^{n-i}$}
	{$\sum_{i=20}^{\infty}\binom{n}{i}p^{i}q^{n-i}$}
	{$\sum_{i=1}^{20}\binom{n}{i}p^{n-i}q^{i}$}
\end{questionblock}

\begin{questionblock} %6
\question{
کدام یک از وظایف زیر را برعهده یک پروتکل نیست؟}
\answer
{قالب پیام}
{ترتیب ارسال پیام‌ها}
{واکنش در برابر هر پیام}
{میزان تاخیر هر بسته}
\end{questionblock}

\begin{questionblock} %7
	\question{
	یک فایل $2$ مگابایتی را در نظر بگیرید. فرض کنید می‌خواهیم این فایل را به بسته‌های $100$ کیلو بیتی تقسیم‌کنیم. کل فایل چند بسته خواهد شد و در ضمن  اگر کارت شبکه ‌ای با سرعت $2$ مگابیتی داشته‌باشیم، چقدر طول می‌کشد تا کل فایل بر روی پیوند ارتباطی قرار گیرد؟ (هر مگا بیت را برابر با $1000$ کیلوبیت در نظر بگیرید.)
	}
	\answer
	{در کل 20 بسته - ۸ ثانیه طول خواهد کشید}
	{در کل 160 بسته - ۸ ثانیه طول خواهد کشید}
	{در کل 20 بسته - 16 ثانیه طول خواهد کشید}
	{در کل 160 بسته -16 ثانیه طول خواهد کشید}
\end{questionblock}
