\documentclass{article}

\input{gozineh}

\newcount\myn
\begin{document}
\baselineskip=.7cm

\loop
\advance\myn by 1
\ifnum\myn>3
\else

\begin{table}[H]
	\begin{tabular}{|p{5cm}|p{3cm}|p{3cm}|c|}\hline
نام و نام خانوادگی & شماره دانشجویی & تاریخ  & شماره برگه 
			\\\hline
		& & 1396/11/19& \the \myn
		\\\hline
	\end{tabular}
\end{table}
\begin{goal}{نکات}
الف) این امتحان نمره منفی دارد.

ب) دقت کنید که نام و نام‌خانوادگی خود را بر روی تمامی برگه‌ها بنویسید.

ج) در صورتی که احساس می‌کنید سوالی غلط است یا دو جواب دارد فرض خود را در کنار سوال بنویسید. البته قاعدتا یک جا دارید اشتباه می کنید و هر سوال فقط یک جواب دارد.
\end{goal}
%\begin{problem}
%روش برخورد 
%\lr{TCP Reno}
%با 
%\lr{three duplicate ACKs}
%را توضیح دهید؟ (در پایان سوالات تستی به این سوال تشریحی نیز پاسخ دهید.)
%\end{problem}


 \renewcommand*{\arraystretch}{1.9}
 \begin{tabular}{|p{5cm}p{6cm}p{5cm}|}\hline
 	\multicolumn{3}{|r|}{ در سوالاتی که در ادامه می‌آید، پارامترهای زیر را در نظر بگیرید.}\\
 	$L$: طول بسته  & $R$: \mygls{LinkCapacity} & $D$: \mygls{Distance} \\
 	$a$: میانگین نرخ رسیدن بسته‌ها & $S$: \mygls{PropagationSpeed} موج & \\\hline
 \end{tabular}
 \vskip 6mm
\setcounter{question}{0}
\begin{mcquestions}
\randomquestionsfrombank{bankA.tex}{7}
\end{mcquestions}
\clearpage
\ifthenelse{\isodd{\thepage}}{}{\clearpage\mbox{}\clearpage}
\repeat

\end{document}