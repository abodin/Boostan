\chapter{مقدمه}
در این فصل ابتدا مقدمه‌ای بر استفاده از پهپاد و طرح مسئله ارایه می‌شود، سپس نو‌آوری این پژوهش و ساختار این پایان‌نامه شرح داده می‌شود.
\section{طرح مسئله}
پرنده هدایت‌پذیر از راه دور (پهپاد) که عموماً به‌عنوان هواپیماهای بدون سرنشین نیز شناخته می‌شوند، به دلیل استقرار، انعطاف و دامنه وسیعی که این حوزه تحقیقاتی دارد، موضوع تحقیقات در چند سال اخیر بوده است \cite{7317490}.
در واقع، پهپادها کاربردهای مختلفی در حوزه‌های گوناگون دارند که شامل: نظارت و پایش، ارتباطات از راه دور، تحویل تجهیزات، پایش محیط و عملیات نجات است \cite{yanmaz2018drone} - \cite{9419470}. بااین‌حال، اکثر تحقیقات متعارف پهپاد معمولاً بر روی موضوعات پیمایش، جایابی، تحرک، واپایش و نظارت متمرکز است، در مقابل نیز کاربردهای انگیزه دهنده‌ای مانند رباتیک یا نظامی نیز وجود دارد. در مقابل، چالش‌های ارتباطی پهپادها جز چالش‌هایی است که همچنان باز و موردبحث است \cite{nawaz2021uav}.
امروزه استفاده از پهپادها به‌عنوان وسایل هوایی متحرک و مقرون به صرفه‌ای که می‌توانند کاربردهای گوناگونی داشته باشند، روبه‌رشد است. پیش‌ازاین، این‌طور فرض می‌شود که کاربرد پهپادها فقط به کاربردهای نظامی محدود می‌شود، اما امروزه با پیشرفت نمونه‌های مقرون‌به‌صرفه و ارزان‌قیمت شاهد کاربردهای گوناگون این وسیله هدایت‌پذیر از راه دور هستیم. این کاربردها از زمینه‌های مختلف اعم از استفاده پهپادها در شبکه‌های بی‌سیم، کاربردهایی نظیر \gls{Control} امنیت عمومی کاربران و پایش و رصدکردن مناطق را می‌توان نام برد. پهپادها را می‌توان به انواع مختلف با کارایی مختلف تقسیم کرد. به‌طورکلی، پهپادها به دودسته بال ثابت و بال چرخان تقسیم می‌شوند، و هر کدام از آن‌ها در سناریو خاص خود کاربرد دارند. در فصل دوم در مورد پهپادها و دسته‌بندی آن‌ها بیشتر توضیح داده خواهد شد، از جمله این موارد می‌توان به دسته‌بندی پهپاد، سکوی قرارگیری پهپاد و موارد دیگر اشاره کرد، ازاین‌رو آشنایی با انواع پهپادها باتوجه‌به کارایی و سناریو آن‌ها اهمیت بالایی دارد \cite{mozaffari2019tutorial}.

یکی از کاربردهایی که در این پژوهش مورد بررسی بیشتری قرار می‌گیرد و از اهداف مطالعاتی این پژوهش است، کاربرد پهپاد در پایش محیط است، پایش محیط می‌تواند جنبه‌های مختلفی را شامل شود، از این دسته از کاربردها می‌توان به پایش یک محیط به‌منظور مدیریت بهتر محصولات کشاورزی یا مراقب از محیط ویژه‌ای دانست. همچنین نظارت و پایش به‌منظور رسیدگی به کاربران زمینی در عملیات نجات در شرایط بلایای طبیعی یا غیرطبیعی که سناریو اصلی این پژوهش است. از طرفی باید اشاره شود که استفاده از پهپادها امروزه محدود به کاربردهای نظامی نمی‌شود، شاهد آن هستیم که می‌تواند کاربردهای متنوعی در خدمت‌رسانی در محیط‌های شهری یا غیرشهری داشته باشد. هدف اصلی این پژوهش نیز علاوه بر مدیریت تحرک پهپادها که در این فصل به‌صورت کامل شرح داده‌ایم که شرایط را برای خدمت‌رسانی به کاربران مهیا می‌کند، به‌عنوان یک سناریو در دنیای واقعی پهپاد را به‌عنوان پایشگر محیط در نظر گرفته‌ایم. پهپادها این قابلیت را دارند که در محیط‌های مختلف بتوانند کارکرد مناسبی از خود نشان دهند، از ویژگی‌های مفید پهپادها که برای پایش محیط نیاز است، دسترسی به نقاط سخت‌گذر یا محیط‌هایی که دسترسی به آن‌ها مشکل است. پهپادها می‌توانند در محیط‌های مختلف مأموریت‌های خود را به‌خوبی انجام دهند. آن چیزی که در این پژوهش مورد بررسی قرار می‌گیرد، ابتدا مفاهیم اولیه در مورد شناخت پهپاد و معرفی کاربردهای پهپاد در شبکه‌های بی‌سیم و امنیت عمومی به‌منظور پایش و نگه‌داری از محیط که لازمه پایش محیط است، معرفی و مورد بررسی قرار می‌گیرد. در ادامه پژوهش‌های پیشین دررابطه‌با پایش محیط با استفاده از پهپاد همچنین روش‌های مختلف جایابی و تحرک پهپادها به‌منظور مدیریت هر چه‌بهتر تحرک پهپادها مورد پژوهش قرار گرفته شده است.

باید ادامه دهیم که استفاده از پهپاد به‌عنوان یک فناوری قابل‌دسترس برای نظارت و پایش محیط به‌منظور نگه‌داری در امنیت عمومی، به دلیل داشتن ویژگی‌های منحصری، گزینه مناسبی جهت استفاده در این زمینه است. استفاده از پهپاد به دلیل داشتن ویژگی فیزیکی پهپادها که دررابطه‌با آن پژوهش شده است، از جمله این ویژگی‌ها که کمک‌کننده در پایش و نظارت یک محیط است، به ارتفاع قابل‌تنظیم و وقف پذیر در محیط‌های مختلف، کم‌هزینه‌بودن استفاده از پهپاد، وجود تنوع در مدل‌های مختلف پهپاد بسته به سناریو مورداستفاده، دسترسی ساده به نقاط سخت‌گذر و همچنین همیار شبکه‌های دیگر می‌توان اشاره کرد. این پژوهش با شناخت ویژگی‌های فیزیکی پهپاد به دنبال بهبود عملکرد پهپادها در زمینه پایش محیط است، شناخت بهتر ویژگی پهپادها کمک می‌کند که در یک سناریو دنیای واقعی بتوانیم بهترین عملکرد را داشته باشیم.
این مورد را می‌توان یکی دیگر از ویژگی‌های پهپادها دانست، آن‌ها می‌توانند با محیط خود درهم زیستی داشته باشند، به‌عبارت‌دیگر، پهپادها محدود به محیط خواصی نیستند، آن‌ها را می‌توان در محیط‌های مختلف شهری و غیرشهری به کار برد.


هدف اصلی که این پژوهش با آن روبه‌رو است، همان‌طور که به آن اشاره شد، مدیریت تحرک پهپادها به‌منظور بهره‌وری از کاربرد پهپاد در محیط‌های شهری و غیرشهری است که این بررسی خود می‌تواند شامل کاربردهای بالقوه در یک سناریو پایش و نظارت برای خدمت‌رسانی نیز باشد، به‌عبارت‌دیگر در کاربردهایی که در زمینه‌های مختلف استفاده از پهپاد وجود دارد، فناوری‌های مختلف در کنار پهپادها هم زیستی دارند، به‌عنوان‌مثال استفاده پهپاد در سناریوهای اینترنت اشیا خود می‌تواند شرایط مختلفی را به همراه داشته باشد، مثلاً استفاده از فناوری خاص ارتباطی در این سناریو قابل‌بحث و پژوهش است. استفاده از مدل‌های یادگیری ماشین نیز در سناریوهای پهپادی به دلیل نظارت دقیق پهپادها در کاربردهای پهپاد در محیط دیده می‌شود. شناخت بعضی از مدل‌های یادگیری و استفاده‌کردن این مدل‌ها در پردازش تصویر بی‌درنگ با استفاده از پهپادها کمک شایانی در زمینه پایش محیط با استفاده از پهپاد می‌کند. این کاربرد نیز محدود به محیط خواصی نمی‌شود، پهپاد را می‌توان به‌منظور جمع‌آوری تصاویر بی‌درنگ و پردازش آن در محیط‌های مختلف مورداستفاده قرار داد.



دررابطه‌با پایش محیط‌های گوناگون به‌طورکلی می‌توان رویکردهای مختلف و رایجی را معرفی کرد که دررابطه‌با هر یک توضیحاتی داده شده است، ادامه به بررسی هر یک پرداخته شده است.

رویکرد اول، بررسی پایش محیط با محوریت شبکه بی‌سیم، در این کاربرد، پهپادها به‌عنوان جمع‌کننده داده‌ها عمل می‌کنند. در یک شبکه حسگر بی‌سیم که می‌تواند در محیط شهری یا غیرشهری قرار گیرد، پهپادها می‌توانند در یک مقیاس بزرگ داده‌های حسگرها را جمع‌آوری کنند \cite{Zhang2020}، همچنین پهپادها برای پایش محیط‌هایی که زیر ساخت مناسب ارتباطی ندارند، کاربرد دارد. در این پژوهش کاربردهای پهپاد در این سناریوها نیز مورد بررسی قرار گرفته است، از مواردی که می‌توان به آن اشاره کرد، استقرار پهپاد در یک محیط به‌منظور پایش زمین کشاورزی که به آن کشاورزی دقیق نیز گفته می‌شود، با استفاده از پهپادها امکان‌پذیر است. کاربرهای پهپاد با استفاده از این رویکرد محدود به محیط‌های غیرشهری نمی‌شود، امروزه می‌توان پهپادها را در محیط شهری نیز به کار برد، به‌منظور \gls{Control} و پایش ترافیک شهری کاربرد دارد. دررابطه‌با موارد کاربرد پهپاد در محیط‌های مختلف، در بیشتر مورد بررسی قرار گرفته شده و کارهای مرتبط در زمینه پایش محیط مورد بررسی قرار گرفته شده است\cite{Tan2021} - \cite{Ma2021}.

رویکرد دوم، بررسی پایش محیط با استفاده از پردازش تصویر و مدل‌های یادگیری عمیق، در این رویکرد، پهپادها با استفاده از دوربین‌هایی که به آن مجهز شده‌اند، محیط را پایش می‌کنند\cite{Yilmaz2020} -\cite{Kannadaguli2020}. با استفاده از مدل‌های یادگیری عمیق پهپادها این توانایی را دارند که بتوانند در محیط‌های مختلف نقش مفیدی داشته باشند. به طور ویژه پهپادها با استفاده از این نوع پایش می‌توانند از خیلی از خسارات جبران‌ناپذیر جلوگیری کنند، به‌عنوان نمونه، پایش یک محیط جنگلی با استفاده از پهپاد به‌منظور تشخیص آتش‌سوزی یکی از کاربردهای پهپاد در این زمینه است، در این سناریو پهپادها که با دوربین‌هایی تجهیز شده‌اند، این قابلیت را دارند که آتش‌سوزی را بادقت بسیار بالایی تشخیص داده و به‌سرعت گزارش دهند \cite{Jiao2020}. دررابطه‌با پایش محیط توسط پهپاد با رویکرد مذکور در بیشتر مورد پژوهش قرار خواهد گرفت.


استفاده از پهپاد باتوجه‌به سه عامل: ویژگی، محیط مورد پایش و فناوری‌های قابل‌استفاده با پهپاد مورد بررسی قرار می‌گیرد، این موارد انگیزه ایجاد می‌کند که در این زمینه بتوان به پژوهش پرداخت و با بررسی مدل‌های ارایه شده در این زمینه بتوان به مدلی مناسب در جهت پایش هر چه‌بهتر و دقیق‌تر دست‌یافت، پایش یک محیط از جنبه‌های مختلف از اهمیت بالایی برخوردار است، پایش یک محیط با استفاده از پهپاد می‌تواند نتایج دقیق‌تری را به همراه داشته باشد، همچنین پایش محیط با استفاده از پهپاد می‌تواند استفاده از نیروی انسانی را به‌شدت کاهش دهد، کار پایش با استفاده از پهپاد می‌تواند در وقت و هزینه‌ها صرفه‌جویی داشته باشد اگر به‌درستی برنامه‌ریزی شوند این امر امکان‌پذیر است. همچنین استفاده از پهپاد می‌توان یک مقایس عظیمی را از محیط پوشش دهد تا پایش محیط به این واسطه به‌دقت و پوشش انجام شود. در این پژوهش سعی شده تا فناوری‌های موردنیاز برای پایش محیط با استفاده از پهپادها معرفی و بررسی شود، از جمله این موارد می‌توان به شبکه بی‌سیم بر پایه پهپاد، شبکه اقتضایی پهپادی و یادگیری ماشین در استفاده از پهپاد اشاره کرد که هر یک در پایش محیط در دسته‌بندی خود مورداستفاده قرار می‌گیرند.


در این پژوهش با دانستن ویژگی‌ها، کاربردها و استفاده پهپاد در محیط‌های مختلف به تحقیق در این زمینه پرداختیم تا علاوه بر شناخت درست این زمینه به بهبود عملکرد پهپاد در پایش محیط بپردازیم، این بهبود شامل مدیریت تحرک پهپادها است. باید در نظر داشت که استفاده پهپاد در محیط شهری و غیرشهری می‌تواند چالش‌های متفاوتی داشته باشد، به‌عنوان‌مثال استفاده پهپاد در یک محیط شهری می‌تواند بحث امنیت فیزیکی و حریم خصوصی افراد جامعه را به همراه داشته باشد، در نتیجه استفاده از پهپاد علاوه در وجود ساختمان‌ها و موانع بزرگ که باید در نظر گرفته شود، دارای محدودیت‌هایی از جمله امنیت فیزیکی و حریم خصوصی نیز است که باید در نظر گرفته شود. در محیط غیرشهری خیلی از محدودیت‌های محیط شهری وجود ندارد، می‌تواند به‌آسانی طرح‌ریزی شوند و مورداستفاده قرار گیرد.


در انتها باید اشاره شود بهبود پایش محیط و خدمت‌رسانی در یک محیط به‌وسیله پهپادها که در این پایان‌نامه به دنبال آن هستیم شامل مدیریت تحرک پهپادها می‌شود، تحرک پهپادها بعد از شناخت ابعاد مختلف استفاده از پهپادها در محیط‌های مختلف با رویکردهای مختلف امکان‌پذیر است. در این پژوهش با بررسی مدل‌های تحرک پهپادها و مقایسه این مدل‌ها با یکدیگر در سناریو موردنظر به دنبال آن هستیم که بتوانیم با این کار روند پایش محیط با استفاده از پهپاد را بهبود دهیم. علت مهم بودن تحرک در پهپادها به دلیل ویژگی فیزیکی پهپادها است چرا که محدود به انرژی هستند، از طرفی با استفاده و مقایسه مدل‌های تحرک مختلف در به‌کارگیری پهپادها می‌توان به سمت بهبوددادن تحرک پهپاد در یک محیط به کاربرد پایش در محیط دست‌یافت چرا که پایش محیط می‌تواند در محیط‌های شهری و غیرشهری انجام گیرد، می‌توان استفاده از یک مدل تحرک و بهبود عملکرد پهپاد را در محیط به‌منظور پایش با استفاده از مدیریت تحرک بهتر کرد.


ممقایسه و عملکرد مدل‌های تحرکی رایج پیشنهادی مورد بررسی قرار می‌گیرد تا بتوان با درنظرگرفتن مدل‌های مختلف عملکرد پهپاد به‌منظور پایش در محیط را بهبود دهیم. در فصل پنجم مدل‌سازی و ارزیابی مدل تحرک پیشنهادی شبیه‌سازی و مورد بررسی قرار می‌گیرد تا اطمینان حاصل شود مدل یاد شده می‌تواند عملکرد مناسبی در پایش محیط داشته باشد. در فصل ششم نیز نتیجه‌گیری کار انجام شده و مسیر پیشروی آینده‌پژوهش انجام شده برای توسعه بیشتر آن موردتوجه قرار گرفته است.


در این پایان‌نامه با به‌کارگیری از الگوریتم‌های فراابتکاری و روش‌های حل دقیق سعی بر بهینه‌سازی تحرک و جانمایی پهپادها داریم، تحرک و جانمایی پهپادها در سناریوهای مختلف از اهمیت بالایی برخوردار است، زیرا با یک طرح‌ریزی دقیق پروازی می‌توان کاربران را بهتر مورد پوشش و خدمت‌رسانی قرار داد. در این پایان‌نامه پارامترهای اصلی شبکه در سناریو استفاده از پهپاد مورد بررسی قرار می‌گیرد، پهپادها در چنین سناریوهایی نقش ایستگاه‌های پایه هوایی را بازی می‌کنند. با استفاده از این ایستگاه‌های پایه هوایی علاوه بر خدمت‌دهی به کاربران می‌توان نقش پایشگر محیط را نیز بازی کند.

\section{نوآوری}
	\subsection{مدیریت تحرک پهپادها} \label{Chapter1Section1} 

در این بخش به تعریف مدیریت تحرک پهپادها و هدف از مدیریت تحرک پهپادها را شرح می‌دهیم. همان‌طور که به آن اشاره شده، یکی از عوامل مهم پایش محیط یا به عبارتی خدمت‌رسانی به کاربران زمینی در شرایط خاص، بلایا یا حفظ امنیت عمومی استفاده از روش بهبودیافته استقرار و جایابی در نهایت تحرک بهینه پهپادها است، چرا که پهپادها علی‌رغم داشتن ویژگی‌های متعدد استقرار دارای محدودیت‌هایی از جمله انرژی مصرفی و محدود به انرژی هستند. درنظرگرفتن محدودیت انرژی در استقرار و تحرک پهپادها از جمله کارهایی است که این پژوهش سعی در بهبود آن را دارد، ازاین‌رو می‌توان با کمینه‌کردن مصرف انرژی در یک عملیات پروازی در کنار استقرار بهبودیافته پهپادها مدیریت تحرک بهتری را در یک سناریو شهری یا غیرشهری تجربه کنیم. چندین مورد از انگیزه‌های مدیریت تحرک پهپادها شامل: 
\begin{itemize} 
	
	\item 
	پشتیبانی از تحرک بهینه به جهت ارتباط پهپاد با کاربران زمینی، 
	\item 
	تحرک بهبودیافته پهپاد می‌توان در مصرف انرژی پهپاد نیز تأثیرگذار باشد، 
	\item 
	تحرک بیش از اندازه پهپاد باعث هدررفت انرژی محدود پهپاد می‌شود، 
	\item 
	استفاده از مدل‌های تحرک رایج باعث هدررفت توان پهپاد می‌شود، 
	\item 
	کمینه‌کردن مصرف انرژی با تحرک بهینه.
\end{itemize} 

به‌طورکلی در تعریف مدیریت تحرک؛ مدیریت تحرک را در شبکه‌های پهپاد شامل مدیریت تحرک پهپادها برای اطمینان از عملکرد کارآمد شبکه، مانند حفظ اتصال شبکه، بهینه‌سازی تحویل داده‌ها و به‌حداقل‌رساندن مصرف انرژی دانست. این پژوهش بر اساس یافته‌ها مدلی را معرفی می‌کند که در هم زیستی با کاربران زمینی با کمترین تحرک می‌تواند (صرف کمترین میزان توان) می‌تواند حداکثر پوشش‌دهی را داشته باشد. این امر کمک می‌کند که در مأموریت‌های پروازی به‌منظور پایش، مراقب و خدمت‌رسانی به کاربران زمینی ثابت یا غیرثابت بهترین بازدهی را داشته باشد.
در مقابل مدیریت تحرک می‌توان مدل‌های تحرک رایج را به آن اشاره کرد، مدل‌های رایج تحرک از یک الگوی از پیش تعیین شده و کلی پیروی می‌کنند، بدیهی است که الگوها کلی و از پیش تعیین شده به‌مانند \gls{RW} که حرکت تصادفی با سرعت تصادفی را دارد نمی‌تواند در سناریوها یا عملیات پهپادی بهره‌وری مناسبی داشته باشد. پس می‌توان نتیجه گرفت بهبود استفاده از پهپاد در یک سناریو و عملیات واقعی وابسته به کمینه‌کردن انرژی و در نهایت مدیریت تحرک بهبودیافته پهپادها است.


\section{ساختار پایان‌نامه}
در این پایان‌نامه در ابتدا با ادبیات موضوع استفاده از پهپاد‌ها و کاربرد آن در محیط‌های مختلف آشنا خواهیم شد، سناریو‌های مختلف و کاربر‌د‌های آن در فصل دوم بررسی می‌شود، علاوه بر این با دسته‌بندی پهپادها و قوانین استفاده از پهپاد‌ها نیز آشنا خواهیم شد، در ادامه فصل سوم کارهای مرتبط با این پژوهش معرفی می‌گردد که خود شامل چند بخش متمایز است. در فصل چهارم نیز ساختاری از روش ارایه شده در این پژوهش معرفی می‌گردد، در ادامه معرفی روش، در فصل پنجم الزامات مدل‌سازی بیان می‌گردد و همچنین نتایج مدل‌سازی ارایه و تشریح می‌گردد. در نهایت نیز در فصل پایانی جمع‌بندی و کار‌های آینده و مرتبط با این موضوع معرفی می‌گردد.


