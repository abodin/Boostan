\chapter{شبیه‌سازی}
\label{chap:simulation}

\section{وارد کردن کد در متن}
مثالی از نوشتن کد مطلب درون یک نوشتار:

\begin{latin}
\lstinputlisting[language=Matlab]{Code/code3.m}
\end{latin}

در این مثال یک کد \lr{MATLAB} دیگر وارد می کنیم، با این تفاوت که می خواهیم یکسری از کلمات کلیدی را مشخص کنیم که لاتک آن ها را با رنگی به خصوصی نشان دهد. 
\begin{latin}
\lstset{emph={binornd},emphstyle=\color{Magenta}}
\lstinputlisting[language=Matlab, morekeywords={ksdensity}]{Code/prog3.m}
\end{latin}

مثالی دیگر از نوشتن کد مطلب در یک نوشتار. فقط در این حالت می خواهیم برخی از تنظیمات پیش فرض را که قبل از شروع نوشتار تعیین کرده ایم، تغییر دهیم. 
\begin{latin}
\lstinputlisting[numbers=right,language=Matlab, framexleftmargin=5mm, frame=shadowbox,rulesepcolor=\color{Yellow}]{Code/code4.m}
\end{latin}

در ضمن شما می توانید حتی در خود همین نوشتار اصلی خود کد مورد نظرتان را بنویسید. 
\begin{latin}
\begin{lstlisting}[mathescape=true]
// calculate  $a_{ij}$
$a_{ij} = a_{jj}/a_{ij} + \alpha$;
\end{lstlisting}
\end{latin}