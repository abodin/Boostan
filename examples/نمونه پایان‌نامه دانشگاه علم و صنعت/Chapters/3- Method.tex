\chapter{طرح پیشنهادی}\label{Chapter3}
در این فصل باتوجه‌به مفاهیم و کارهای مرتبط بررسی شده در فصل اول و دوم، دررابطه‌با روش استفاده از پهپاد و چالش‌های آن توضیحاتی می‌دهیم، این توضیحات شامل طرح مسئله، چالش‌های این مسئله و روش حل این چالش‌ها بیان شده است.
\section{مقدمه}
مفهوم علمی مدیریت تحرک در پهپادها به تکنیک‌ها و راهبُردهایی اشاره دارد که برای واپایش موثر حرکت و موقعیت‌یابی پهپادها در کاربردهای مختلف استفاده می‌شود. مدیریت تحرک برای اطمینان از عملکرد کارآمد و ایمن پهپادها بسیار مهم است، به‌ویژه در سناریوهایی که پهپادها برای کارهایی مانند جمع‌آوری داده، نظارت، تحویل یا ارتباط مستقر می‌شوند. به‌عبارت‌دیگر برای مدیریت بهتر پهپاد در یک محیط به‌منظور پایش و نظارت بر آن محیط یا در سناریوهای دیگر به‌منظور برقرارکننده ارتباط در بلایای طبیعی یا غیرطبیعی اشاره کرد، این مدیریت مجموعه‌ای از روندها را مورداستفاده قرار می‌دهد. برای تحرک بهتر، مدیریت تحرک بهتر است موقعیت‌یابی بهینه‌ای در نظر داشت بر اساس موقعیت پهپاد در لحظه اول در مقایسه با لحظه بعدی آن به‌منظور پوشش‌دهی کاربران، می‌توان علاوه بر موقعیت‌یابی بهینه به مدل تحرک با مدیریت دست‌یافت. بعضی از روش‌ها و الگوهای رایج تحرک‌پذیری شامل: 
\begin{itemize}
	\item \lr{RW}
	\item \lr{RWP}
	\item \lr{GMD}
	\item \lr{BMM}
\end{itemize}
الگوریتم مورد استفاده در این روش \lr{JAYA} است، الگوریتم \lr{JAYA} یک الگوریتم بهینه‌سازی فراابتکاری است که توسط پروفسور \lr{Ravipudi Venkata Rao} در سال ۲۰۱۶ معرفی شده است. این الگوریتم برای حل مسائل بهینه‌سازی مختلف استفاده می‌شود و دارای مزایای خاصی مانند سادگی و عدم نیاز به تنظیم پارامترهای کنترلی پیچیده است. 
\section{طرح مسئله}
روش‌های تحرک رایج مانند \lr{RW} و \lr{RWP} نمی‌تواند بهینه بودن در خدمت‌رسانی را تضمین کند، باتوجه‌به رویکرد تحرک این مدل‌ها که از الگوهایی پیش‌فرض پیروی می‌کنند، نمی‌تواند مدل تحرک مناسبی برای استفاده در یک سناریو استفاده از پهپاد باشد. استقرار بهبودیافته به همراه تحرک وقف پذیر یکی از شرایط مدیریت تحرک پهپادها است، ازاین‌رو، راه‌حل ارایه شده می‌تواند تحرک مناسبی را تضمین کند.


یکی از سناریوهای مهم که نیازمند سازگاری، اتصال پایدار و حداقل تأخیر در شبکه‌های سلولی \lr{5G} و فراتر آن در هنگام بلایای طبیعی، به‌مانند طوفان‌های شدید، سیل آب‌ها و موارد مشابه است، در چنین شرایطی انگیزه استفاده از پهپادها به‌عنوان سلول‌های کوچک به‌عنوان زیرساخت \lr{5G} و فراتر از آن عمل می‌کنند، به‌ویژه در مناطقی که آسیب‌دیده‌اند و آسیب‌دیدگی به تجهیزات ارتباطی یا برقی شهری یا ناحیه جغرافیایی همان‌طور که در \autoref{fig:sysmodel} نشان‌داده‌شده است، می‌تواند مورد استفاده قرار گیرد. در چنین شرایطی \lr{UAV-BS${i}$} قابل‌استقرار در $ (x_{i,t}^ \mathrm{ \mathrm{{UAV-BS}}}, y_{i,t}^ \mathrm{ \mathrm{{UAV-BS}}}, h_{i,t}^ \mathrm{ \mathrm{{UAV-BS}}}) $ در زمان $t$. این تحرک‌پذیری به‌منظور افزایش تلاش‌های امدادی و تقویت ظرفیت‌های ارتباط رادیویی برای \lr{UE$_{j}$ } در $ (x_{j,t}^{UE}, y_{j,t}^{UE}) $ در شعاع \lr{UAV-BS${i}$} که با $R_{t}^ \mathrm{ \mathrm{{UAV-BS}}}$ در زمان $t$ اندازه‌گیری می‌شود. به‌منظور قرارگیری بهینه مختصات نیز با استفاده از $ (x_{i,t}^ \mathrm{ \mathrm{{UAV-BS}}}, y_{i,t}^ \mathrm{ \mathrm{{UAV-BS}}}, h_{i,t}^ \mathrm{ \mathrm{{UAV-BS}}}) $ تعیین می‌شود. علاوه برای این مدل متغیرهای ویژه‌ای $\theta_{i,j,t}^{\mathrm{UE}}$ که نشان‌دهنده زاویه ارتفاع بین \lr{UAV-BS$_{i}$} و \lr{UE$_{j}$} در زمان $t$ است، همچنین $dist_{i,j,t}^{UAV-UE}$ نشان‌دهنده فاصله بین \lr{UE$_{j}$} و \lr{UAV-BS$_{i}$} در زمان $t$ بر روی محور x است. به دلیل اینکه در این مدل از \lr{BS}های رایج چه به‌عنوان سرویس‌دهنده و یا استفاده از آن به‌عنوان یک راه ارتباطی بین پهپادها استفاده نشده، در این روش \lr{UAV-BS$_{i}$} مستقیم با \lr{UE$_{j}$} در زمان $t$ ارتباط دارند. در ادامه این روش بیشتر شرح داده می‌شود، به طور خاص این روش پیشنهادی از موقعیت‌یابی بهینه و قرارگیری بهینه پهپادها در نتیجه مدیریت تحرک به جهت سرویس‌دهی مناسب و با حداقل اتلاف انرژی به‌عنوان راهکاری برای تحرک‌پذیری بهینه پهپادها در سناریوهای مختلف استفاده شده است.
 
 
 
\begin{figure}
\includegraphics[width=0.8\textwidth]{SystemModel.pdf}
\caption[\lofimage{SystemModel.pdf}%
مدل پیشنهادی استقرار پهپاد‌ها به عنوان زیر ساخت ارتباطی درشبکه سلولی]{مدل پیشنهادی استقرار پهپاد‌ها به عنوان زیر ساخت ارتباطی درشبکه سلولی}
\label{fig:sysmodel}
\end{figure} 
 
 مدل شبکه‌ای شامل چندین \lr{UAV-BS} در ارتفاع پایین و گروهی از \lr{UE}ها در محیطی است که برای شبیه‌سازی یک شبکه \lr{Ad-Hoc} پروازی پویا طراحی شده است. الگوی تحرک \lr{UE} توسط مدل تحرک \lr{RW} واپایش می‌شود که با محدوده سرعت متغیر مشخص می‌شود و الگوهای حرکتی غیرقابل‌پیش‌بینی شبیه به سناریوهای دنیای واقعی را منعکس می‌کند. مفروضات کار پیشنهادی شامل موارد زیر است: 
 \begin{itemize}
 	\item 
 	محیط عملیاتی \lr{UAV-BS} در مساحت دو کیلومترمربع است.
 	\item
 	محدوده متنوع ارتباطی \lr{UAV-BS} با $R_{t}^ \mathrm{ \mathrm{{UAV-BS}}}$ 
 	\item 
 	نرخ داده موردنیاز برای \lr{UE}ها در هر مرحله زمانی متفاوت است که به طور تصادفی برای هر \lr{UE} تعیین شده است.
 	\item 
 	برای هر گره (شامل \lr{UAV-BS}ها و \lr{UE}ها) یک شناسه منحصربه‌فرد برای تمایز شبکه اختصاص‌داده‌شده است.
 	\item 
 	همه تحرک‌های \lr{UE} بر اساس مدل حرکتی \lr{RW}، با سرعت‌های ۵ تا ۱۰۰ کیلومتر در ساعت انجام شده است.
 	\item 
 	تنظیمات ارتباطی به‌صورت ارتباط مستقیم تک هاپ بین \lr{UE} و \lr{UAV-BS} مفروض است.
 	\item
 	فرض بر این است که سرعت عمودی \lr{UAV-BS} تحت‌تأثیر سرعت و جهت باد قرار نمی‌گیرد.
 	\item
 	ایستگاه‌های پایه زمینی و پیوندهای بک‌هال بین \lr{UAV-BS} و \lr{BS} در این زمینه در نظر گرفته نمی‌شوند.
 \end{itemize}
 هدف مدل بهینه‌سازی دستیابی به دو هدف اصلی است. هدف اولیه به‌حداقل‌رساندن مصرف انرژی در تحرک و ارتباطات ($P_{i,t}^{\mathrm{Total}}$) همچنین پوشش‌دهی حداکثری \lr{UE}ها ($I_{i,j,t}^{\mathrm{UE}}$) در مدل مدیریت تحرک پیشنهادی است.
 \begin{table*}%[]
\caption{خلاصه ای از مجموعه ها، پارامترها و متغیرهای به کار رفته در مدل بهینه سازی ریاضی پیشنهادی}
%\begin{tabular}{|lll|}
\begin{adjustbox}{width=\columnwidth,center}
	\begin{latin}
\begin{tabular}{|lll|}
\hline
\multicolumn{1}{|l|}{\textbf{Abbreviation}} & \multicolumn{1}{l|}{\textbf{Description}} & \textbf{Values} \\ \hline
\multicolumn{3}{|l|}{\textbf{Sets}}                                                                       \\ \hline
\multicolumn{1}{|l|}{$I$}                     & \multicolumn{1}{l|}{Set of UAV-BSs}          & \{2, 3 , 5, 10\}    \\ 
\multicolumn{1}{|l|}{$J$}                     & \multicolumn{1}{l|}{Set of UEs}         & \{50, 75, 150\}   \\ 
\multicolumn{1}{|l|}{$T$}                     & \multicolumn{1}{l|}{Set of  Times}        & \{10\} \\ \hline
\multicolumn{3}{|l|}{\textbf{Scalars}}                                                                    \\  \hline
\multicolumn{1}{|l|}{$w_{1}/w_{2}$}                      & \multicolumn{1}{l|}{Weight coefficients $(w_{1}+w_{2}=1)$}                     &             {0.5, 0.5}    \\ 
\multicolumn{1}{|l|}{$dt$}                      & \multicolumn{1}{l|}{Time step}                     &     1\            \\ 
\multicolumn{1}{|l|}{$P_{c}$}                      & \multicolumn{1}{l|}{Circuit power	}                     &  56W   \            \\ 

\multicolumn{1}{|l|}{$P_{0}$}                      & 
\multicolumn{1}{l|}{Constant $P_{0}\overset{\underset{\mathrm{\blacktriangle}}{}}{=}\frac{\delta}{8}ps\Omega _{}^{3}R_{}^{3}$}                     &                 \\ 
\multicolumn{1}{|l|}{$P$}                      & \multicolumn{1}{l|}{Transmit power	}                     & 38 dBm\            \\ 
\multicolumn{1}{|l|}{$U_{tip}$}                      & \multicolumn{1}{l|}{Tip speed of the rotor blade, $U_{tip}\overset{\underset{\mathrm{\blacktriangle}}{}}{=}\Omega R$}                     &      200\           \\ 
\multicolumn{1}{|l|}{$V_{0}$}                      & \multicolumn{1}{l|}{Mean rotor induced velocity in hover, with $v_{0}\overset{\underset{\mathrm{\blacktriangle}}{}}{=}\sqrt{\frac{W}{2\rho A}}$}                     &         7.2\        \\ 
\multicolumn{1}{|l|}{$\rho$}                      & \multicolumn{1}{l|}{Air density}                     &       1.225 $kg/m^{3}$\          \\ 
\multicolumn{1}{|l|}{ $\tau _{t}$}                      & \multicolumn{1}{l|}{Normalized traffic load}                     &    1\            \\ 


\multicolumn{1}{|l|}{$\eta $}                      & \multicolumn{1}{l|}{Amplifier Efficiency}                     &  2.6\            \\ 


\multicolumn{1}{|l|}{$\Omega $}                      & \multicolumn{1}{l|}{Blade angular velocity}                     &      400 $rad/s$\           \\ 
\multicolumn{1}{|l|}{$R$}                      & \multicolumn{1}{l|}{Rotor radius}                     &       0.5$m$\          \\ 
\multicolumn{1}{|l|}{$A$}                      & \multicolumn{1}{l|}{Rotor disc area, $A\overset{\underset{\mathrm{\blacktriangle}}{}}{=}\Pi R_{}^{2}$}                     &       0.79 $m^{2}$\          \\ 
\multicolumn{1}{|l|}{$W$}                      & \multicolumn{1}{l|}{UAV-BS weight}                     &       100$N$\          \\ 
\multicolumn{1}{|l|}{$T$}                      & \multicolumn{1}{l|}{The rotor thrust}                     & 100                \\ 
\multicolumn{1}{|l|}{$c$}                      & \multicolumn{1}{l|}{Blade length}                     &        0.0196\         \\ 
\multicolumn{1}{|l|}{$C_{r}$}                      & \multicolumn{1}{l|}{thrust coefficient}                     &   0.001195              \\ 
\multicolumn{1}{|l|}{$s$}                      & \multicolumn{1}{l|}{Rotor solidity, defined as the ratio of the total blade area (bcR) to the disc area A, or $s\overset{\underset{\mathrm{\blacktriangle}}{}}{=}\frac{bc}{\Pi R}$}                     &        0.05\         \\ 
\multicolumn{1}{|l|}{$x_{}^{\mathrm{Min}}/x_{}^{\mathrm{Max}}$}                      & \multicolumn{1}{l|}{Minimum and maximum x-coordinate of the area}                     &            0-2000m\     \\ 
\multicolumn{1}{|l|}{$y_{}^{\mathrm{Min}}/y_{}^{\mathrm{Max}}$}                      & \multicolumn{1}{l|}{Minimum and maximum y-coordinate of the area}                     &         0-2000m\        \\ 
\multicolumn{1}{|l|}{$h_{}^{\mathrm{Min}}/h_{}^{\mathrm{Max}}$}                      & \multicolumn{1}{l|}{Minimum and maximum h-coordinate of the area}                     &          0-1000m\                   \\ 
\multicolumn{1}{|l|}{$S_{FP}$}                      & \multicolumn{1}{l|}{Fuselage equivalent flat plate area in horizontal status}                     &          0. 0118 $m^{2}$\       \\ 
\multicolumn{1}{|l|}{$\overset{\sim}{k}$}                      & \multicolumn{1}{l|}{Thrust-to-weight ratio, $\overset{\sim}{k} =\frac{T}{W}\simeq 1$}                     &         1\        \\ \hline
\multicolumn{3}{|l|}{\textbf{Parameters}}                                 \\  \hline
\multicolumn{1}{|l|}{$V_{t}^{\mathrm{Wind}}$}                      & \multicolumn{1}{l|}{Wind speed}                     &    -             \\ 
\multicolumn{1}{|l|}{$\theta_{t}^{\mathrm{Wind}}$}                      & \multicolumn{1}{l|}{Speed angle}                     &  -              \\ 
\multicolumn{1}{|l|}{$x_{i,t}^{\mathrm{UAV-BS,initial}}$}                      & \multicolumn{1}{l|}{Initial x-coordinate of the UAV-BS$_{i}$}                     &  -               \\ 
\multicolumn{1}{|l|}{$y_{i,t}^{\mathrm{UAV-BS,initial}}$}                      &  \multicolumn{1}{l|}{Initial y-coordinate of the UAV-BS$_{i}$}                     &  -               \\  
\multicolumn{1}{|l|}{$h_{i,t}^{\mathrm{UAV-BS,initial}}$}                      & \multicolumn{1}{l|}{Initial altitude of the UAV-BS$_{i}$}                     &    -             \\ 
\multicolumn{1}{|l|}{$P_{i}$}                      & \multicolumn{1}{l|}{Constant $P_{i}\overset{\underset{\mathrm{\blacktriangle}}{}}{=}(1+k)\frac{W_{}^{\frac{3}{2}}}{\sqrt{2\rho A}}$}                     &  -               \\ 
\multicolumn{1}{|l|}{$B_{i,j}$}                      & \multicolumn{1}{l|}{Sub-channel bandwidth allocated to UE$_{j}$ by UAV-BS$_{i}$ }                     &   -              \\ 
\multicolumn{1}{|l|}{$P_{i,j}$}                      & \multicolumn{1}{l|}{Transmit power allocated to UE$_{j}$ by UAV-BS$_{i}$ }                     &     -            \\ 

\multicolumn{1}{|l|}{$R_{t}^{\mathrm{UAV-BS}}$}                      & \multicolumn{1}{l|}{The coverage radius of the UAVs}                     &     400-500m            \\ 



\multicolumn{1}{|l|}{$x_{j,t}^{\mathrm{UE}}$}                      & \multicolumn{1}{l|}{The x-coordinate of the UE$_{j}$}                     &     0-2000m           \\ 


\multicolumn{1}{|l|}{$y_{j,t}^{\mathrm{UE}}$}                      & \multicolumn{1}{l|}{The y-coordinate of the UE$_{j}$}                     &     0-2000m           \\ \hline


\multicolumn{3}{|l|}{\textbf{Variables}}                                   \\  \hline
\multicolumn{1}{|l|}{$V_{i,t}^{\mathrm{Vertical}}$}                      & \multicolumn{1}{l|}{Vertical speed of the UAV-BS$_{i}$ at time $t$}                     &  -               \\ 
\multicolumn{1}{|l|}{$V_{i,t}^{\mathrm{Horizontal}}$}                      & \multicolumn{1}{l|}{Horizontal speed of the UAV-BS$_{i}$ at time $t$, considering wind speed and direction}                     &     -            \\ 
\multicolumn{1}{|l|}{$P_{i,t}^{\mathrm{Vertical}}$}                      & \multicolumn{1}{l|}{Vertical power consumption of the UAV-BS$_{i}$ at time $t$}                     &  -               \\ 
\multicolumn{1}{|l|}{$P_{i,t}^{\mathrm{Horizontal}}$}                      & \multicolumn{1}{l|}{Horizontal power consumption of the UAV-BS$_{i}$ at time $t$}                     &     -            \\ 
\multicolumn{1}{|l|}{$V_{i,t}^{X}$}                      & \multicolumn{1}{l|}{Horizontal speed of the UAV-BS$_{i}$ at time $t$ in X-axis, considering wind speed}                     & -                \\ 
\multicolumn{1}{|l|}{$V_{i,t}^{Y}$}                      & \multicolumn{1}{l|}{Horizontal speed of the UAV-BS$_{i}$ at time $t$ in Y-axis, considering wind speed}                     &  -               \\ 
\multicolumn{1}{|l|}{$x_{i,t}^{\mathrm{UAV-BS}}$}                      & \multicolumn{1}{l|}{X-coordinate of the UAV-BS$_{i}$ at time $t$}                     &    -             \\ 
\multicolumn{1}{|l|}{$y_{i,t}^{\mathrm{UAV-BS}}$}                      & \multicolumn{1}{l|}{Y-coordinate of the UAV-BS$_{i}$ at time $t$}                     &  -               \\ 

\multicolumn{1}{|l|}{$h_{i,t}^{\mathrm{UAV-BS}}$}                      & \multicolumn{1}{l|}{The altitude of the UAV-BS$_{i}$ at time $t$}                     &    -             \\ 
\multicolumn{1}{|l|}{$P_{i,t}^{\mathrm{Total}}$}                      & \multicolumn{1}{l|}{Total power consumption of the UAV-BS$_{i}$ at time $t$}                     &  -               \\ 
\multicolumn{1}{|l|}{$dist_{i,j,t}^{\mathrm{UAV-UE}}$}                      & \multicolumn{1}{l|}{Distance between UE$_{j}$ and the projection of UAV-BS$_{i}$ on X-axis at time $t$}                     &    -             \\ 

\multicolumn{1}{|l|}{$\theta_{i,j,t}^{\mathrm{UE}}$}                      & \multicolumn{1}{l|}{The elevation angle between UAV-BS$_{i}$ and UE$_{j}$ at time $t$}                     &  -               \\ 

\multicolumn{1}{|l|}{$UE_{i,j,t}^{\mathrm{UE}}$}                & \multicolumn{1}{l|}{The data rate requirement of UE$_{j}$ at time $t$}   &  -       \\ \hline 

\multicolumn{3}{|l|}{\textbf{Binary Variables}}                     \\  \hline
\multicolumn{1}{|l|}{$I_{i,t}^{\mathrm{UAV-BS}}$}                      & \multicolumn{1}{l|}{If UAV-BS$_{i}$ is located at location (x, y, h) at time $t$ }                     &        \{0,1\}       \\ 
\multicolumn{1}{|l|}{$I_{i,j,t}^{\mathrm{UE}}$}                      & \multicolumn{1}{l|}{If UE$_{j}$ at location (x, y) is associated to UAV-BS$_{i}$ at time $t$}                     &        \{0,1\}         \\ \hline
\end{tabular}
\label{tab:tbl_formulatio}
\end{latin}
\end{adjustbox}
\end{table*}
 فهرست نمادهای استفاده شده در این پژوهش و فرمول بهینه‌سازی مسئله در \autoref{tab:tbl_formulatio} نشان‌داده‌شده است.
\begin{align}
f_{1}&=\sum_{i}^{I}\sum_{t}^{T}  \mathrm{P_{i,t}^{Total}} , \forall i, t \label{eq:obj1} \\
f_{2}&=-\sum_{i}^{I} \sum_{j}^{J}\sum_{t}^{T} I_{i,j,t}^{ \mathrm{UE}} , \forall i, j, t \label{eq:obj2}\\
\mathrm{min} f&=w_{1} \cdot f_{1}+w_2 \cdot f_{2} \label{eq:min}\\
w_{1}+w_{2}&=1 \label{eq:wheights}
\end{align}  
تابع $f_{1}$ که در (\autoref{eq:obj1}) نشان‌داده‌شده، مجموع مصرف انرژی هر \lr{UAV-BS} را در تمام مراحل زمانی محاسبه می‌کند. هدف این تابع به‌حداقل‌رساندن توان مصرفی \lr{UAV-BS} در طول زمان مشخص شده را در بر می‌گیرد. علامت $\sum_{i}^{I}$ بیانگر مجموع پهپادهای $I$ است. به طور مشابه، $\sum_{t}^{T}$ نشان‌دهنده یک جمع کل است. مراحل زمانی $T$. $P_{i,t}^{Total}$ نشان‌دهنده کل مصرف انرژی UAV-BS$_{i}$ در مرحله زمانی $t$ است. تابع $f_{2}$ در (\autoref{eq:obj2}) تعداد کل \lr{UE}های پوشش داده نشده را نشان می‌دهد. علامت منفی قبل از جمع بیانگر این است که هدف به‌حداقل‌رساندن این متریک باهدف کاهش تعداد \lr{UE}های پوشش داده نشده باقی‌مانده ($I_{i,j,t}^{UE} = 0$) است.
تابع $f$ در (\autoref{eq:min}) که می‌توان به‌عنوان تابع اصلی مدل پیشنهادی نام برد، ترکیبی از دو تابع هدف  $f_{1}$ و $f_{2}$ را در یک تابع هدف منفرد نشان می‌دهد. $f$ با اختصاص وزن‌های $w_{1}$ و $w_{2}$ به هر تابع هدف است. در (\autoref{eq:wheights}) ثابت می‌کند که وزن‌ها به‌درستی نرمال شده‌اند، به این معنی که $w_{1}$ و $w_{2}$ باید مجموع آن ۱ شوند. 

محدودیت‌های مربوط به مسئله بهینه‌سازی پیشنهادی را می‌توان به چند دسته طبقه‌بندی کرد: محدودیت‌های مربوط به تحرک، محدودیت‌های مصرف انرژی، ازدست‌رفتن مسیر، نرخ داده \lr{UE}ها، و محدودیت‌های عملیاتی دانست.

\textbf{الف: محدودیت‌های تحرک}
درنظرگرفتن محدودیت‌های تحرک \lr{UAV-BS}ها مانند محدودکردن مرزها، تعیین نقاط شروع \lr{UAV-BS} و سرعت \lr{UAV-BS}، روشی که موفقیت مأموریت را در عین رعایت این محدودیت‌ها بهینه می‌کنند، بسیار مهم است. سرعت عمودی و افقی پهپادها تأثیر بسزایی در تحرک و عملکرد آنها دارد. این سرعت‌ها پارامترهای کلیدی هستند که بر توانایی \lr{UAV-BS} در جهت‌یابی، انجام مأموریت‌ها و انطباق با سناریوهای مختلف تأثیر می‌گذارند. علاوه بر این، تعیین مرزهای واضحی که \lr{UAV-BS} می‌توانند در آن کار کنند، برای اطمینان از ایمنی، امنیت و انطباق با مقررات ضروری است.
\begin{align}
\label{eq:vvertical}
V_{i,t}^{\mathrm{Vertical}}&=\frac{h_{i,t}^{\mathrm \mathrm{ \mathrm{{UAV-BS}}}} - h_{i,t-1}^{\mathrm \mathrm{ \mathrm{{UAV-BS}}}}}{dt},   \forall i,t \\
\label{eq:VX_speed}
V_{i,t}^{ \mathrm{X}}&=\frac{x_{i,t}^{ \mathrm \mathrm{ \mathrm{{UAV-BS}}}} - x_{i,t-1}^ \mathrm{ \mathrm{{UAV-BS}}}}{dt}+ V_{i,t}^{Wind} cos(\theta _{t}^{Wind})I_{i,t}^ \mathrm{ \mathrm{{UAV-BS}}},   \forall i,t
\end{align} 




\begin{align}
V_{i,t}^{\mathrm{Y}}&=\frac{y_{i,t}^ \mathrm{ \mathrm{ \mathrm{{UAV-BS}}}} - y_{i,t-1}^ \mathrm{ \mathrm{ \mathrm{{UAV-BS}}}}}{dt}+ V_{i,t}^\mathrm{{Wind}} sin(\theta _{t}^{\mathrm{Wind}})I_{i,t}^ \mathrm{{UAV-BS}},  \forall i,t \label{eq:VY_speed}\\
V_{i,t}^{\mathrm{Horizontal}}&= \sqrt{\left ( V_{i,t}^{\mathrm{X}} \right )^{2}+\left ( V_{i,t}^{\mathrm{Y}} \right )^{2}}, \forall i,t \label{eq:V_hor}\\
V_{i,t}&= \sqrt{\left ( V_{i,t}^\mathrm{{Vertical}} \right )^{2}+\left ( V_{i,t}^{\mathrm{Horizontal}} \right )^{2}},  \forall i,t \label{eq:V_total}\\
x_{i,t}^ \mathrm{ \mathrm{ \mathrm{{UAV-BS}}}}&= x_{i,t}^{\mathrm{UAV,Initial}} \cdot I_{i,t}^ \mathrm{{UAV-BS}} ,   \forall i,t=0 \label{eq:UAVinitiPos_x}\\
y_{i,t}^ \mathrm{ \mathrm{ \mathrm{{UAV-BS}}}}&= y_{i,t}^{\mathrm{UAV,Initial}} \cdot I_{i,t}^ \mathrm{ \mathrm{ \mathrm{{UAV-BS}}}},   \forall i,t=0 \label{eq:UAVinitiPos_y}\\
h_{i,t}^ \mathrm{ \mathrm{ \mathrm{{UAV-BS}}}}&= h_{i,t}^{\mathrm{UAV,Initial}}\cdot I_{i,t}^ \mathrm{ \mathrm{ \mathrm{{UAV-BS}}}},   \forall i,t=0 \label{eq:UAVinitiPos_h}\\
&x^{\mathrm{Min}}\leq x_{i,t}^ \mathrm{{UAV-BS}}\leq x^{\mathrm{Max}},  \forall i,t	 \label{eq:xlimit}\\
&y^{\mathrm{Min}}\leq y_{i,t}^ \mathrm{{UAV-BS}}\leq y^{\mathrm{Max}},  \forall i,t	 \label{eq:ylimit}\\
&h^{\mathrm{Min}}\leq h_{i,t}^ \mathrm{UAV-BS}\leq h^{\mathrm{Max}},  \forall i,t \label{eq:hlimit}
\end{align} 

در (\autoref{eq:vvertical}) سرعت عمودی \lr{UAV-BS$_{i}$} را در زمان $t$ در محور عمودی نشان می‌دهد، با این فرض که سرعت و جهت باد بر سرعت عمودی تأثیر نمی‌گذارد. در (\autoref{eq:VX_speed}) و (\autoref{eq:VY_speed}) سرعت افقی \lr{UAV-BS$_{i}$} را در گام زمان $t$ در محور \lr{X} و در محور \lr{Y} به ترتیب با درنظرگرفتن سرعت و جهت باد نشان می‌دهند. در (\autoref{eq:V_hor}) سرعت افقی \lr{UAV-BS$_{i}$} را در گام زمان $t$ نشان می‌دهد.

رابطه اقلیدسی بین سرعت‌های عمودی و افقی، سرعت کل \lr{UAV-BS$_{i}$} را در زمان $t$ تشکیل می‌دهد که در (\autoref{eq:V_total}) نشان‌داده‌شده است. منطق پشت (\autoref{eq:VX_speed}) به (\ref{eq:V_total}) در \autoref{fig:Vwind} نشان‌داده‌شده است. بر اساس \autoref{fig:Vwind}، سرعت \lr{UAV-BS$_{i}$} در زمان $t$ و در مختصات $ (x_{i,t}^ \mathrm{ \mathrm{ \mathrm{{UAV-BS}}}}, y_{i,t}^ \mathrm{ \mathrm{ \mathrm{{UAV-BS}}}}, h_{i,t}^ \mathrm{UAV-BS}$) در دو بعد افقی و عمودی قابل‌محاسبه است. همان‌طور که در (\autoref{eq:vvertical}) نشان‌داده‌شده است سرعت \lr{UAV-BS} در طول پرواز عمودی معادل مسافت طی شده در امتداد محور عمودی ($h$) است. بر اساس \autoref{fig:Vwind} برای پرواز افقی، می‌توان حرکت یک \lr{UAV-BS} را روی یک صفحه دوبعدی طراحی کرد تا سرعت \lr{UAV-BS} را در امتداد محور \lr{x} و همچنین محور \lr{y} در (\autoref{eq:VY_speed})، با درنظرگرفتن تأثیر سرعت باد (($V_{i,t}^{\mathrm{Wind}}$) و جهت ($\theta) {t}^{\mathrm{wind}}$) محاسبه کند. در نهایت، در (\autoref{eq:Vhor}) سرعت افقی \lr{UAV-BS} نشان‌داده‌شده است.
\begin{figure}
\includegraphics[width=0.9\textwidth]{Vwind.pdf}
\caption[\lofimage{Vwind.pdf}%
سرعت و جهت باد]{سرعت و جهت باد} %revise formulations
\label{fig:Vwind}
\end{figure}

در (\autoref{eq:UAVinitiPos_x})، (\autoref{eq:UAVinitiPos_y}) و (\autoref{eq:UAVinitiPos_h}) مقادیر اولیه مختصات \lr{UAV-BS} را تعیین می‌کند. در (\autoref{eq:xlimit})، (\autoref{eq:ylimit}) و (\autoref{eq:hlimit}) مرزهای فضایی که مختصات سه‌بعدی پهپادها باید در آن قرار گیرند مشخص می‌شود. از طرفی $x^{\mathrm{Min}}$، $y^{\mathrm{Min}}$، $h^{\mathrm{Min}}$ و $x^{\mathrm{Max}}$، $y^{\mathrm{Max}}$ حداقل و حداکثر مجاز (\lr{x})، \lr{(y)} و \lr{(h)} را برای \lr{UAV-BS} مشخص می‌کند.

\textbf{ب: محدودیت‌های انرژی مصرفی}

توان مصرفی پهپادها را می توان به سه جز اصلی تقسیم بندی کرد: \cite{Aloqaily2022}, \cite{9700536}, \cite{9708417}, \cite{Zhou2023}:
\begin{itemize}
	\item 
	\textit{توان موتور} 
	این مؤلفه که به‌عنوان نیروی محرکه نیز شناخته می‌شود اساسی‌ترین بخش مصرف انرژی در پهپادها است که شامل توان موردنیاز برای حرکات مختلف \lr{UAV-BS} از جمله حرکت افقی و عمودی، فرود، برخاستن و شناور ماندن است. ارتباط بین سرعت و زاویه باد با نیروی مصرفی پهپاد یک چالش باز است، زیرا سرعت باد در ورزش مخالف پرواز پهپاد می‌تواند باعث مصرف انرژی بیشتر شود، از طرفی در جهت موافق نیز ممکن است باعث مصرف انرژی کمتری شود، این مورد را به طور دقیق نمی‌توان مطرح کرد، ولی سرعت و زاویه باد را می‌توان همیشه جز شروط پروازی در محیط واقعی در نظر گرفت \cite{10.1007/978-3-319-99996-8_16}.
	\item 
	\textit{انرژی برقراری ارتباط} 
	انرژی مربوط به انتقال و دریافت داده در پهپادها را شامل می‌شود.
	\item 
	\textit{انرژی مربوط به پردازش}
	این عامل در حالی وجود دارد که مصرف انرژی آن‌ها حداقلی است و در مدل انرژی پیشنهادی نادیده گرفته شده است.	
	\end{itemize}
	این اجزا در مجموع مصرف انرژی \lr{UAV-BS} ها را تشکیل می‌دهند و توان موتور و انرژی مرتبط با ارتباطات از اصلی‌ترین فاکتورها هستند.


\begin{equation}
\label{eq:pHorizontal}
P_{i,t}^{\mathrm{Horizontal}}=P_{0}\left ( 1+ \frac{\left (V_{i,t}^{\mathrm{Horizontal}}  \right )^{2}}{\Omega ^{2} R^{2}} \right )+ P_{i} \widetilde{\kappa} \left (\sqrt{\widetilde{\kappa}^{2} + \frac{\left ( V_{i,t}^{\mathrm{Horizontal}} \right )^{4}}{4V_{0}^{4}}}-\frac{\left ( V_{i,t}^{\mathrm{Horizontal}} \right )^{2}}{2 V_{0}^{2}} \right )^{\frac{1}{2}}+\frac{\rho}{2} S_{FP}\left ( V_{i,t}^{\mathrm{Horizontal}} \right )^{3}, \forall i,t
\end{equation}

در (\autoref{eq:pHorizontal}) به بیان مدل مصرف انرژی \lr{UAV-BS} را در پرواز افقی نشان می‌دهد که بر اساس فرمول ارایه شده در \cite{8663615} است. در (\autoref{eq:pHorizontal})، دو عبارت اول به ترتیب نشان‌دهنده توان پروفیل تیغه و توان القایی در طول پرواز روبه‌جلو هستند. عبارت سوم به معنای توان پارازیتی است. این شرایط مشروط به‌سرعت افقی $V_{i,t}^{\mathrm{Horizontal}}$ هستند.

\begin{equation}
	\label{eq:pVertical}
	P_{i,t}^{\mathrm{Vertical}}=P_{0}+ P_{i}+ \frac{1}{2} R_{T} V_{i,t}^{\mathrm{Vertical}}+\frac{R_{T}}{2}\sqrt{\left (V_{i,t}^{\mathrm{Vertical}}  \right )^{2}+\frac{2 R_{T}}{\rho A}},  \forall i,t
\end{equation}

در (\autoref{eq:pVertical}) پرواز عمودی \lr{UAV-BS} را نشان می‌دهد که از \cite{10175052} استخراج شده است. $R_{T}$ رانش روتور را نشان می‌دهد، در حالی که سایر پارامترها با پارامترهایی که قبلا معرفی شده‌اند مطابقت دارند. نسبت رانش به وزن $\widetilde{\kappa}$ در (\autoref{eq:pHorizontal}) تقریبی به
$\frac{R_{T}}{W}\simeq 1$
 است که نشان دهنده $\left ( R_{T}\simeq W \right )$ است. با استفاده از این تقریب، در (\autoref{eq:pHorizontal}) و (\autoref{eq:pVertical}) که مصرف انرژی افقی و عمودی پهپاد را در زمان $t$ نشان می‌دهند را می‌توان به صورت \autoref{eq:pHorizontal1} و \autoref{eq:pVertical1} ساده‌سازی کرد. 
\begin{align}
	P_{i,t}^{\mathrm{Horizontal}}&=P_{0}\left ( 1+  \frac{\left (V_{i,t}^{Horizontal}  \right )^{2}}{\Omega ^{2} R^{2}} \right )+ P_{i} \left (\sqrt{\frac{\left ( V_{i,t}^{Horizontal} \right )^{4}}{4V_{0}^{4}}}-\frac{\left ( V_{i,t}^{Horizontal} \right )^{2}}{2 V_{0}^{2}} \right )^{\frac{1}{2}}+\frac{\rho}{2} S_{FP}\left ( V_{i,t}^{Horizontal} \right )^{3},  \forall i,t	 \label{eq:pHorizontal1}\\
	P_{i,t}^{\mathrm{Vertical}}&=P_{0}+ P_{i}+ \frac{1}{2} W_{i} V_{i,t}^{\mathrm{Vertical}}+\frac{W_{i}}{2}\sqrt{\left (V_{i,t}^{\mathrm{Vertical}}  \right )^{2}+\frac{2T}{\rho A}},  \forall i,t 	\label{eq:pVertical1}\\
	P_{i,t}^{\mathrm{Comm}}&=P_{c}+  \tau _{t} \eta  P,  \forall i,t	 \label{eq:comm}\\
	P_{i,t}^{\mathrm{Total}}&= P_{i,t}^{\mathrm{Horizontal}}+ P_{i,t}^{\mathrm{Vertical}}+P_{i,t}^{\mathrm{Comm}},  \forall i,t \label{eq:pTotal}
\end{align}

مصرف انرژی حاصل از ارتباطات که از \cite{ABUBAKAR2023109854} استخراج می‌شود، در (\autoref{eq:comm}) نشان‌داده‌شده است، جایی که $P_{c}$ توان مداری را نشان می‌دهد، $\tau _{t}$، $\eta$ و $P$ بار ترافیک عادی را به نشان می‌دهد، راندمان تقویت‌کننده و توان انتقالی به ترتیب. توان کل در (\autoref{eq:pTotal}) نشان‌داده‌شده است که مجموع قدرت حرکت افقی، عمودی و توان مصرفی است.

\textit{پ: محدودیت مسیر و نرخ داده}
استقرار \lr{UAV-BS} دارای پیامدهای دوگانه است که هم بر ناحیه پوشش در دسترس برای \lr{UE}ها و هم بر قابلیت اطمینان پیوندهای ارتباطی هوا به زمین (\lr{AtG}) تأثیر می‌گذارد. در شبیه‌سازی ازدست‌دادن مسیر \lr{AtG}، انواع مدل‌های کانال را می‌توان در  \cite{smith2020survey}، \cite{lee2018performance} و \cite{brown2019path} یافت. در این پژوهش، مدل کانال به‌تفصیل در \cite{9112268} به طور خاص به دلیل سابقه عملکرد اثبات شده آن انتخاب شده است.

بسته به شرایط انتشار، پیوندهای ارتباطی \lr{AtG} را می‌توان به‌عنوان خط دید مستقیم (\lr{LoS}) یا دید غیرمستقیم (\lr{NLoS}) طبقه‌بندی کرد. احتمال برقراری ارتباط \lr{LoS} بین گیرنده و فرستنده اهمیت قابل‌توجهی دارد؛ زیرا مستقیماً استفاده از توان توسط \lr{UE}ها را شکل می‌دهد. احتمال دستیابی به اتصال \lr{LoS} بین \lr{UE} و \lr{UAV-BS} به عواملی مانند تراکم ساختمان، موقعیت جغرافیایی \lr{UE} و زاویه ارتفاع \lr{UAV-BS} دررابطه‌با \lr{UE} و پهپاد بستگی دارد. برای محاسبه احتمالات اتصالات \lr{LoS} و \lr{NLoS} بین \lr{UE$_{j}$} و \lr{UAV-BS$_{i}$}، محاسبات از روش مشخص شده در \cite{9112268} پیروی می‌کند.

نرخ داده بین \lr{UE$_{j}$} و \lr{UAV-BS$_{i}$} را می‌توان بر اساس مدل ازدست‌رفتن مسیر ارایه شده در \cite{9112268} به‌صورت \autoref{eq:Rij} محاسبه کرد.
	\begin{equation}
		\label{eq:Rij}
		DR _{i,j,t}^{\mathrm{UE}} =B_{ij} \times \log_{2}\left ( 1+ \frac{p_{ij} \times 10^{\frac{-\zeta  _{ij}}{10}}}{B_{ij} \times \varepsilon ^{2}}\right ),  \forall i,j,t
	\end{equation}
	
	\textbf{ت: محدودیت های عملیاتی}
با استفاده از فرمول‌های عملیاتی، تخصیص \lr{UE}ها بر اساس نزدیکی آن‌ها به حداکثر یک \lr{UAV-BS} در هر مرحله زمانی محدود می‌شود تا از ارتباطات اطمینان حاصل شود. نرخ کل داده یا منابع موردنیاز \lr{UE}های اختصاص‌داده‌شده به یک \lr{UAV-BS} تضمین می‌کند که از ظرفیت یا قابلیت‌های \lr{UAV-BS} تجاوز نکند. همچنین جلوگیری از برخورد بین \lr{UAV-BS} با نظارت و حفظ حداقل فاصله ایمن بین آنها تضمین می‌شود.
\begin{equation}
	\label{eq:association1}
	\left ( x_{i,t}^ \mathrm{ \mathrm{ \mathrm{{UAV-BS}}}} - x_{j,t}^{\mathrm{UE}}\right )^{2} + \left ( y_{i,t}^ \mathrm{ \mathrm{ \mathrm{{UAV-BS}}}} - y_{j,t}^{\mathrm{UE}}\right )^{2} \leq \left ( R_{t}^ \mathrm{ \mathrm{ \mathrm{{UAV-BS}}}} \right )^{2} +\left ( \left ( x^{\mathrm{Max}}-x^{\mathrm{Min}} \right )^{2}+ \left ( y^{Max}-y^{Min} \right )^{2}+ \left ( h^{\mathrm{Max}}-h^{\mathrm{Min}} \right )^{2} \right )\cdot \left ( 1-I_{i,j, t}^{\mathrm{UE}} \right ),  \forall i,j,t
\end{equation}




\begin{equation}
	\label{eq:association2}
	\sum_{I}^{i} \sum_{J}^{j}\sum_{T}^{t} I_{{i,j,t}}^{\mathrm{UE}}\leq 1,  \forall i,j,t
\end{equation}
\begin{equation}
	\label{eq:UERate}
	\sum_{i}^{I} \sum_{j}^{J}\sum_{t}^{T} DR_{i,j,t}^{\mathrm{UE}}\leq Capacity_{i}^ \mathrm{{UAV-BS}}, \forall i,j,t
\end{equation}
\begin{equation}
	\label{eq:collAv}
	\left ( x_{i,t}^ \mathrm{{UAV-BS}} - x_{k,t}^ \mathrm{UAV-BS}\right )^{2} + \left ( y_{i,t}^ \mathrm{ \mathrm{{UAV-BS}}} - y_{k,t}^ \mathrm{ \mathrm{{UAV-BS}}}\right )^{2} +  \left ( h_{i,t}^ \mathrm{ \mathrm{{UAV-BS}}} - h_{k,t}^ \mathrm{ \mathrm{{UAV-BS}}}\right )^{2}\geq D^{\mathrm{Crash}}, \forall i,k,t
\end{equation}

در (\autoref{eq:association1}) نشان‌داده‌شده که \lr{UE$_{j}$} باید در محدوده پوشش \lr{UAV-BS$_{i}$} باشد. به‌عبارت‌دیگر، فاصله \lr{2D} بین \lr{UE$_{j}$} و پیش‌بینی \lr{UAV-BS$_{i}$} در محور \lr{x} در زمان $t$ باید برابر یا کمتر از محدوده پوشش و مرزهای منطقه‌ای پهپاد باشد، در (\autoref{eq:association2}) نشان‌داده‌شده که هر \lr{UE$_{j}$} باید حداکثر توسط یک پهپاد در محل $i$ خدمت‌رسانی شود.
در (\autoref{eq:UERate}) نشان داده کل نرخ داده همه \lr{UE}ها توسط یک \lr{UAV-BS$_{i}$} (ظرفیت پیوند بی‌سیم بین \lr{UAV-BS$_{i}$} و \lr{UE$_{j}$}) در زمان $t$ که نمی‌تواند از حداکثر ظرفیت نرخ داده \lr{UAV-BS} فراتر رود). در \autoref{eq:collAv} برای جلوگیری از برخورد بین \lr{UAV-BS}ها در هنگام حرکت، فاصله بین هر جفت پهپاد ($i$ ، $k$) در زمان $t$، باید بیش از یک آستانه مشخص باشد.

\section{مشکلات مسئله}
یکی از پرسش‌های اصلی که این مدل به دنبال پاسخ‌دادن به آن است، استفاده از روش مناسب تحرک برای خدمت‌رسانی به کاربران زمینی است، در حقیقت مشکل اساسی مدل‌های رایج در تحرک بر اساس الگوی از پیش تعیین شده است، مدل‌های رایج ایراداتی را در ارایه خدمات دارد، چرا که پهپادها وابسته به باتری است، و این انرژی وابسته به مدل آن محدود است، پس می‌توان این نتیجه را گرفت که مدل‌هایی به‌مانند \lr{RW} و \lr{RWP} باتوجه‌به اینکه الگویی از پیش تعیین شده دارد نمی‌تواند مدل تحرک مناسبی برای مدیریت بهینه تحرک پهپادها باشد؛ بنابراین مدل بهینه ارایه شده تضمین مناسبی برای خدمت‌رسانی می‌دهد، انرژی تحرک و اتصال پایین‌تر می‌تواند ضامن ارایه خدمات بهتر در یک مأموریت پایش یا نظارت بر محیط باشد.

\section{حل مشکلات}
باتوجه‌به مشکلات مطرح شده، و اهمیت تحرک بهبودیافته پهپادها روش ارایه شده می‌تواند یک راهکار امیدوارکننده به‌منظور پایش و نظارت بر یک محیط شهری یا غیرشهری باشد، امیدوار هستیم مدل مذکور بتواند در مواقع بلایای طبیعی یا غیرطبیعی نیز کارآمد باشد، زیر به دلیل ماهیت فیزیکی پهپاد که می‌توان حتی مناطق سخت‌گذر را نیز پایش کند، راه‌حل مناسبی برای این دست از عملیات نیز است.
یکی از مهم‌ترین پرسش‌هایی که این روش به آن پاسخ می‌دهد، روش تحرک بهبودیافته بر اساس الگوریتم است که می‌تواند عملکرد مناسب‌تری در شرایط مختلف محیطی را داشته باشد.
%\input{Algorithm/algorithm}

\begin{table*}%[]
\centering
\caption{پارامترهای تولید داده}
\begin{tabular}{cp{9.5cm}}
\hline
پارامترهای داده & توضیحات \\ \hline
     $V_{t}^{\mathrm{wind}}$          &       سرعت باد مقداری بین ۱ تا ۲۱ کیلومتر        \\ 
     $\theta_{t}^{\mathrm{wind}}$     &        زاویه وزش باد مقداری بین ۰ تا ۳۶۰ درجه\\
     $Capacity_{t}^{\mathrm{UAV-BS}}$    &    ظرفیت هر پهپاد مقداری بین ۴۰۰ تا ۶۰۰ مگابایت\\
     $R_{t}^{\mathrm{UAV-BS}}$      &      شعاع تحت پوشش پهپاد مقداری بین ۴۰۰ تا ۵۰۰ متر\\
     $DR_{i,j,t}^{\mathrm{UE}}$       &      داده مورد نیاز هر کاربر مقداری بین ۱۰۰ تا ۲۰۰ مگابایت\\
     $X_{j,t}^{\mathrm{UE}}$          &      مختصات محور$x$ کاربران مقداری بین ۰ تا ۲۰۰۰ متر \\
     $Y_{j,t}^{\mathrm{UE}}$          &      مختصات محور $y$ کاربران مقداری بین ۰ تا ۲۰۰۰ متر\\
     $Id_{j,t}^{\mathrm{UAV-BS}}$        &      شناسه پهپاد‌ها\\
     $Id_{j}^{\mathrm{UE}}$           &      شناسه کاربران \\ 
     $I,J,T$                 &   مجموعه پهپاد، کاربران و زمان\\
     $x^{\mathrm{Min}},x^{\mathrm{Max}}$, $y^{\mathrm{Min}},y^{\mathrm{Max}}$, $h^{\mathrm{Min}},h^{\mathrm{Max}}$ &  حداقل و حداکثر محدوده مجاز مساحت که برای همه آزمایش ها یکسان است \\ \hline
\end{tabular}
\label{tab:gen}
\end{table*}

یکی از چالش‌های مدل‌سازی عدم وجود مجموعه‌داده‌های متناسب با این مدل‌سازی بوده است، یکی از کارهای مهم انجام شده در خلل این مدل‌سازی ایجاد مجموعه‌داده مرتبط برای مدل‌سازی بود که یکی از نیازهای اساسی مدل‌سازی در این حوزه است. زیرا که داده‌ها در استفاده از پهپاد متفاوت با گره‌های زمینی است، گره‌های هوایی که به‌عنوان آنتن‌های متحرک در حال ارایه خدمات هستند، شامل پارامترهای دیگری هستند که می‌تواند در ارایه خدمات پهپادها تأثیرگذار باشد. در جدول پارامترها قابل‌مشاهده است.
%Start Here
در (\autoref{eq:min}) یک مشکل بهینه‌سازی \lr{MINLP} مطرح شده است که قدرت حرکت و ارتباطی را که توسط \lr{UAV-BS} مصرف می‌شود، به حداقل می‌رساند، درحالی‌که تعداد \lr{UE}های پوشش داده نشده را به حداقل می‌رساند. چنین مشکلی را می‌توان با استفاده از تکنیک‌های مختلف حل کرد. رویکردهای مبتنی بر الگوریتم‌های فراابتکاری یکی از تکنیک‌های حل مسئله است که از الگوریتم‌های جستجوی تکرارشونده برای یافتن راه‌حل‌های بهینه برای مسائل بهینه‌سازی، قراردادن پهپاد استفاده می‌کنند. الگوریتم‌های ژنتیک (\lr{GA})، بهینه‌سازی اذحام ذرات (\lr{PSO})، بهینه‌ساز گرگ خاکستری (\lr{GWO})، جستجوی فاخته (\lr{CS})، بهینه‌سازی کلونی مورچه (\lr{ACO}) و \lr{JAYA} تکنیک‌های قدرتمند حل مسئله هستند که می‌توانند برای حل مسائل پهپادها استفاده شود.

برای فعال‌کردن این مدل در کاربردهای مبتنی بر پهپاد، یک الگوریتم فراابتکاری به نام \lr{JAYA} ارایه شده است. این الگوریتم با سرعت، سادگی و کارایی آن از نظر زمان محاسباتی و استفاده از حافظه شناخته می‌شود. \lr{JAYA} توانایی همگرایی سریع به سمت راه‌حل‌های جامع بهینه نشان می‌دهد. الگوریتم \lr{JAYA} پارامترهای الگوریتمی کمتری در مقایسه با برخی دیگر از تکنیک‌های بهینه‌سازی دارد و یک رفتار قطعی یا تقریباً قطعی دارد برای مسائل جاگیری و تحرک پهپادها دارد.

پس از مقداردهی اولیه پارامترهای مسئله بهینه‌سازی الگوریتم \lr{JAYA}، اولین مکان‌های بالقوه برای \lr{UAV-BS} در زمان ۰ با استفاده از خوشه‌بندی \lr{k-means} به دست خواهد آمد ($x_{i,t}^{\mathrm{UAV-BS,initial}}$, $y_{i,t}^{\mathrm{UAV-BS,initial}}$, $h_{i,t}^{\mathrm{UAV-BS,initial}}$).
در \autoref{fig:jaya} فلوچارت الگوریتم \lr{JAYA} را برای مسئله بهینه‌سازی مکان‌یابی و تحرک \lr{UAV-BS} نشان می‌دهد که یک نمای کلی جامع مطابق با الگوریتم ارایه شده در الگوریتم \ref{alg:proposedalgo} ارایه می‌دهد. پس از فرمول‌بندی مسئله مکان‌یابی بر اساس فلوچارت ارایه شده در \autoref{fig:jaya}، پارامترهایی مانند اندازه جمعیت، تعداد تکرارها، متغیرها و معیارهای پایان باید تنظیم شوند. پس از آن، راه‌حل‌هایی که به‌عنوان مختصات \lr{UAV-BS} نشان داده، در جمعیت رتبه‌بندی می‌شوند و بهترین و بدترین راه‌حل‌ها را تشخیص می‌دهند. راه‌حل‌های جدید شامل استفاده از (\autoref{eq:jayax})، (\autoref{eq:jayay}) و (\autoref{eq:jayah}) است که به‌روزرسانی‌های تکراری را برای بهینه‌سازی مختصات ($X_{i,t,New}^{\mathrm{UAV-BS}}$, $Y_{i,t,New}^{\mathrm{UAV-BS}}$ و $H_{i,t,New}^{\mathrm{UAV-BS}}$) از مکان‌های \lr{UAV-BS} در فرایند بهینه‌سازی پیشنهادی نشان می‌دهد. هر مجموعه معادله نحوه محاسبه مجدد مختصات جدید به‌منظور تحرک‌پذیری بهبودیافته (که با $New$ نشان داده می‌شود) براین‌اساس مختصات فعلی ($Current$)، بهترین مختصات ($Best$) و بدترین مختصات ($Worst$) با مشخص شده است. این فرایند شامل پیکربندی مختصات پهپاد به طور تکرارشونده تا زمانی که همگرایی یا یک معیار خاص به دست آید، این امر می‌تواند به بهبود تحرک‌پذیری کمک شایانی کند. پارامتر $r$ به‌عنوان یک عامل سنجش در این فرایند به‌روزرسانی می‌شود. اگر راه‌حل بهبودیافته پیدا شود، جایگزین راه‌حل قبلی می‌شود. در غیر این صورت، الگوریتم به راه‌حل قبلی برمی‌گردد. هنگامی که تمام راه‌حل‌های موجود ارزیابی شود، و اگر تعداد تکرار به $N$ برسد، عملیات به پایان می‌رسد. در صورت برآورده‌شدن بهینه‌ترین راه‌حل‌ها مختصات بهینه برگردانده می‌شوند. در غیر این صورت، روند با یک تکرار جدید ادامه خواهد یافت.


\begin{equation}
	\label{eq:jayax}
	X_{i,t,New}^{\mathrm{UAV-BS}} \&= X_{i,t,Current}^{\mathrm{UAV-BS}} + r \cdot (X_{i,t,Best}^{\mathrm{UAV-BS}}-\left | X_{i,t,Current}^{\mathrm{UAV-BS}} \right |) \
	\& \quad - r \cdot (X_{i,t,Worst}^{\mathrm{UAV-BS}}-\left | X_{i,t,Current}^{\mathrm{UAV-BS}} \right |)
\end{equation}

\begin{equation} \label{eq:jayay} 
	Y_{i,t,New}^{\mathrm{UAV-BS}} \&= Y_{i,t,Current}^{\mathrm{UAV-BS}} + r \cdot (Y_{i,t,Best}^{\mathrm{UAV-BS}}-\left | Y_{i,t,Current}^{\mathrm{UAV-BS}} \right |) \
	\& \quad - r \cdot (Y_{i,t,Worst}^{\mathrm{UAV-BS}}-\left | Y_{i,t,Current}^{\mathrm{UAV-BS}} \right |)
\end{equation}

\begin{equation} \label{eq:jayah}
	H_{i,t,New}^{\mathrm{UAV-BS}} \&= H_{i,t,Current}^{\mathrm{UAV-BS}} + r \cdot (H_{i,t,Best}^{\mathrm{UAV-BS}}-\left | H_{i,t,Current}^{\mathrm{UAV-BS}} \right |) \
	\& \quad - r \cdot (H_{i,t,Worst}^{\mathrm{UAV-BS}}-\left | H_{i,t,Current}^{\mathrm{UAV-BS}} \right |)
\end{equation}
الگوریتم نشان‌داده‌شده \ref{alg:proposedalgo} از الگوریتم \lr{JAYA} برای بهینه‌سازی قرارگیری و تحرک پهپادها استفاده می‌کند. همچنین به‌عنوان ورودی پارامترهای زیر را می‌گیرد: 
موقعیت \lr{UE}، نرخ داده موردنیاز \lr{UE}ها، سرعت \lr{UAV-BS} در طول پروازهای افقی و عمودی، ظرفیت \lr{UAV-BS}، شعاع پوشش، مشخصات موتور، مرزهای منطقه، سرعت باد و جهت باد. هم مسئله بهینه‌سازی و هم پارامترهای الگوریتم \lr{JAYA} را مقداردهی اولیه می‌کند. همچنین ($x_{i,t}^{\mathrm{UAV-BS,initial}}$, $y_{i,t}^{\mathrm{UAV-BS,initial}}$, $h_{i,t}^{\mathrm{UAV-BS,initial}}$) با استفاده از خوشه‌بندی\lr{K-Means}، مکان‌های بالقوه اولیه پهپادها را در زمان اولیه ۰ ایجاد می‌کند. پس از آن، توابع هدف $f_{1}$, $f_{2}$ و $f$ به ترتیب بر اساس (\autoref{eq:obj1})، (\autoref{eq:obj2}) و (\autoref{eq:min}) محاسبه می‌شوند. پس از مرتب‌سازی راه‌حل‌ها برای شناسایی بهترین و بدترین آنها با ($x_{i,t,Best}^{\mathrm{UAV-BS}}$, $y_{i,t,Best}^{\mathrm{UAV-BS}}$, $h_{i,t,Best}^{\mathrm{UAV-BS}}$ و $x_{i,t,Worst}^{\mathrm{UAV-BS}}$, $y_{i,t,Worst}^{\mathrm{UAV-BS}}$, $h_{i,t,Worst}^{\mathrm{UAV-BS}}$) الگوریتم یک حلقه را شروع می‌کند و تا حداکثر تعداد تکرار می‌شود، به‌ازای $k$ برای هر 
$i$ پهپاد
، 
$j$ \lr{UE}
و $t$ به‌عنوان گام زمانی، در صورت‌یافتن راه‌حل‌های بهبودیافته جایگزین راه‌حل‌های فعلی می‌شوند. در نهایت، در صورت رسیدن به حداکثر تکرار، مختصات \lr{UAV-BS} بهینه را متوقف می‌کند و بر می‌گرداند. در غیر این صورت، تا زمان همگرایی ادامه می‌یابد و به طور مداوم مکان‌های پهپاد را بر اساس اهداف و فرمول‌های از پیش تعریف‌شده اصلاح می‌کند.
\begin{figure}
\includegraphics [width=\textwidth] {JAYA.pdf}
\caption [\lofimage{JAYA.pdf}%
 الگوریتم جایا برای جایابی و تحرک‌پذیری پهپادها] {الگوریتم جایا برای جایابی و تحرک‌پذیری پهپادها}
\label{fig:jaya}
\end{figure}

\section{تحلیل پیچیدگی}
تجزیه‌وتحلیل پیچیدگی زمانی الگوریتم بهینه‌سازی مبتنی بر \lr{JAYA} می‌تواند چالش‌برانگیز باشد، زیرا به عوامل مختلفی مانند نوع مسئله، اندازه جمعیت، معیارهای خاتمه و پیچیدگی تابع هدف بستگی دارد.
\begin{itemize}
	\item \textit{مقداردهی اولیه متغیر}
	این بخش بر اساس داده‌های دریافتی، متغیرهای مختلف را مقداردهی می‌کند. پیچیدگی زمانی اولیه این متغیرها $O(I + J)$ است که $I$ مقدار متغیر تعداد  \lr{"UAV-BS"} و $J$ مقدار متغیر تعداد \lr{"UE"} است. با درنظرگرفتن تعداد کل عناصر پردازش شده به‌صورت $n = \max(I, J)$، پیچیدگی زمانی که به‌صورت $O(I + J)$ بیان شده است به‌صورت $O(n)$ نمایش داده می‌شود. این انتقال به $ O(n)$ نشان می‌دهد که پیچیدگی محاسباتی به بزرگی دو متغیر $I$ و $J$ بستگی دارد که توسط متغیر "$n$" تلقی شده‌اند. این نمایش تأکید می‌کند که «$n$» به‌عنوان یک معیار یکپارچه عمل می‌کند که حجم کل کار تعیین‌شده حداکثری بین $I$ و $J$ را محاسبه می‌کند.
	\item \textit{مقداردهی اولیه جمعیت}
	پیچیدگی زمانی این بخش به‌اندازه جمعیت بستگی دارد که با مقادیر \lr{I} و \lr{J} تعیین می‌شود. باتوجه‌به اینکه \lr{I} و \lr{J} نشان‌دهنده اندازه جمعیت هستند، تعداد کل عناصر پردازش شده در بهترین شرایط با $n = I \times J$ با مقدار بزرگ‌تر بین \lr{I} و \lr{J} اندازه جمعیت را تعیین می‌کند؛ بنابراین، پیچیدگی زمانی این بخش $ O(I \times J)$ است که می‌تواند با فرض $n = max(I, J)$، به‌صورت $ O(n^{2})$ نمایش داده شود.
	\item \textit{ارزیابی توابع هدف}
	این بخش توابع هدف را برای متغیرهای جمعیت ارزیابی می‌کند. پیچیدگی زمانی این بخش وابسته به‌اندازه جمعیت و محاسبات است که شامل حلقه‌های تودرتو بر روی \lr{I} و \lr{J} که منجر به پیچیدگی زمانی $O(I \times J)$ یا معادل آن به‌صورت $ O(n^{2})$ می‌شود.
	
	
	
	به‌طورکلی، پیچیدگی زمانی تحت سلطه بخش‌های اولیه سازی جمعیت و ارزیابی عملکرد است که هر دو دارای پیچیدگی زمانی $O(I \times J)$ یا $ O(n^{2})$ هستند. در مورد شدت محاسباتی کد، به مقادیر خاص $I$ و $J$ بستگی دارد. اگر این مقادیر نسبتاً کوچک باشد، کد باید از نظر محاسباتی قابل‌اجرا بر روی \lr{UAV-BS} باشد. بااین‌حال، اگر مقادیر بزرگ باشند، ممکن است کد از نظر محاسباتی فشرده شود و همچنین ممکن است به منابع محاسباتی قدرتمندتری برای اجرا در زمان معقول نیاز داشته باشد.
	
\end{itemize}
\section{ارزیابی عملکرد}
معماری شبکه پیشنهادی، مدل بهینه‌سازی و الگوریتم پیشنهادی با استفاده از زبان برنامه‌نویسی پایتون پیاده‌سازی شد. راه‌اندازی شبیه‌سازی برای مدل بهینه‌سازی پیاده‌سازی شده در پایتون شامل طراحی و ساخت یک معماری شبکه خاص با استفاده از کتابخانه بهینه‌سازی \lr{Pyomo} \cite{hart2011pyomo} است. در این چارچوب مبتنی بر پایتون، اجزای مختلفی مانند پیش‌پردازش داده‌ها، ایجاد مدل بهینه‌سازی ریاضی و الگوریتم مبتنی بر \lr{JAYA} ادغام شده‌اند.
\section{نتیجه}
پهپادها به‌عنوان ایستگاه‌های پایه هوایی متحرک، کاربردهای فراوانی در عملیات هوایی دارد، بنابراین، استفاده از روش‌های متناسب برای به‌کارگیری از آن یا به‌عبارت‌دیگر استقرار آن‌ها از اهداف این روش است، استقرار پهپادها و مکان‌یابی بهینه پهپادها راهکار امیدوارکننده به جهت کاهش انرژی و در نتیجه عملیات موفقیت‌آمیز پایش و خدمت‌رسانی در محیط است، باتوجه‌به روش پیشنهاد شده برای محیط‌های شهری و غیرشهری در صورت پیش آمد بلایای طبیعی و غیرطبیعی می‌تواند کاربردی باشد، از طرفی همان‌طور که اشاره شد، استفاده از الگوریتم‌های متناسب با شرایط می‌تواند شرایط بهتری را به‌منظور ارایه خدمات فراهم سازد.


