\chapter{مدلسازی و ارزیابی}\label{Chapter4}
در این فصل توضیحاتی پیرامون مدل‌سازی جایابی و تحرک بهینه پهپادها بر پایه الگوریتم فراابتکاری \lr{JAYA} داده می‌شود، قابل‌ذکر است که مدل‌سازی یا شبیه‌سازی این مدل با استفاده از زبان برنامه‌نویسی پایتون انجام شده. این فصل دارای چندین بخش است، در این بخش‌ها؛ مراحل شبیه‌سازی، سناریوهای آن، تولید داده‌های موردنیاز آزمایش‌ها و طراحی آزمایش‌ها و ارزیابی آن توضیح داده شده است. یکی از بحث‌های مهم در این فصل نیز تولید داده است، ازآنجایی‌که داده مناسب و مرتبط در این زمینه یافت نشد، باتوجه‌به درنظرگرفتن ملاحظاتی سعی شده داده‌های اولیه مناسب و نزدیک به واقعیت برای شبیه‌سازی‌ها استفاده گردد.
\section{مقدمه}
در شبیه‌سازی سعی شده سناریوهای مختلف ارزیابی شود، سناریوهای مختلف را می‌توان برای محیط‌های مختلف در نظر گرفت به طور خاص این سناریوها می‌توانند در محیط‌های مختلفی در نظر گرفته شوند، تعداد پهپادها و کاربران در سناریو برای شبیه‌سازی دقیق‌تر متفاوت است، به‌عبارت‌دیگر با آزمایش تعداد مختلف به دنبال ارزیابی بهتر مدل ارایه شده هستیم، سناریوهای پایه شبیه‌سازی شامل به ترتیب تعداد پهپاد ۲، ۳، ۵، ۱۰ و تعداد کاربران ۵۰، ۷۵، ۱۵۰ است. پارامترهای مورد هدف ارزیابی در سناریوهای مختلف، باتوجه‌به اینکه انرژی یکی از پارامترهای مهم استفاده پهپاد در سناریوهای مختلف است، انرژی یکی از پارامترهای اصلی ارزیابی مدل ارایه شده در شبیه‌سازی است، از طرفی میزان گذردهی، اتلاف بسته‌های ارسالی و تأخیر نیز شبیه‌سازی‌شده است.
\section{مراحل شبیه‌سازی}
در شبیه‌سازی مدل ارایه شده از زبان برنامه‌نویسی \lr{Python} با استفاده از کتابخانه‌های موردنیاز مدل‌سازی ریاضی انجام شده، همچنین دادهایی به‌منظور بهره‌بری از مدل یاد شده ایجاد شده، همان‌طور که اشاره شده، \gls{Dataset} مناسب و قابل‌دسترسی برای مدل‌سازی در این حوزه پژوهشی یافت نشد، بدین منظور با بررسی دقیق ماهیت پهپادها و شرایط محیطی اسکریپتی تهیه شده که با استفاده از آن می‌توان داده‌هایی برای سناریوهای مختلف ایجاد کرد، این امر می‌تواند به‌عنوان یک پژوهش مکمل در کنار ایده اصلی این پژوهش موردتوجه باشد. بعد از مدل‌کردن و استفاده از الگوریتم یاد شده، نتایج به‌دست‌آمده در قالب نمودارهایی برای ارایه نتایج ترسیم شده است، نمودارهای شامل اتلاف انرژی پهپادها، میزان گذردهی، میزان اتلاف بسته و تأخیر نشان‌داده‌شده است.
\subsection{سناریو شبیه‌سازی}
سناریوهای مورد ارزیابی در نظر گرفته شده برای بهره‌مندی از قابلیت تحرک در یک سناریو پایش و نظارت بر محیط را می‌توان با آزمایش‌های مختلف مورد ارزیابی قرار گیرد. سناریویی را فرض می‌کنیم که در آن \lr{BS}های زمینی ممکن است به دلیل آسیب تجهیزات یا قطع برق غیرفعال باشند. در چنین شرایطی، \lr{UAV-BS} می‌توانند نقش حیاتی در راه‌اندازی سریع و کارآمد خدمات ارتباطی داشته باشند. چندین پهپاد مجهز به آنتن‌های سلولی در مناطق هدف مستقر می‌شود و خدمات ارتباطی بی‌سیم موقت را ارایه می‌دهند. یکی از موارد استفاده زمانی به وجود می‌آید که نیاز اساسی به انعطاف‌پذیری در استقرار (استقرار پذیری سریع)، دسترسی مطمئن و تأخیر کم در چارچوب شبکه‌های \lr{6G} وجود داشته باشد. در این شرایط، پهپادها به‌عنوان زیرساخت \lr{5G} یا \lr{6G} در محیط‌های شهری یا غیرشهری به ارایه راه‌حلی بالقوه به‌عنوان سلول‌های کوچک و کم‌هزینه عمل می‌کنند، همان‌طور که در \cite{Bajracharya2022} مشخص شده است. پهپادها در هنگام بلایای طبیعی، مانند طوفان‌های شدید، نقش حیاتی ایفا می‌کنند و در نتیجه تلاش‌های امدادی و قابلیت‌های ارتباط رادیویی را افزایش می‌دهند. به طور ویژه، پیامدهای طوفان کاترینا در ایالات متحده بر نیاز ضروری برای چنین فناوری ارتباطی تأکید می‌کند، همچنین در کشور ایران به توجه به وجود بلایای طبیعی یا غیرطبیعی که باعث مشکلات متعددی می‌شود و ممکن است باعث خسارت‌های جبران‌ناپذیر یا پرهزینه‌ای شود، نیاز به چنین فناوری ارتباطی و پایشگری موردتوجه و تأکید است. علاوه بر این، پهپادها می‌توانند پیامدهای بلایای غیرطبیعی مانند جنگ یا آتش‌سوزی را که می‌توانند ارتباط رادیویی و دسترسی به اینترنت را مختل کنند، همان‌طور که در \cite{Parvaresh2023} و \cite{Liao2022} بررسی شده‌اند، بهبود بخشند. هدف این تحقیق بررسی این سناریوها از طریق شبیه‌سازی است. در \autoref{fig:scenario2} دو سناریو نشان‌دهنده بلایای طبیعی و غیرطبیعی ارایه شده است. \lr{UAV-BS} می‌تواند به طور مؤثر در این موقعیت‌های فاجعه‌بار کمک کنند، در این شبیه‌سازی، این سناریوها برای نمایش موقعیت‌های دنیای واقعی که در آن \lr{UAV-BS}ها استفاده می‌شوند، مدل‌سازی می‌شوند.
\begin{figure}
\includegraphics[width=0.83\textwidth]{Scenario.pdf}
\caption[\lofimage{Scenario.pdf}%
سناریوهای بلایای غیر طبیعی و بلایای طبیعی]{سناریوهای بلایای غیر طبیعی و بلایای طبیعی}
	\label{fig:scenario2}
\end{figure}
\subsection{تولید دادگان}
برای تولید دادگان مرتبط با سناریو نزدیک به واقعیت استفاده از پهپاد با درنظرگرفتن شرایط محیطی، چرا که شرایط محیطی می‌تواند تأثیرات ویژه‌ای بر روی عملکرد سناریو و نتایج داشته باشد. در این پژوهش از یک اسکریپت پایتون برای تولید مجموعه‌داده اولیه با فیلدهای ذکر شده برای چهار آزمایش استفاده شده است. به روشی مشابه شبیه‌سازی مونت‌کارلو، ۱۰۰۰ تغییر مجزا از مجموعه‌داده اولیه برای هر آزمایش تولید می‌شود. این رویکرد مدل پیشنهاد شده را قادر می‌سازد تا بر روی هر تغییر داده خروجی‌ها مشاهده شود، در نتیجه میانگینی از این خروجی‌ها محاسبه خواهد شد، باتوجه‌به نیازهای محاسباتی قابل‌توجه و با حجم زیاد برای I، J و T، این مدل تنها بر روی ۱۰ تغییر مجموعه‌داده اول برای هر آزمایش اجرا شده و خروجی‌های آن را ارزیابی می‌کند. سپس میانگین این ۱۰ تغییر محاسبه شده و در نتیجه یک نتیجه واحد که نمایانگر میانگین داده‌ها است نشان داده می‌شود. شبیه‌سازی فقط یک‌بار بر روی این میانگین داده اجرا می‌شود. مشاهده می‌شود که خروجی اجرای ۱۰ بار شبیه‌سازی و میانگین‌گیری نتایج معادل یک‌بار اجرای شبیه‌سازی بر روی داده‌های متوسط است. این رویکرد امکان حفظ منابع محاسباتی را با حفظ دقت نتایج فراهم می‌کند. همچنین بر اساس تجزیه‌وتحلیل انجام شده، شبیه‌سازی با استفاده از میانگین داده‌های حاصل از ۱۰۰۰ تغییر انجام می‌شود و خروجی‌های حاصل مستند می‌شوند. این خروجی‌ها در ادامه به‌تفصیل آمده است.
\subsection{طراحی آزمایش‌ها}
اجرای یک آزمایش فاکتوریل کامل و جامع نیاز به برنامه‌ریزی دقیق و ساختار سامان‌مند برای ارزیابی پویایی و اثرات  \lr{UE} و \lr{UAV-BS} در یک شبکه ارتباطی دارد. این پژوهش به بررسی این موضوع می‌پردازد که چگونه مقادیر مختلف کاربران به ترتیب با تعداد ۵۰، ۷۵ و ۱۵۰ و پهپادها به ترتیب با تعداد ۲، ۵، ۷ و ۱۰ بر معیارهای عملکرد شبکه، مانند از اتلاف بسته، میزان گذردهی، تأخیر، انرژی مصرفی تحرک و ارتباط پهپادی تأثیرگذار است. طراحی آزمایش‌ها شامل اجرای ۱۲ آزمایش مجزا، واپایش دقیق و تغییر متغیرهای مستقل برای مشاهده اثرات آنها بر متغیرهای وابسته است. ماتریس طراحی آزمایش شامل دو بعد است: اول که در آن تعداد \lr{UAV-BS}ها ثابت هستند از طرفی \lr{UE}ها غیرثابت (تغییرپذیر) هستند، و دیگری که در آن \lr{UE}ها ثابت هستند درحالی‌که \lr{UAV-BS}ها تغییر می‌کند. هر آزمایش (کارآزمایی) ترکیب‌های خاصی از \lr{UE} و \lr{UAV-BS} را برای انجام آزمایش‌ها به طور مؤثر جدا می‌کند. در مجموعه اولیه آزمایش‌ها، جایی که تعداد \lr{UAV-BS} ثابت نگه داشته می‌شوند، مقادیر مختلفی از \lr{UE} برای ارزیابی تأثیر آنها بر معیارهای عملکرد استفاده می‌شود. برعکس، در مجموعه آزمایش‌های بعدی، تعداد \lr{UE} ثابت می‌ماند درحالی‌که تعداد \lr{UAV-BS} برای درک تأثیر آنها بر معیارهای ذکر شده متفاوت است.
\begin{figure}
\includegraphics[width=0.9\textwidth]{Postions-1}
\caption[\lofimage{Postions-1}%
موقعیت‌های بهینه پهپادها در حین خدمت به تعداد مختلف کاربران]{موقعیت‌های بهینه پهپادها در حین خدمت به تعداد مختلف کاربران}
\label{fig:PostionsPlots}
\end{figure}
\section{نتیجه‌گیری}
در این بخش نتایج شبیه‌سازی به‌دست‌آمده از توصیف مسئله بهینه‌سازی پیشنهادی و الگوریتم را مورد نتیجه‌گیری قرار می‌دهد که شامل انرژی حرکتی و ارتباطی، اتلاف بسته‌ها، تأخیر، گذردهی، و تعداد کاربران متصل به هر پهپاد به‌صورت تفکیکی می‌شود. این قسمت شامل بررسی تفکیک شده بر روی هر یک از نتیجه‌ها بر اساس پارامترهای ارزیابی مدل است که به‌صورت نشانه گذری شده به‌صورت جداگانه بررسی و مورد ارزیابی قرار گرفته است.
\begin{itemize}
	\item \textit{مکان‌یابی و تحرک بهینه پهپادها در سناریوهای مختلف}
	مدل بهینه‌سازی ارایه شده باهدف تعیین موقعیت‌های بهینه \lr{UAV-BS} برای یک طراحی آزمایش (کارآزمایی) جامع شامل ۵۰، ۷۵ و ۱۵۰ کاربر، با تعداد متفاوتی از پهپاد شامل ۲، ۳، ۵ و ۱۰ در واحد زمانی که نشان‌دهنده تغییرات تحرکی پهپادها به‌منظور تحرک‌پذیری بهینه برای خدمت‌رسانی بهتر به \lr{UE}ها است. این محاسبات در پیکربندی‌های مختلف داده ورودی که ۱۰ مرحله زمانی را شامل شده، انجام می‌شود. \autoref{fig:PostionsPlots} تحرک بهینه \lr{UAV-BS} را در ۱۲ آزمایش و ارتباط آن‌ها با \lr{UE} را در مرحله زمانی اولیه برای هر سناریو نشان می‌دهد. مطابق \autoref{fig:PostionsPlots} هر \lr{UAV-BS} با یک مربع رنگی متفاوت نشان‌داده‌شده است. \lr{UE}ها با شناسه آن‌ها به‌صورت نقاط کوچک رنگی مطابق با \lr{UAV-BS} متصل آن‌ها نشان داده می‌شوند. \lr{UE}های غیرمتصل به رنگ خاکستری نشان داده شده‌اند. \autoref{fig:PostionsPlots} نشان می‌دهد که وقتی تعداد \lr{UAV-BS} روی ۱۰ قرار می‌گیرد، احتمال پوشش ۱۰۰ درصد به دست می‌آید که عملاً همه \lr{UE}ها را پوشش می‌دهد. به دلیل محدودیت فضا، تنها موقعیت‌ها در زمان ۰ در این پژوهش آورده شده است. درحالی‌که جایگاه پهپادها و کاربران در ۱۰ مرحله زمانی برای هر آزمایش در نظر گرفته شده است که نشان‌دهنده تحرک‌پذیری پهپاد و مدیریت هر چه‌بهتر تحرک پهپاد در ۱۰ مرحله زمانی تحرک پهپادها در پوشش‌دهی کاربران زمینی است.
	\item \textit{نرخ اتلاف بسته}
	در سناریوهای ارتباط بی‌سیم مانند ارتباطات مبتنی بر پهپاد، کاهش نرخ اتلاف بسته برای افزایش عملکرد شبکه بسیار مهم است. همان‌طور که پهپادها شروع به تحرک از موقعیت‌های اولیه خود می‌کند، \lr{UE}های مختلف با آن‌ها ارتباط برقرار می‌کنند. همچنین جدول زمانی برای \lr{UAV-BS} به‌منظور رسیدن به مجاورت \lr{UE} در تعیین اندازه‌گیری دقیق اتلاف بسته اهمیت دارد. تخمین نرخ اتلاف بسته شامل یک رابطه خطی است که عوامل مختلفی مانند تعداد کاربران متصل، نرخ داده \lr{UE}ها و نزدیکی آن‌ها به \lr{UAV-BS} را در نظر می‌گیرند. در \autoref{fig:PacketLossPlot}، نرخ اتلاف بسته برای ۲، ۳، ۵ و ۱۰ پهپاد در طول زمان در آزمایش‌های (کارآزمایی‌های) مختلف نشان‌داده‌شده که هر کدام شامل تعداد متفاوتی از \lr{UE} (۷۵،۵۰ و ۱۵۰) است. با مشاهده \autoref{fig:PacketLossPlot}، آشکار است که میانگین اتلاف بسته برای تعداد پهپاد ۲، ۳ و ۵ در مقایسه با \lr{UE}ها به ترتیب سناریوهای ۵۰ و ۷۵ بیشتر از سناریوهای ۵۰ و ۷۵ با ۱۰ پهپاد. بااین‌حال، هنگام استفاده از تعداد ۱۰ \lr{UAV-BS} برای پوشش این مجموعه از \lr{UE}ها، مقادیر اتلاف بسته به دلیل پوشش افزایش‌یافته تقریباً به همان سطوح همگرا می‌شوند. با استقرار تعداد ۱۰ \lr{UAV-BS} پوشش کامل برای هر زیر مجموعه ۵۰، ۷۵ و ۱۵۰ \lr{UE} به دست می‌آید. به طور قابل‌توجه، واریانس مشاهده شده در نمودارهای مربوط به اتلاف بسته‌ها به ماهیت تصادفی درخواست‌های داده از \lr{UE} در موارد زمانی مختلف نسبت داده می‌شود.
\begin{figure}
\includegraphics[width=0.83\textwidth]{pktloss0.pdf}
\caption[\lofimage{pktloss0.pdf}%
میزان اتلاف بسته‌ها در طول آزمایش و تعداد مختلف کاربر]{میزان اتلاف بسته‌ها در طول آزمایش و تعداد مختلف کاربر}
	\label{fig:PacketLossPlot}
\end{figure}		
	\item  \textit{نرخ تأخیر}
	تأخیر بین \lr{UE}ها و \lr{UAV-BS}ها در بسیاری از کاربردهای مدرن، به‌ویژه در حوزه سنجش از راه دور، نظارت و ارتباطات از اهمیت بالایی برخوردار است. تأخیر کم برای تصمیم‌گیری و واپایش در زمان واقعی بسیار مهم است، زیرا مستقیماً بر پاسخگویی \lr{UAV-BS} به درخواست \lr{UE} و تحویل به‌موقع داده‌ها یا خدمات تأثیر می‌گذارد. تأخیر کاهش‌یافته، کارایی \lr{UAV-BS} را در کارهایی مانند واکنش به بلایا که در آن تصمیم‌گیری و ارتباطات سریع می‌تواند جان انسان‌ها را نجات دهد، یا در خدمات تحویل که در آن ناوبری سریع و اجتناب از موانع ضروری است، اهمیت دارد. به طور خلاصه، به‌حداقل‌رساندن تأخیر بین \lr{UE} و \lr{UAV-BS} برای اطمینان از کارایی، ایمنی و قابلیت اطمینان عملیات پهپاد در طیف وسیعی از صنایع و کاربردها حیاتی است. 
	در \autoref{tab:avglatency} مقایسه میانگین تأخیر مدل بهینه‌سازی پیشنهادی را برای یک طرح فاکتوریل کامل نشان می‌دهد. همان‌طور که در \autoref{tab:avglatency} نشان‌داده‌شده است، متوسط تأخیر برای هر مجموعه از \lr{UE}ها و \lr{UAV-BS} مربوطه آن‌ها تقریباً ۱ میلی‌ثانیه است.
	تأخیر در این زمینه به طور قابل‌توجهی تحت‌تأثیر جداسازی فضایی بین \lr{UE}ها و \lr{UAV-BS}ها است. یک همبستگی مستقیم وجود دارد که در آن افزایش فاصله بین \lr{UE}ها و \lr{UAV-BS}ها منجر به تأخیر مشاهده شده بالاتر می‌شود. مدل بهینه‌سازی پیشنهادی به طور راهبردی فاصله بین \lr{UE}ها و \lr{UAV-BS}ها را به حداقل می‌رساند و در نتیجه به طور ذاتی تأخیر را کاهش می‌دهد. کاهش فاصله نیز منجر به کاهش قابل‌توجه در تأخیر می‌شود که نمونه‌ای از کارآمدی مدل پیشنهادی در دستیابی به تأخیر ارتباطی کمتر است. این رویکرد به‌عنوان روشی برای مدیریت تحرک به‌منظور بهبود در ارتباطات و تأخیر نقش مهمی در افزایش کارایی و پاسخگویی سامانه ارتباطی با کاهش نگرانی‌های تأخیر از طریق به‌حداقل‌رساندن فاصله بین \lr{UE} و \lr{UAV-BS} ایفا می‌کند.
	% Please add the following required packages to your document preamble:
% \usepackage{multirow}
% \usepackage[table,xcdraw]{xcolor}
% Beamer presentation requires \usepackage{colortbl} instead of \usepackage[table,xcdraw]{xcolor}
\begin{table}%[]
\centering
\caption{ متوسط ​​تاخیر هر پهپاد در هر سناریو در طول زمان}
\begin{tabular}{|c|c|c|c|}
\hline
\multicolumn{1}{|l|}{کارآزمایی} & \multicolumn{1}{l|}{تعداد کاربران} & \multicolumn{1}{l|}{تعداد پهپادها} & \multicolumn{1}{l|}{میانگین تاخیر (ms)} \\ \hline
1                                & \multirow{4}{*}{50}                & 2                                      & 0.60                                      \\ \cline{1-1} \cline{3-4} 
2                                &                                    & 3                                      & 0.40                                      \\ \cline{1-1} \cline{3-4} 
3                                &                                    & 5                                      & 0.24                                      \\ \cline{1-1} \cline{3-4} 
4                                &                                    & 10                                     & 0.20                                      \\ \hline
5                                & \multirow{4}{*}{75}                & 2                                      & 0.90                                      \\ \cline{1-1} \cline{3-4} 
6                                &                                    & 3                                      & 0.72                                      \\ \cline{1-1} \cline{3-4} 
7                                &                                    & 5                                      & 0.61                                      \\ \cline{1-1} \cline{3-4} 
8                                &                                    & 10                                     & 0.37                                      \\ \hline
9                                & \multirow{4}{*}{150}               & 2                                      & 1.82                                      \\ \cline{1-1} \cline{3-4} 
10                               &                                    & 3                                      & 1.22                                      \\ \cline{1-1} \cline{3-4} 
11                               &                                    & 5                                      & 1.66                                      \\ \cline{1-1} \cline{3-4} 
12                               &                                    & 10                                     & 0.67                                      \\ \hline
\end{tabular}
\label{tab:avglatency}
\end{table}\textbf{}
	شایان‌ذکر است که مقادیر پایین مشاهده شده در هر دو پارامتر، تأخیر و اتلاف از ارتباط مستقیم تک گام بین \lr{UE}ها و \lr{UAV-BS}ها نشئت می‌گیرد. برعکس، از طرفی قابل‌ذکر است که عوامل تضعیف کانال در نظر گرفته نشده، به‌عبارت‌دیگر، سناریوها به طور ایده‌آل ارتباط را با یک پرش واحد بین \lr{UE} و \lr{UAV-BS} آزمایش شده است. بااین‌حال، بر اساس مدل ارایه شده، عوامل مؤثر بر این پارامترها شامل فاصله بین \lr{UE} و \lr{UAV-BS}، تعداد اتصالات هم‌زمان به \lr{UAV-BS} و الزامات داده‌ای \lr{UE}ها در نظر گرفته شده است.
	\item  \textit{انرژی مصرفی تحرک و ارتباط پهپاد}
	مصرف انرژی پهپادها به طور قابل‌توجهی تحت‌تأثیر تحرک آن‌ها است. بااین‌حال، توجه به این نکته مهم است که انتقال و دریافت داده در \lr{UAV-BS} انرژی مربوط به برقراری ارتباطات را نیز مورد تأکید قرار می‌دهد، عاملی که غالباً در مقایسه با مصرف انرژی مربوط به حرکت نادیده گرفته می‌شود. انرژی تحرکی جنبه‌های مختلف تحرک مانند حرکت افقی و عمودی، فرود، برخاستن و شناور شدن را در بر می‌گیرد. به‌حداقل‌رساندن مدیریت تحرک \lr{UAV-BS} و درعین‌حال بهینه‌سازی راهبُردی قرارگیری آنها برای به حداکثر رساندن پوشش کاربران به ما امکان می‌دهد مصرف انرژی مربوط به تحرک را کاهش دهیم.
	در \autoref{fig:movepower} و \autoref{fig:commpowerfact} به ترتیب مقایسه انرژی حرکتی و مصرف توان ارتباطی پهپادها را در سناریوها و آزمایش‌های مختلف نشان می‌دهند.
	باتوجه‌به \autoref{fig:movepower}، مصرف انرژی حرکتی مهم‌ترین بخش مصرف توان پهپادها در مقایسه با توان ارتباطی است، همان‌طور که توسط مدل پیشنهادی تعیین می‌شود، انرژی حرکتی پهپادها تابعی از مشخصات موتور محرک، موقعیت مکانی کاربران، موقعیت پهپادها، سرعت باد، جهت باد و سرعت پهپاد است. باتوجه‌به \autoref{fig:movepower} که در آن خطوط نشان‌دهنده انرژی تحرک متوسط زمانی است که حرکت پهپاد، کوچک می‌مانند.
	همان‌طور که می‌توان در \autoref{fig:movepower} مشاهده می‌شود، مواردی وجود دارد که انرژی تحرک در طی مراحل زمانی خاص افزایش می‌یابد. این اتفاق زمانی می‌افتد که \lr{UAV-BS} برای دستیابی به مکان‌های بهینه کاربران برای پوشش‌دهی و تحرک‌پذیری بهینه در نتیجه مدیریت بهتر تحرک به آن نیاز دارد.
	در \autoref{fig:movepower}، به‌ویژه در نمودار (a)، مقادیر پیک در مرحله زمانی ۲ برای آزمایش‌ها با مقادیر ۵۰ و ۱۵۰ کاربر وجود دارد که می‌تواند به \lr{UAV-BS}هایی نسبت داده شود که مسافتی نزدیک به ۱۰۰ درصد را پوشش می‌دهند. در واحدهای بین مراحل زمانی ۱ و ۴. این فاصله بسیار مهم است؛ زیرا با داده‌های موردنیاز \lr{UE}ها هماهنگ است و تحت‌تأثیر عوامل مختلفی مانند الگوهای تحرک\lr{UAV-BS}، حرکت پویا، تغییرات موقعیت، ظرفیت پهپاد و محدوده پوشش قرار می‌گیرد. علاوه بر این، شبیه‌سازی شرایط باد را به طور تصادفی انتخاب می‌کند، سرعت باد بازه‌ای از سرعت ۱ تا ۲۱ کیلومتر در ساعت و زاویه وزش آن ۰ تا ۳۶۰ درجه است، در مرحله زمانی ۲، مقادیر ویژه سرعت باد ($V_{t}^{Wind}$) با سرعت ۱۷.۲۳ کیلومتر در ساعت و جهت باد ($\theta_{t}^{Wind}$) در ۲۵۱.۰۶ درجه ثبت می‌شوند. این مقدار نشان‌دهنده وزش باد شدید مخالف جهت حرکت \lr{UAV-BS} است. در نتیجه، این مخالفت منجر به افزایش مصرف انرژی توسط \lr{UAV-BS} به دلیل اثر مقاومت باد بر روی پهپاد می‌شود. تنظیمات تصادفی مشابهی ممکن است در نمودارهای دیگر مشاهده شود که می‌تواند تحت‌تأثیر این عوامل محیطی و تحرک‌پذیری‌های متنوعی شود.
\begin{figure}
\includegraphics[width=0.83\textwidth]{movementpower.pdf}
\caption[\lofimage{movementpower.pdf}%
انرژی مصرفی حرکتی پهپاد تعداد کاربران مختلف]{انرژی مصرفی حرکتی پهپاد تعداد کاربران مختلف}
\label{fig:movepower}
\end{figure}
 در \autoref{fig:commpowerfact} توان ارتباطی مصرف شده توسط هر \lr{UAV-BS} را در طول زمان در آزمایش‌های مختلف با تعداد متغیر \lr{UE} نشان می‌دهد. یکی از عوامل مهمی که بر این قدرت ارتباطی تأثیر می‌گذارد، تعداد \lr{UE}هایی است که توسط هر \lr{UAV-BS} پوشش داده می‌شوند. به‌طورکلی، هنگامی که یک \lr{UAV-BS} تعداد \lr{UE}های بیشتری را پوشش می‌دهد، تمایل به مصرف انرژی بیشتری برای ارتباط دارد، زیرا نیاز به مدیریت انتقال و دریافت داده برای یک مجموعه بزرگ‌تر \lr{UE} را دارد. برعکس، زمانی که \lr{UE}های کمتری متصل می‌شوند، توان ارتباطی کمتر می‌شود. این رابطه در تمام نمودارهای نشان‌داده‌شده صادق است.
 \begin{figure}
\includegraphics[width=0.83\textwidth]{commpowerfactorial.pdf}
\caption[\lofimage{commpowerfactorial.pdf}%
انرژی مصرفی ارتباطی پهپاد تعداد کاربر مختلف]{انرژی مصرفی ارتباطی پهپاد تعداد کاربر مختلف}
\label{fig:commpowerfact}
\end{figure}
علاوه بر این، \autoref{tab:connectdiseduser} بینش‌های ارزشمندی را در مورد میانگین درصدی \lr{UE}های متصل و غیرمتصل در طول مدت هر تعامل \lr{UAV-BS} را ارایه می‌کند که به طور مستقیم بر روند مشاهده شده در انرژی ارتباطی نمایش‌داده‌شده در نمودار انرژی مرتبط با ارتباطات تأثیر می‌گذارد. تجزیه‌وتحلیل جدول، یک‌روند واضح را نشان می‌دهد: با افزایش تعداد \lr{UAV-BS}، درصد اتصال مشاهده شده در بین \lr{UE}ها افزایش می‌یابد. علاوه بر این، قابل‌توجه است که در هر آزمایش، نرخ پوشش ۱۰۰ درصد با استفاده از ۱۰ \lr{UAV-BS} به‌دست‌آمده است که بر همبستگی بین کمیت \lr{UAV-BS} و اتصال افزایش‌یافته مشاهده شده در سراسر تعامل با \lr{UE}ها تأکید می‌کند.
\item \textit{نرخ گذردهی}
در \autoref{fig:Throughput} میزان گذردهی (اندازه‌گیری شده در میلی‌ثانیه) \lr{UAV-BS} را در طول زمان نشان می‌دهد، و بینش‌هایی را در مورد نرخ انتقال داده به‌دست‌آمده توسط این \lr{UAV-BS} را ارایه می‌دهد. عوامل متعددی در تغییر گذردهی در این نمودار نقش دارند. یکی از عوامل مهم تعداد \lr{UE}های متصل به هر \lr{UAV-BS} در یک‌زمان معین است. هنگامی که یک \lr{UAV-BS} به تعداد بیشتری از \lr{UE}ها به طور هم‌زمان سرویس دهد، معمولاً منجر به میزان گذردهی بالاتری می‌شود، زیرا داده‌های بیشتری ارسال و دریافت خواهد شد. برعکس، زمانی که \lr{UE}های کمتری متصل شوند، به دلیل کاهش تقاضای تبادل داده، توان عملیاتی کمتر خواهد شد. علاوه بر این، کیفیت پیوند بی‌سیم، تراکم شبکه و کارایی پروتکل‌های انتقال داده نیز می‌تواند بر میزان گذردهی تأثیر بگذارد. گذردهی بالا زمانی به دست می‌آید که این عوامل به طور مطلوب در یک راستا قرار گیرند و امکان انتقال کارآمد و سریع داده‌ها را فراهم کنند، درحالی‌که گذردهی پایین اغلب نتیجه شرایط نامطلوب یا تراکم شبکه است که مانع از جریان داده می‌شود. بازده نیز تحت‌تأثیر ازدست‌دادن بسته، ظرفیت \lr{UAV-BS} که به طور تصادفی در محدوده ۴۰۰-۵۰۰ متر تولید می‌شود و نرخ داده \lr{UE}ها است. نرخ داده‌ای که به طور تصادفی تولید می‌شود، تأثیر مستقیم و قابل‌توجهی بر نرخ گذردهی دارد و به عدم تعادل مشاهده شده در نمودارهای گذردهی کمک می‌کند.
	% Please add the following required packages to your document preamble:
% \usepackage{multirow}
% \usepackage[table,xcdraw]{xcolor}
% Beamer presentation requires \usepackage{colortbl} instead of \usepackage[table,xcdraw]{xcolor}
\begin{table}%[]
\centering
\caption{میانگین درصدی متصل یا غیر متصل بودن کاربران به ازای هر کار‌آزمایی در زمان}
\begin{tabular}{|c|c|c|c|c|}
\hline
کار‌آزمایی & تعداد کاربران        & تعداد پهپاد‌ها & میانگین متصل & میانگین‌ غیر متصل‌ \\ \hline
1          & \multirow{4}{*}{50}  & 2                 & 51                   & 49                      \\ \cline{1-1} \cline{3-5} 
2          &                      & 3                 & 66                   & 34                      \\ \cline{1-1} \cline{3-5} 
3          &                      & 5                 & 91.80                & 8.2                     \\ \cline{1-1} \cline{3-5} 
4          &                      & 10                & 100                  & 0                       \\ \hline
5          & \multirow{4}{*}{75}  & 2                 & 51.99                & 48.01                   \\ \cline{1-1} \cline{3-5} 
6          &                      & 3                 & 60.94                & 39.06                   \\ \cline{1-1} \cline{3-5} 
7          &                      & 5                 & 78.93                & 21.07                   \\ \cline{1-1} \cline{3-5} 
8          &                      & 10                & 100                  & 0                       \\ \hline
9          & \multirow{4}{*}{150} & 2                 & 39.07                & 60.93                   \\ \cline{1-1} \cline{3-5} 
10         &                      & 3                 & 62                   & 38                      \\ \cline{1-1} \cline{3-5} 
11         &                      & 5                 & 88.81                & 11.19                   \\ \cline{1-1} \cline{3-5} 
12         &                      & 10                 & 100                  & 0                       \\ \hline
\end{tabular}
\label{tab:connectdiseduser}
\end{table}
\begin{figure}
\includegraphics[width=0.9\textwidth]{throughput.pdf}
\caption[\lofimage{throughput.pdf}%
میزان گذردهی پهپاد‌ها در طول آزمایش‌ها و تعداد کاربران مختلف]{میزان گذردهی پهپاد‌ها در طول آزمایش‌ها و تعداد کاربران مختلف}
\label{fig:Throughput}
\end{figure}
\end{itemize}
\subsection{تجزیه تحلیل}\label{subsection:referencecompare}
برای اعتبارسنجی کارایی مدل بهینه‌سازی پیشنهادی در این پژوهش، الگوریتم‌های مرتبط، از جمله \cite{Pasandideh2023} و \cite{8892933} برای مقایسه و تحلیل انتخاب شده‌اند. در \cite{Pasandideh2023}، مشاهده شد که کاربران خاص نرخ ازدست‌دادن بسته‌های قابل‌توجهی را تجربه می‌کنند. این مشکل عمدتاً از موقعیت دور از دسترس این کاربران از منطقه پوشش‌دهی پهپادها مربوطه ناشی می‌شود و باعث می‌شود داده‌ها از این کاربران نتوانند به پهپادها دسترسی پیدا کنند. بااین‌حال، آزمایش‌های جامع نشان‌داده‌شده در \autoref{fig:PacketLossPlot}، نشان‌دهنده بر اساس مدیریت تحرک نرخ اتلاف بسته‌های پهپاد یک بهبود مشهود آشکار است. نرخ کلی اتلاف بسته به‌دست‌آمده از طریق روش پیشنهادی به طور قابل‌توجهی کمتر از میانگین نرخ اتلاف بسته است که در \cite{Pasandideh2023} بیان شده است. این پیشرفت را می‌توان به افزایش دقت مدل فعلی در تعیین موقعیت‌ها و مدیریت تحرک بهینه پهپادها نسبت داد. در نتیجه، کاربران به طور مؤثرتری به پهپادهای نزدیک‌تر تخصیص داده می‌شوند که منجر به کاهش قابل‌توجه نرخ اتلاف بسته می‌شود.
\\
در \cite{Pasandideh2023} متوسط تأخیر ثبت شده برای هر مجموعه کاربر و پهپاد مربوط به آن تقریباً $ms$15 بود. این مقدار تأخیر در مقایسه با نرخ تأخیر به‌دست‌آمده در روش پیشنهادی، همان‌طور که در \autoref{tab:avglatency} ارایه شده است، به طور قابل‌توجهی بالاتر است، جایی که میانگین تأخیر تقریباً 1$ms$ است. تأخیر به طور مستقیم بافاصله بین کاربران و پهپادها ارتباط دارد که نشان می‌دهد فاصله فیزیکی بیشتر بین این موجودیت‌ها منجر به تأخیر بالاتر می‌شود. روش پیشنهادی بر به‌حداقل‌رساندن فاصله بین کاربران و پهپادها نیز تمرکز دارد، در نتیجه تأخیر مدل پیشنهادی را در مقایسه با تأخیر گزارش شده در \cite{Pasandideh2023} کاهش می‌دهد. علاوه بر این، یک ارتباط مستقیم بین ازدست‌دادن بسته و تأخیر وجود دارد. روش پیشنهادی به طور قابل‌توجهی ازدست‌دادن بسته را در مقایسه با کار مرتبط اشاره شده کاهش داده است. هنگامی که بسته‌ها در طول انتقال از طریق شبکه گم می‌شوند، زمان بیشتری برای شناسایی و بازیابی این بسته‌های ازدست‌رفته توسط فرستنده لازم است؛ بنابراین، کاهش تلفات بسته در روش پیشنهادی به طور مستقیم به تأخیر کمتر مشاهده شده کمک می‌کند.
\\
مقایسه مصرف انرژی نشان‌داده‌شده در \autoref{fig:powercomp} یک‌روند صعودی را در خط‌قرمز نشان می‌دهد که نشان‌دهنده مصرف انرژی بر اساس روش مشخص شده در \cite{8892933} است. این روند حاکی از همبستگی مستقیم بین افزایش تعداد \lr{UE}ها و افزایش مصرف انرژی است. بااین‌حال، مدل پیشنهادی که با خط آبی نشان داده می‌شود، تشخیص می‌دهد که مصرف انرژی صرفاً به تعداد کاربرانی که خدمات ارایه می‌دهند وابسته نیست. در مدل پیشنهادی، عوامل اضافی مختلف به طور قابل‌توجهی در مصرف انرژی نقش دارند. عناصری مانند سرعت و جهت باد، سرعت \lr{UE}ها، نرخ داده موردنیاز آن‌ها، ظرفیت پهپاد و سایر عوامل کمک‌کننده اجزای جدایی‌ناپذیر در مدل مصرف انرژی هستند. این عوامل مجموعاً پویایی مصرف انرژی را شکل می‌دهند و بر آن تأثیر می‌گذارند. این دیدگاه گسترده‌تر منجر به الگوی مصرف انرژی پیچیده‌تر می‌شود. به‌عنوان‌مثال، تشدید سرعت باد و جهت مخالف می‌تواند به طور قابل‌توجهی بر استفاده از توان تأثیر بگذارد، حتی در سناریوهایی با تعداد ثابتی از \lr{UE} این تأثیر قابل‌مشاهده است. با ادغام این عوامل متنوع، مدل پیشنهادی درک جامعی از دینامیک مصرف انرژی ارایه می‌کند که فراتر از تمرکز منحصربه‌فرد بر روی کمیت \lr{UE}هایی است که برای به‌تصویرکشیدن تأثیر متقابل پیچیده متغیرهای متعدد مؤثر بر مصرف انرژی عمل می‌کنند.
\begin{figure}
\includegraphics[width=0.6\textwidth]{moveComparison_5UAVs.pdf}
\caption[\lofimage{moveComparison_5UAVs.pdf}%
مقایسه توان مصرفی با روش‌ دیگر]{مقایسه توان مصرفی بین روش پیشنهادی و \cite{8892933}}
	\label{fig:powercomp}
\end{figure}