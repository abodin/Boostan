\chapter{مفاهیم پایه‌ای}
\label{chap:concepts}

در این فصل در تلاش هستیم تا مفاهیمی که خواننده برای درک هر چه بهتر و بیشتر موضوع مورد پژوهش نیاز دارد را به‌اختصار بیان کنیم. نخست در 
\autoref{sec:ueArch} 
اندکی در مورد 
\gls{Architecture} \gls{UE} در
شبکه‌های تلفن همراه سخن به میان خواهد آمد. سپس در 
\autoref{sec:propagateChannel}،
توضیحاتی در مورد
\gls{PropagationChannelModel}
ارائه خواهد شد. در انتها نیز در
\autoref{sec:MobileNetworkdata} 
به توضیح پارامترهای واسط هوایی در شبکه‌های تلفن همراه پرداخته خواهد شد.


\section{معماری \glsentryname{UE} در شبکه‌های تلفن‌همراه} \index{معماری!\gls{UE}}
\label{sec:MobileNetworkdata}
\label{sec:ueArch}
‎\gls{UE} 
یک واسط چندرسانه‌ای برای ارائه خدمات شبکه به کاربر است. در حقیقت ‎\gls{UE}‎ همان ابزاری است که کاربر برای مبادله اطلاعات با شبکه، مورداستفاده قرار می‌دهد. معماری ‎\gls{UE}‎ در شبکه‌های ‎\gls{GSM}‎ تا شبکه‌های نسل پنج تفاوت چندانی نکرده است. 