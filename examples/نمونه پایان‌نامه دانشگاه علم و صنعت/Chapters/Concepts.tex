\chapter{مفاهیم پایه‌ای}
\label{chap:concepts}
در ابتدای هر فصل از پایان‌نامه، سعی کنید نخست بدون شروع هرگونه 
\lr{Section}، 
به طور خلاصه بگویید که قرار است در این فصل در مورد چه چیزی صحبت کنید. به عنوان مثال در این فصل، نخست در 
\autoref{sec:boostanenv}،
در مورد محیط‌های مختلفی که می‌توانید در استایل بوستان از آن استفاده کنید، صحبت خواهیم کرد. سپس در 
\autoref{sec:codeintext}،
در مورد نحوه وارد کردن یک کد در متن سخن به میان خواهد آمد. 

\section{محیط‌های مختلف در استایل بوستان}\index{محیط}
\label{sec:boostanenv}
‎\ptext[1]
\subsection{محیط نکات}
\begin{note}
شهر مردگان، شهر انسان های «بی دفاع» است. این تعبیر اقتباس از قرآن کریم است که «غیبت» را خوردن گوشت «مرده» خوانده است. 
\begin{equation}
A = B + \sin (x)
\end{equation}
در تفاسیر آمده است که خداوند «انسان بی دفاع» را که به دلیل عدم حضور در مجلس بدگویی نمی تواند از خود دفاع کند، «مرده» دانسته است. پس آنجا که نسبت های ناروا دادن مباح، و دفاع کردن ممنوع است، در حقیقت «شهر مردگان» است.
\end{note}
‎\ptext[2]
\begin{problem}
شهر مردگان، شهر انسان های «بی دفاع» است. این تعبیر اقتباس از قرآن کریم است که «غیبت» را خوردن گوشت «مرده» خوانده است . در تفاسیر آمده است که خداوند «انسان بی دفاع» را که به دلیل عدم حضور در مجلس بدگویی نمی تواند از خود دفاع کند، «مرده» دانسته است. پس آنجا که نسبت های ناروا دادن مباح، و دفاع کردن ممنوع است، در حقیقت «شهر مردگان» است.


\end{problem}
‎\ptext[3]
\begin{refer}
شهر مردگان، شهر انسان های «بی دفاع» است. این تعبیر اقتباس از قرآن کریم است که «غیبت» را خوردن گوشت «مرده» خوانده است.
\begin{latin}
\lstset{numbers=none,frame=none}
\begin{lstlisting}
for i:=maxint to 0 do
begin
{ do nothing }
end;
\end{lstlisting}
\end{latin}
  در تفاسیر آمده است که خداوند «انسان بی دفاع» را که به دلیل عدم حضور در مجلس بدگویی نمی تواند از خود دفاع کند، «مرده» دانسته است. پس آنجا که نسبت های ناروا دادن مباح، و دفاع کردن ممنوع است، در حقیقت «شهر مردگان» است.
\end{refer}

\begin{info}
شهر مردگان، شهر انسان های «بی دفاع» است. این تعبیر اقتباس از قرآن کریم است که «غیبت» را خوردن گوشت «مرده» خوانده است . 
در تفاسیر آمده است که خداوند «انسان بی دفاع» را که به دلیل عدم حضور در مجلس بدگویی نمی تواند از خود دفاع کند، «مرده» دانسته است. پس آنجا که نسبت های ناروا دادن مباح، و دفاع کردن ممنوع است، در حقیقت «شهر مردگان» است.
\begin{equation}
A = B + \sin (x)
\end{equation}
\end{info}

\index{نکته}
\begin{warning}{نکات مهم}
شهر مردگان، شهر انسان های «بی دفاع» است. این تعبیر اقتباس از قرآن کریم است که «غیبت» را خوردن گوشت «مرده» خوانده است . در تفاسیر آمده است که خداوند «انسان بی دفاع» را که به دلیل عدم حضور در مجلس بدگویی نمی تواند از خود دفاع کند، «مرده» دانسته است. پس آنجا که نسبت های ناروا دادن مباح، و دفاع کردن ممنوع است، در حقیقت «شهر مردگان» است.
\end{warning}


\begin{goal}{نکات مهم}
شهر مردگان، شهر انسان های «بی دفاع» است. این تعبیر اقتباس از قرآن کریم است که «غیبت» را خوردن گوشت «مرده» خوانده است . 

در تفاسیر آمده است که خداوند «انسان بی دفاع» را که به دلیل عدم حضور در مجلس بدگویی نمی تواند از خود دفاع کند، «مرده» دانسته است.

  پس آنجا که نسبت های ناروا دادن مباح، و دفاع کردن ممنوع است، در حقیقت «شهر مردگان» است.
\end{goal}

\subsection{محیط‌های ریاضی}
کارکرد لاتک مبتنی بر این اندیشه است که نویسندگان باید قادر باشند بر نوشتن در درون ساختار منطقی متن‌شان تمرکز کنند، نه اینکه وقت خود را برای کار کردن بر روی جزئیات شکل‌دهی صرف کنند. 
\begin{ntdefinition}
شهر مردگان، شهر انسان های «بی دفاع» است. 
\end{ntdefinition}
کارکرد لاتک مبتنی بر این اندیشه است که نویسندگان باید قادر باشند بر نوشتن در درون ساختار منطقی متن‌شان تمرکز کنند، نه اینکه وقت خود را برای کار کردن بر روی جزئیات شکل‌دهی صرف کنند. 
\begin{ntdefinition}
شهر مردگان، شهر انسان های «بی دفاع» است. 
\end{ntdefinition}
کارکرد لاتک مبتنی بر این اندیشه است که نویسندگان باید قادر باشند بر نوشتن در درون ساختار منطقی متن‌شان تمرکز کنند، نه اینکه وقت خود را برای کار کردن بر روی جزئیات شکل‌دهی صرف کنند. 
\begin{ntexample}
شهر مردگان، شهر انسان های «بی دفاع» است. 
\end{ntexample}
کارکرد لاتک مبتنی بر این اندیشه است که نویسندگان باید قادر باشند بر نوشتن در درون ساختار منطقی متن‌شان تمرکز کنند، نه اینکه وقت خود را برای کار کردن بر روی جزئیات شکل‌دهی صرف کنند. 
\begin{ntexample}
شهر مردگان، شهر انسان های «بی دفاع» است. 
\end{ntexample}
\begin{ntsolution}
شهر مردگان، شهر انسان های «بی دفاع» است. این تعبیر اقتباس از قرآن کریم است که «غیبت» را خوردن گوشت «مرده» خوانده است . در تفاسیر آمده است که خداوند «انسان بی دفاع» را که به دلیل عدم حضور در مجلس بدگویی نمی تواند از خود دفاع کند، «مرده» دانسته است. پس آنجا که نسبت های ناروا دادن مباح، و دفاع کردن ممنوع است، در حقیقت «شهر مردگان» است.
\end{ntsolution}
کارکرد لاتک مبتنی بر این اندیشه است که نویسندگان باید قادر باشند بر نوشتن در درون ساختار منطقی متن‌شان تمرکز کنند، نه اینکه وقت خود را برای کار کردن بر روی جزئیات شکل‌دهی صرف کنند. 

\begin{ntpoint}

شهر مردگان، شهر انسان های «بی دفاع» است. شهر مردگان، شهر انسان های «بی دفاع» است. شهر مردگان، شهر انسان های «بی دفاع» است. شهر مردگان، شهر انسان های «بی دفاع» است. 
\end{ntpoint}
کارکرد لاتک مبتنی بر این اندیشه است که نویسندگان باید قادر باشند بر نوشتن در درون ساختار منطقی متن‌شان تمرکز کنند، نه اینکه وقت خود را برای کار کردن بر روی جزئیات شکل‌دهی صرف کنند. 

کارکرد لاتک مبتنی بر این اندیشه است که نویسندگان باید قادر باشند بر نوشتن در درون ساختار منطقی متن‌شان تمرکز کنند، نه اینکه وقت خود را برای کار کردن بر روی جزئیات شکل‌دهی صرف کنند. 

\begin{nttheorem}
شهر مردگان، شهر انسان های «بی دفاع» است. این تعبیر اقتباس از قرآن کریم است که «غیبت» را خوردن گوشت «مرده» خوانده است . در تفاسیر آمده است که خداوند «انسان بی دفاع» را که به دلیل عدم حضور در مجلس بدگویی نمی تواند از خود دفاع کند، «مرده» دانسته است. پس آنجا که نسبت های ناروا دادن مباح، و دفاع کردن ممنوع است، در حقیقت «شهر مردگان» است.
\end{nttheorem}
\begin{nttheorem}
شهر مردگان، شهر انسان های «بی دفاع» است. این تعبیر اقتباس از قرآن کریم است که «غیبت» را خوردن گوشت «مرده» خوانده است . در تفاسیر آمده است که خداوند «انسان بی دفاع» را که به دلیل عدم حضور در مجلس بدگویی نمی تواند از خود دفاع کند، «مرده» دانسته است. پس آنجا که نسبت های ناروا دادن مباح، و دفاع کردن ممنوع است، در حقیقت «شهر مردگان» است.
\end{nttheorem}
\begin{proof}
شهر مردگان، شهر انسان های «بی دفاع» است. این تعبیر اقتباس از قرآن کریم است که «غیبت» را خوردن گوشت «مرده» خوانده است . در تفاسیر آمده است که خداوند «انسان بی دفاع» را که به دلیل عدم حضور در مجلس بدگویی نمی تواند از خود دفاع کند، «مرده» دانسته است. پس آنجا که نسبت های ناروا دادن مباح، و دفاع کردن ممنوع است، در حقیقت «شهر مردگان» است.
\end{proof}

\begin{lemma}
شهر مردگان، شهر انسان های «بی دفاع» است. این تعبیر اقتباس از قرآن کریم است که «غیبت» را خوردن گوشت «مرده» خوانده است . در تفاسیر آمده است که خداوند «انسان بی دفاع» را که به دلیل عدم حضور در مجلس بدگویی نمی تواند از خود دفاع کند، «مرده» دانسته است. پس آنجا که نسبت های ناروا دادن مباح، و دفاع کردن ممنوع است، در حقیقت «شهر مردگان» است.
\end{lemma}
\begin{lemmaproof}
شهر مردگان، شهر انسان های «بی دفاع» است. این تعبیر اقتباس از قرآن کریم است که «غیبت» را خوردن گوشت «مرده» خوانده است . در تفاسیر آمده است که خداوند «انسان بی دفاع» را که به دلیل عدم حضور در مجلس بدگویی نمی تواند از خود دفاع کند، «مرده» دانسته است. پس آنجا که نسبت های ناروا دادن مباح، و دفاع کردن ممنوع است، در حقیقت «شهر مردگان» است.
\end{lemmaproof}
\begin{ntremember}
شهر مردگان، شهر انسان های «بی دفاع» است. این تعبیر اقتباس از قرآن کریم است که «غیبت» را خوردن گوشت «مرده» خوانده است . در تفاسیر آمده است که خداوند «انسان بی دفاع» را که به دلیل عدم حضور در مجلس بدگویی نمی تواند از خود دفاع کند، «مرده» دانسته است. پس آنجا که نسبت های ناروا دادن مباح، و دفاع کردن ممنوع است، در حقیقت «شهر مردگان» است.
\end{ntremember}
\begin{ntproblems}
شهر مردگان، شهر انسان های «بی دفاع» است. این تعبیر اقتباس از قرآن کریم است که «غیبت» را خوردن گوشت «مرده» خوانده است . در تفاسیر آمده است که خداوند «انسان بی دفاع» را که به دلیل عدم حضور در مجلس بدگویی نمی تواند از خود دفاع کند، «مرده» دانسته است. پس آنجا که نسبت های ناروا دادن مباح، و دفاع کردن ممنوع است، در حقیقت «شهر مردگان» است.
\end{ntproblems}

\section{وارد کردن کد در متن}
\label{sec:codeintext}
مثالی از نوشتن کد مطلب درون یک نوشتار:

\begin{latin}
\lstinputlisting[language=Matlab]{Code/code3.m}
\end{latin}

در این مثال یک کد \lr{MATLAB} دیگر وارد می کنیم، با این تفاوت که می خواهیم یکسری از کلمات کلیدی را مشخص کنیم که لاتک آن ها را با رنگی به خصوصی نشان دهد. 
\begin{latin}
\lstset{emph={binornd},emphstyle=\color{Magenta}}
\lstinputlisting[language=Matlab, morekeywords={ksdensity}]{Code/prog3.m}
\end{latin}

مثالی دیگر از نوشتن کد مطلب در یک نوشتار. فقط در این حالت می خواهیم برخی از تنظیمات پیش فرض را که قبل از شروع نوشتار تعیین کرده ایم، تغییر دهیم. 
\begin{latin}
\lstinputlisting[numbers=right,language=Matlab, framexleftmargin=5mm, frame=shadowbox,rulesepcolor=\color{Yellow}]{Code/code4.m}
\end{latin}

در ضمن شما می توانید حتی در خود همین نوشتار اصلی خود کد مورد نظرتان را بنویسید. 
\begin{latin}
\begin{lstlisting}[mathescape=true]
// calculate  $a_{ij}$
$a_{ij} = a_{jj}/a_{ij} + \alpha$;
\end{lstlisting}
\end{latin}