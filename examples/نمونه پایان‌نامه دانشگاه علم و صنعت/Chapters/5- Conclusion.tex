\chapter{نتیجه‌گیری و کا‌ر‌های آینده}\label{Chapter5}
در این فصل نتیجه‌گیری کلی بر اساس شبیه‌سازی‌های انجام شده برای آزمایش‌ها مختلف را بررسی می‌کنیم، همچنین کارهایی در تعمیم‌دادن این پژوهش به‌عنوان کارهای مکمل و هم راستا تحت عنوان کارهای آینده مطرح می‌شود. 
\section{نتیجه‌گیری}
همان‌طور که به آن تأکید شد، استفاده از پهپادها در محیط‌های مختلف برای پایش و نظارت یک روش نوین برای استفاده در عملیات مختلف است. یکی از چالش‌های بزرگی که در استفاده از پهپادها با آن روبه‌رو هستیم، موضوع محدودیت انرژی مصرفی پهپادها است، ازاین‌رو با مدیریت بهبودیافته تحرک پهپادها به‌عنوان یک موضوع بااهمیت بالا در صدد بهبود این مسئله بودیم.
باتوجه‌به نتایج به‌دست‌آمده از الگوریتم فراابتکاری \lr{JAYA} و استفاده از روش \lr{K-Means} برای جاگیری و تحرک بهینه پهپادها از طرفی استفاده از \lr{K-Means} به‌عنوان روشی برای جاگیری اولیه پهپادها، مشاهده شد که روش ذکر شده می‌تواند در حد قابل‌قبولی از هدررفت انرژی در تحرکات پهپاد جلوگیری کند، در نتیجه یک عملیات پایش محیط بهبودیافته را تجربه کند. از طرفی روش‌های رایج تحرک نمی‌تواند بهینه بودن استفاده از پهپاد را در یک محیط برای خدمت‌رسانی به کاربران تضمین کند، زیرا، مدل‌های رایج باتوجه‌به ماهیت از پیش تعیین شده و پیروی از الگوی خاص این امر را تصدیق می‌کند، ازاین‌رو استفاده از روش تحرک وقف پذیر با موقعیت مکانی کاربران بر اساس الگوریتم ذکر شده پهپادها می‌تواند تحرک بهینه‌ای را تجربه کنند. قابل‌ذکر است که کاربران زمینی می‌توانند بر اساس الگوهای تحرک از رایج و از پیش تعیین شده کار کنند، به‌عنوان نمونه در این پژوهش از الگوی \lr{RW} برای تحرک کاربران زمینی استفاده شد؛ در این روش کاربران حرکات تصادفی با سرعت تصادفی در بازه از پیش تعیین شده را دارا هستند. بر اساس نتایج به‌دست‌آمده و کارآزمایی مختلف انجام شده به جهت آزمایش دقیق‌تر طبق نتایج به دست آماده با استفاده از ۱۰ پهپاد امکان پوشش‌دهی کاربران با تعداد ۱۵۰ با استفاده از این مدل امکان‌پذیر است. این کارآزمایی‌ها با تعداد مختلفی از پهپادها با تعداد متفاوتی از کاربران انجام شده تا به بهترین شرایط دست پیدا کنیم. مدل پیشنهاد شده این امکان را می‌دهد که در شرایط بلایا و شرایط مختلف از این مدل برای استقرار پهپادها به‌عنوان ایستگاه‌های پایه هوایی به جهت خدمت‌رسانی و پایش محیط مدیریت تحرک بهبودیافته‌ای را تجربه کنیم. این امر باعث شده علاوه به جایابی بهینه پهپاد با استفاده از مدل یاد شده به مدیریت تحرک به جهت بهبود مصرف انرژی پهپادها کمک بسیاری می‌کند.
یکی از عوامل مهم مدیریت تحرک همان‌طور که پیش‌تر به آن اشاره شده، مصرف بهینه و هر چه کمتر انرژی در پهپادها است، تحرک هر چه کمتر پهپادها کمک می‌کند علاوه بر افزایش طول پرواز به پوشش‌دهی بهینه نیز کمک کند، چرا که می‌توان با کمترین تحرک به پوشش‌دهی مناسب با تعداد ذکر شده پهپاد دست‌یافت، باتوجه‌به این که در این پژوهش سعی شده، چندین کارآزمایی مختلف انجام شده تا بتوانیم به بهترین نتایج، یعنی تحرک کمتر و پوشش‌دهی حداکثری در مدیریت تحرک پهپادها دست پیدا کرده‌ایم.
\section{کارهای آینده}
در کارهای آینده که به این موضوع مرتبط است، بر اساس تجربیات این پژوهش، می‌توان به داده‌های مناسب به جهت شبیه‌سازی در این زمینه اشاره کرد، یکی از موضوع‌های بااهمیت بالا در شبیه‌سازی‌ها وجود داده مناسب به‌منظور به‌کارگیری در شبیه‌سازی‌ها است، متأسفانه داده متناسب در زمینه پهپاد بسیار کم و غیرمرتبط است، ازاین‌رو ساخت و ایجاد داده استاندارد این موضوع می‌تواند به‌عنوان یکی از کارهای آینده برای بهبود چالش‌های زمینه تحقیقاتی پهپادها باشد. در شبیه‌سازی این پژوهش سعی شده داده‌های ورودی دقیق و منطبق با دنیای واقعی ایجاد شود، ولی به‌عنوان یک چالش مستقل می‌توان بر روی آن کارکرد.
همچنین می‌توان به روش‌های یادگیری و مقایسه آن با روش‌های فراابتکاری نیز اشاره کرد، روش‌های یادگیری می‌تواند به عنوان‌های روش‌های دیگر برای بهره‌وری مناسب از انرژی محدود پهپادها در یک محیط و در نتیجه تحرک بهبودیافته با حداقل تحرک را تجربه کنیم. 
روش‌های فراابتکاری دیگر نیز می‌تواند به‌عنوان راهکارهای دیگر برای کارهای مرتبط در زمینه پهپادها استفاده شود، یکی از روش‌هایی که استفاده زیادی در تحقیقات شده، روش \lr{PSO} است، این روش معمولاً برای کارهای جایابی پهپادها انجام شده است که می‌توان این روش‌ها را نیز با مقایسه و بهبود توسعه داد.