\chapter{مروری بر کارهای پیشین}
\label{chap:relatedworks}
در این فصل، به بررسی مقالات و مطالعاتی که به طور مستقیم یا غیرمستقیم با موضوع پژوهش حاضر مرتبط هستند، می‌پردازیم. تا یک دید جامع از مطالعات و مقالاتی که در زمینه‌های مرتبط با پژوهش حاضر انجام شده‌اند، به ما ارائه دهد. این بررسی به ما کمک می‌کند تا ضمن شناسایی روش‌ها و مدل‌های موجود، نقاط قوت و ضعف آنها را درک کرده و جایگاه پژوهش خود را در میان آن‌ها مشخص کنیم. همچنین، این مرور به برجسته‌سازی اهمیت تحقیق حاضر و نحوه بهبود یا تکمیل کارهای پیشین کمک می‌کند.

مسئله‌ی بهینه‌سازی شبکه‌های تلفن همراه از جنبه‌های مختلفی موردمطالعه قرار گرفته است که می‌توان آن‌ها را در دسته‌بندی‌های گوناگونی قرار داد. از دیدگاه جمع‌آوری داده به جهت بهینه‌سازی، می‌توان آن را به دو دسته‌ی جمع‌آوری داده سمت 
\gls{UE} 
و جمع‌آوری داده سمت شبکه دسته‌بندی کرد. عملیات  \gls{DriveTest} از جمله روش‌های جمع‌آوری داده سمت \gls{UE} و \gls{MDT} از جمله روش‌های جمع‌آوری داده سمت شبکه است. در 
\autoref{fig:relatedWorksRoadmap}
حوزه‌های مختلف بهینه‌سازی که در ادامه‌ی این فصل مورد بررسی قرار می‌گیرند، قابل‌مشاهده است. 

\begin{figure}
\includegraphics[width=0.65\linewidth]{/Roadmap/mainFig}
\caption{\lofimage{/Roadmap/mainFig}%
حوزه‌های بهینه‌سازی شبکه‌های تلفن همراه از دیدگاه جمع‌آوری داده}
\label{fig:relatedWorksRoadmap}
\end{figure}

\ptext[3]


\section{دیدگاه‌های دسته بندی کارهای پیشین}
\begin{figure}
\begin{subfigure}[b]{.24\textwidth}\centering
\includegraphics[height = .9\linewidth]{/Person/LukmanMedriavinSilalahi}
\caption{\lr{L. Medriavin Silalahi \cite{silalahi2021improvement}}}
\end{subfigure} 
\begin{subfigure}[b]{.24\linewidth}\centering
\includegraphics[height=0.9\linewidth]{/Person/MarcoSousa}
\caption{\lr{Marco Sousa \cite{Sousa2021Analysis}}}
\end{subfigure}
\begin{subfigure}[b]{.24\linewidth}\centering
\includegraphics[height=0.9\linewidth]{/Person/peeraponguthansakul}
\caption{\lr{Peerapong Uthansakul \cite{charoenlap2016prediction}}}
\end{subfigure}
\begin{subfigure}[b]{.24\linewidth}\centering
\includegraphics[height=0.9\linewidth]{/Person/YuliarmanSaragih}
\caption{\lr{Yuliarman Saragih \cite{Aprillia2020RF}}}
\end{subfigure}
\end{figure}

\section{ملاک های ارزیابی و مقایسه}

\section{مرور مطالعات موجود}

\section{جمع بندی}

 البته شامل یک جدول مقایسه ای هم باشد با توجه به دیدگاه های دسته بندی و ملاک ها