\chapter{ادبیات موضوع}\label{Chapter2}
در این فصل توضیحاتی پیرامون مفاهیم اولیه پهپادها، همچنین کارهای مرتبط انجام شده در زمینه کاربرد پهپادها و شبکه پهپادی داده می‌شود. این توضیحات شامل شناخت پهپادها به‌منظور استفاده دقیق‌تر از آن‌ها و ابزارهای مورداستفاده در این پژوهش است. در ادامه کارهای مرتبط به سه بخش کلی تقسیم‌بندی‌شده و کارهای مرتبط در هر بخش که این پژوهش وابستگی با آن‌ها دارد بررسی شده است.
\section{مفاهیم} 

در این بخش از فصل دوم تعریف و مفاهیم کلی در مورد پهپادها، شبکه پهپادی، طبقه‌بندی و مقررات پهپاد مورد پژوهش قرار می‌گیرد، در ادامه دسته‌بندی محیط‌های کاری پهپادها و انواع کاربردهای پهپاد در شبکه نسل آینده مورد بررسی قرار گرفته است. این مفاهیم بنیادی کمک می‌کند تا با شناخت بهتر پهپادها بتوانیم پهپادها را در سناریوهای مختلف شهری یا غیرشهری مورداستفاده قرار دهیم، می‌دانیم که استفاده از پهپاد در محیط‌های شهری و غیرشهری با چالش‌های مختلفی می‌تواند همراه باشد که در این بخش با بررسی خلاصه ویژگی پهپادها می‌توانیم بهتر در این مسیر وارد شویم. مفاهیم اولیه می‌تواند دید مناسبی برای سیرکردن مسیر پژوهش‌های پیشرو را معین سازد، ازاین‌رو علاوه بر توضیحات اجمالی دررابطه‌با پهپادها به دسته‌بندی آن‌ها نیز پرداخته شده است.
\subsection{مقدمه‌ای بر پهپادها} 

همان‌طور پیش‌تر اشاره شد، پرنده هدایت‌پذیر از راه دور (پهپاد) که عموماً به‌عنوان هواپیماهای بدون سرنشین هم نیز شناخته می‌شود، به دلیل تحرک‌پذیری، انعطاف، استقرار سریع و دامنه وسیعی که این حوزه باتوجه‌به ماهیت پهپادها ارایه داده است، موضوع تحقیقات در چند سال اخیر بوده که می‌توان به‌عنوان یک راه‌حل مکمل با اضطراری در شرایط سخت به آن نگاه کرد.
در واقع، پهپادها کاربردهای مختلفی در حوزه‌های گوناگون دارند که شامل نظارت نظامی یا پایش‌های شهری یا غیرشهری، ارتباطات از راه دور به‌عنوان رله ارتباطی، تحویل تجهیزات، پایش محیط، خدمت‌رسانی در شرایط ویژه و عملیات نجات است. بااین‌حال، اکثر تحقیقات متعارف پهپاد معمولاً بر روی موضوعات پیمایش یا تحرک‌پذیری، واپایش و استقرار متمرکز است، از طرفی برخی از کاربردهای انگیزه دهنده معمولاً رباتیک یا نظامی در این حوزه نیز وجود دارد. در مقابل، چالش‌های ارتباطی پهپادها، تحرک‌پذیری بهینه و مدیریت آن، جانمایی و استقرار جز چالش‌هایی است که همچنان باز و موردبحث است.
همان‌طور که می‌دانیم استفاده از پهپادها می‌تواند در محیط‌های مختلف یک راه‌حل جانبی برای حل مشکلات ارتباطی باشد، ازاین‌رو با استفاده از پهپادها می‌توان مشکلات ارتباطی و پوشش‌دهی در شرایط ویژه بلایای طبیعی یا غیرطبیعی را حل کرد. استفاده از پهپادهای می‌تواند یک راه‌حل امیدبخش برای مشکلات پوشش‌دهی و ارتباطی باشد که در این پژوهش اشاره‌ای به این موضوع و حل مشکل مدیریت تحرک و جانمایی بهبودیافته پهپادها به‌منظور خدمت‌رسانی و پوشش‌دهی داده شده است.
در ادامه این فصل دانش نسبی دررابطه‌با پهپادها کسب می‌شود، تا بتوانیم کاربردهای محیطی پهپادها را بهتر درک و مدل کنیم، ارایه این ویژگی‌ها از پهپادها در پایش محیط توسط پهپاد کمک می‌کند تا موضوع‌های مختلف را بهتر درک کنیم. این مفاهیم در زمینه پهپادها زمینه‌ساز برای درک بهتر پایش محیط نیاز است، با استفاده از این مفاهیم می‌توان هر چه‌بهتر پایش محیط با استفاده از پهپاد را بررسی کرد. چراکه پایش محیط با استفاده از پهپاد را با رویکردهای مختلفی معرفی کرده‌ایم که خود نیازمند دانشی از پیش به‌منظور شناخت بهتر ویژگی پهپادها است که زمینه‌ساز بهتر و تکمیل‌کننده این پژوهش می‌شود.
\subsection{ویژگی پهپادها}
پیشرفت‌های بی‌سابقه در فنّاوری پهپادها باعث می‌شود تا به طور گسترده از پهپاد، هواپیماهای کوچک، بالون و غیره، اگر به‌درستی مستقر و فعال شوند، برای اهداف ارتباطی بی‌سیم استفاده شود\cite{bucaille2013rapidly}. به طور خاص، پهپاد می‌تواند راه‌حل‌های ارتباطی بی‌سیم مؤثر و مقرون‌به‌صرفه برای انواع سناریوهای محیطی را فراهم کند. از طرفی پهپاد را می‌توان به‌عنوان ایستگاه‌های پایه هوایی \gls{BS} به کار برد که می‌توانند ارتباطات بی‌سیم قابل‌اطمینان، مقرون‌به‌صرفه و مبتنی بر تقاضا برای مناطق موردنظر را ارایه دهند. از سوی دیگر، پهپادهای متصل به شبکه می‌توانند در هم زیستی با کاربران زمینی در تحویل یا مراقبت پرواز عمل کنند و می‌توانند به‌عنوان مکمل ارتباطی مورداستفاده قرار گیرند.
به طور خاص، هنگامی که پهپادها به‌عنوان ایستگاه‌های پایه هوایی در مقایسه با ایستگاه‌های پایه زمینی معمولی، مورداستفاده قرار می‌گیرند، پهپادها می‌توانند از اتصال شبکه‌های بی‌سیم موجود مانند سلولی و \gls{broadband} پشتیبانی کنند. مزیت استفاده از پهپادها به‌عنوان ایستگاه‌های پایه هوایی توانایی آن‌ها برای تنظیم ارتفاع خود، اجتناب از موانع و افزایش احتمال برقراری ارتباط دید مستقیم \gls{LoS} با کاربران زمینی است. در واقع، به دلیل ماهیت ذاتی آن‌ها از قبیل تحرک، انعطاف‌پذیری و ارتفاع قابل‌تنظیم، ایستگاه‌های پایه پهپادی می‌توانند به طور مؤثری سامانه‌های سلولی موجود را با فراهم‌کردن ظرفیت اضافی برای مناطق پر تراکم و دسترسی به پوشش شبکه در دسترسی به مناطق روستایی تکمیل کنند. یکی دیگر از کاربردهای مهم پهپادها در \gls{IoT} است که تجهیزات آن اغلب قدرت انتقال کم دارند و ممکن است قادر به برقراری ارتباط بیش از حد طولانی نباشند. پهپادها نیز می‌تواند به‌عنوان رله‌های بی‌سیم برای بهبود اتصال و پوشش تجهیزات بی‌سیم زمینی به کار گرفته شوند و همچنین می‌توانند برای سناریوهای نظارت و پایش، مورداستفاده کلیدی برای اینترنت اشیا قرار گیرد. همچنین در مناطق یا کشورهایی که ساخت یک زیرساخت تلفن همراه گران است به‌کارگیری پهپادها بسیار مفید است، چرا که نیاز به ایستگاه‌های پرهزینه و استقرار زیرساخت‌ها را حذف می‌کند. در نهایت نیز پهپادها در سناریوهایی که بر اثر بلایای طبیعی یا غیرطبیعی به‌مانند سیل، زلزله، آتش‌سوزی و یا حتی جنگ، زیر ساخت شبکه ارتباطی سلولی دچار مشکل شود، پهپادها می‌توانند به‌عنوان یک راه‌حل کمکی خدمات ارتباطی را مهیا سازند
\cite{DelCerro2021,agriculture13071375,agriengineering5010022,s20030817}،
اهمیت ویژه‌ای که استفاده از پهپاد در پایش محیط دارد، با پیشرفت پهپادها روبه‌رشد است. یکی از سناریوهای کاربردی که با ویژگی‌های پهپاد در راستا است، استفاده از پهپاد در سناریوهای پایش کشاورزی است، هر چند که باشد اشاره شود این سناریو با ترکیب دیگر فناوری‌ها ادغام می‌شود. در نتیجه می‌توان سناریوهای مختلف را بر اساس ویژگی‌های پهپادهای مورداستفاده معرفی کرد.

\section{طبقه بندی پهپادها}
طبیعتاً، بر اساس کاربرد و اهداف، می‌توان از یک نوع مناسب از پهپاد استفاده کرد که می‌تواند الزامات مختلفی را از جمله \gls{QoS}، شرایط محیط و مقررات کشورهای مختلف برآورده کند را مورداستفاده قرار داد. در حقیقت، برای استفاده صحیح از پهپادها برای هر کاربرد شبکه‌های بی‌سیم، چندین عامل، از جمله قابلیت‌های پهپادها، میزان مصرف انرژی، ارتفاع پرواز آن‌ها و همچنین ویژگی‌های محیطی باید در نظر گرفته شود. به‌طورکلی، پهپادها را می‌توان بر اساس ارتفاع، به سکوهای ارتفاع بالا \gls{Hap} و سکوی ارتفاع پایین \gls{Lap} طبقه‌بندی کرد \cite{al2014modeling}.
می‌توان این طور نتیجه‌گیری کرد که \lr{Lap} می‌تواند برای جمع‌آوری داده‌ها از حسگرهای زمین مورداستفاده قرار گیرد. علاوه بر این، \lr{Lap}ها می‌تواند به‌راحتی در صورت نیاز جایگزین شود یا به‌عبارت‌دیگر استقرار سریع‌تری دارد. در مقابل، \lr{Hap}ها استقامت و پایداری بیشتری دارند و برای پرواز طولانی‌مدت طراحی شده‌اند. به‌علاوه، تجهیزات \lr{Hap} به طور معمول برای فراهم‌کردن پوشش بی‌سیم گسترده برای مناطق جغرافیایی بزرگ ترجیح داده می‌شوند. بااین‌حال، \lr{Hap}ها پرهزینه هستند و زمان استقرار آن‌ها به طور قابل‌توجهی طولانی‌تر از \lr{Lap} است.
پهپادها می‌توانند بر حسب نوع، به بال ثابت و بال چرخشی طبقه‌بندی شود. در مقایسه این دو، پهپادهای بال ثابت وزن بیشتری دارد، سرعت بیشتری داشته و در حالت افقی قادر به حرکت هستند. در مقابل، پهپادهای بال چرخان، می‌توانند در هوا شناور بمانند یا به‌عبارت‌دیگر در یک نقطه مشخص ساکن بمانند \cite{zeng2016wireless}، مروری بر انواع مختلف پهپاد، عملکرد و قابلیت‌های آن‌ها در \autoref{fig:uav-classification} ارایه شده. باید توجه داشت که زمان پرواز یک پهپاد به چندین عامل از جمله منبع انرژی، نوع، وزن، سرعت، مسیر پرواز و سرعت وزش باد بستگی دارد.
\begin{figure}
\includegraphics[width=0.7\linewidth]{UAV classification}
\caption[\lofimage{	UAV classification}%
دسته بندی پهپاد‌ها]{دسته بندی پهپاد‌ها}
\label{fig:uav-classification}
\end{figure}
\subsection{مقررات پهپادها}
مقررات عوامل محدودکننده و مهمی هستند که با استقرار سامانه‌های ارتباطی مبتنی بر پهپاد با آن‌ها مواجه هستیم. علی‌رغم کاربردهای امیدوارکننده در شبکه‌های بی‌سیم، نگرانی‌های متعددی دررابطه‌با حریم خصوصی، امنیت عمومی، اجتناب از برخورد و حفاظت از داده‌ها وجود دارد. در این زمینه، مقررات پهپاد به طور مداوم برای واپایش عملیات پهپاد به‌منظور درنظرگرفتن عوامل مختلف نظیر نوع پهپاد، طیف رادیویی، ارتفاع و سرعت، در نظر گرفته شده است. در \autoref{Table1} تعدادی از مقررات پهپاد برای استقرار در کشورهای مختلف فهرست شده است \cite{8675384}.
در حقیقت شناخت مقررات لازمه اصلی استقرار پهپادها است و اشاره به مقررات پهپادها در این پژوهش می‌توان اهمیت موضوع و گستردگی استفاده از پهپادها را مشخص سازد. پیروی از مقررات پهپادها می‌تواند از موضوعاتی باشد که قبل از استقرار پهپادها در محیط واقعی باید به آن‌ها توجه شود. این قوانین کمک می‌کند که بر اساس، قوانین کشورهای مختلف سناریوهای مختلف را نیز تعریف کنیم.
	\begin{table}\caption{مقررات استقرار پهپاد}\label{Table1}
	\begin{tabular}{llll}
		\hline
		کشور& حداکثر ارتفاع & حداقل فاصله با شهروند & حداقل فاصله تا فرودگاه \\ \hline
		ایران & 120 متر & 25متر و 50 متر & -  \\ \
		آمریکا & 122 متر & - & 8 کیلومتر \\ 
		آفریقا & 46 متر & 50 متر & 10 متر \\ 
		انگلستان & 122 متر & 50 متر & - \\ 
		شیلی & 130 متر & 36 متر & - \\ \hline
	\end{tabular}
\end{table}
\subsection{ارتباط بی‌سیم در پهپاد}
در این بخش به‌منظور آشنایی بیشتر بر استفاده فناوری بی‌سیم در پهپادها و سناریوهای استفاده شده در آن کاربردهایی در این زمینه مورد بررسی قرار می‌گیرد، در ادامه هر کدام از این فناوری‌های مورداستفاده در این حوزه مورد پژوهش قرار گرفته شده است. سناریوهای امنیت عمومی یا پوشش نقطه‌های پر تراکم و همچنین کاربردهای آینده‌گرا مانند استفاده از پهپادها در اینترنت اشیا هستند که هر کدام نیز می‌توان به‌عنوان یک موضوع پژوهشی نیز مطرح شود، هدف از اشاره به این موضوعات در این حوزه آشنایی هر چه بیشتر با زمینه تحقیقاتی پهپاد است. 
\subsubsection{شبکه موبایل نسل پنجم و بالاتر}
یکی از کاربردهای اصلی استفاده از پهپادها در شبکه سلولی نسل پنجم و ششم است، باتوجه‌به ویژگی‌های پهپادها، می‌توانند به‌عنوان ایستگاه‌های پایه هوایی مورداستفاده قرار بگیرند، رویکرد این پژوهش نیز استفاده از پهپادها به‌عنوان ایستگاه‌های پایه هوایی نسل پنجم و ششم است. پهپادها به کمک شبکه موبایل نسل‌های مختلف می‌توان خیلی از چالش‌ها و الزامات این شبکه‌ها را حل کند از جمله ارتباطات دید مستقیم، الزامات کیفیت سرویس، ارتباطات موقت در فعالیت‌های گوناگون و کاربردهای متعددی که می‌توان به تنظیم مقدار بار شبکه در مواقع خاص، امنیت عمومی و پایش محیط در شرایط ویژه و خیل دیگر از ویژگی‌های پهپادها در کنار شبکه موبایل نسل پنجم و بالاتر \cite{zeng2019accessing,li2018uav}.
\subsubsection{استفاده از پهپاد در اینترنت اشیا}
فنّاوری‌های شبکه‌های بی‌سیم به‌سرعت به یک محیط اینترنت کلان تبدیل می‌شوند که باید ترکیبی ناهمگن از وسایلی که از تلفن‌های هوشمند معمولی و تبلت‌ها گرفته تا وسایل نقلیه، حسگرها، پهپادها را یکپارچه کند. شناخت بیشتر کاربردهای اینترنت مانند مدیریت زیرساخت شهرهای هوشمند، بهداشت، حمل‌ونقل و مدیریت انرژی نیاز به اتصال بی‌سیم مؤثر در میان تعداد زیادی از دستگاه‌های اینترنت اشیا دارد که باید به طور قابل‌اطمینان اطلاعات خود را ارایه دهند، طبیعتاً با نرخ‌های بالای داده یا تأخیر بسیار پایین ارسال می‌شود \cite{zanella2014internet,10.1145/3657287,Banerjee}.
ماهیت اینترنت اشیا نیازمند بازنگری اصلی در روشی است که در آن شبکه‌های بی‌سیم سنتی کار می‌کنند. به‌عنوان‌مثال، در یک محیط اینترنت، کارایی انرژی، تأخیر بسیار پایین و قابلیت اطمینان، با سرعت بالا به چالش‌های عمده‌ای تبدیل می‌شوند که معمولاً در موارد استفاده از شبکه سلولی مرسوم مهم نیستند. به طور ویژه، دستگاه‌های اینترنت اشیا بسیار به لحاظ باتری محدود هستند و به طور معمول قادر به انتقال بیش از یک فاصله مشخص به علت محدودیت‌های انرژی خود نیستند.
به‌عنوان‌مثال، در مناطقی که یک پوشش متناوب یا ضعیف توسط شبکه‌های بی‌سیم زمینی را تجربه می‌کنند، تجهیزات اینترنت اشیا محدود شده ممکن است قادر به انتقال داده‌های خود به ایستگاه‌های پایه دورتر به دلیل محدودیت‌های توانی نباشند. علاوه بر این، باتوجه‌به کاربردهای مختلف تجهیزات اینترنت اشیا، ممکن است آن‌ها در محیط‌های دارای زیرساخت بی‌سیم و بدون هیچ زیرساخت سیمی مانند کوه‌ها و مناطق بیابانی مستقر شوند.
در این راستا، استفاده از پهپادها یک راه‌حل امیدوارکننده در بعضی از چالش‌های اینترنت اشیا می‌تواند باشد \cite{10113154}.
در سناریوهای اینترنت اشیا، پهپادها می‌توانند به‌عنوان ایستگاه‌های پایه در حال پرواز به‌منظور فراهم نمودن ارتباطات قابل‌اطمینان بکار گرفته شوند. در حقیقت، به دلیل ماهیت پهپاد و ارتفاع بالای آن‌ها، آن‌ها می‌توانند به طور مؤثر برای کاهش سایه و انسداد به‌عنوان عامل اصلی پایداری سیگنال در پیوندهای بی‌سیم، به کار روند. در نتیجه جای‌گیری مؤثر پهپادها، شبکه ارتباطی بین تجهیزات اینترنت اشیا و پهپادها می‌تواند به طور قابل‌توجهی بهبود یابد. از طرفی، دستگاه‌های اینترنت اشیا محدود به باتری به توان بسیار پایینی نیاز خواهند داشت تا اطلاعات خود را به پهپادها منتقل کنند. به‌عبارت‌دیگر، پهپادها را می‌توان بر روی موقعیت‌های پیاده‌سازی اینترنت اشیا قرار داد که این دستگاه‌ها را قادر می‌سازد تا با استفاده از حداقل قدرت انتقال، با موفقیت به شبکه متصل شوند.
از ویژگی‌های دیگر پهپادها در این زمینه این است که می‌توانند با به‌روزرسانی پویای محل خود بر اساس الگوی فعال‌سازی دستگاه‌های اینترنت اشیا، سامانه‌های \lr{IoT} بسیار گسترده‌ای را ارایه نمایند. این در مقایسه با استفاده از ایستگاه‌های پایه سلولی است که ممکن است به طور قابل‌توجهی برای خدمات تعداد پیش‌بینی‌شده دستگاه‌ها توسعه داده شوند. ازاین‌رو، با بهره‌برداری از ویژگی‌های منحصربه‌فرد، اتصال و کارایی انرژی شبکه‌های اینترنت اشیا می‌تواند به طور قابل‌توجهی بهبود یابد.
\subsubsection{افزایش پوشش و ظرفیت شبکه سلولی} 

فناوری‌های دسترسی بی‌سیم با سرعت بالا به طور مداوم درحال‌رشد بوده و با رشد سریع دستگاه‌های تلفن همراه با قابلیت بالا مانند گوشی‌های هوشمند، تبلت‌ها و ابزارهای سبک، اینترنت اشیا افزایش می‌یابد \cite{zanella2014internet}. به‌این‌ترتیب ظرفیت و پوشش شبکه‌های تلفن همراه موجود به‌شدت تحت‌فشار قرار گرفته است که منجر به ظهور انبوهی از فنّاوری‌های بی‌سیم شده است که به دنبال غلبه بر این چالش هستند \cite{8675384}. 
چنین فناوری‌هایی که شامل ارتباطات  \gls{D2D}، شبکه‌های سلولی فوق متراکم و ارتباطات موج میلیمتری \gls{mmWave} هستند، به طور جمعی به‌عنوان پیوند سامانه‌های سلولی نسل بعدی نسل پنجم و ششم دیده می‌شوند \cite{samarakoon2016ultra,semiari2015context}. بااین‌حال، آن راه‌حل‌ها محدودیت‌هایی دارند. به‌عنوان‌مثال، ارتباط \lr{D2D} بدون شک به برنامه‌ریزی بَسامد بهتر و کاربرد منابع در شبکه‌های تلفن همراه نیاز دارد. در همین حال، شبکه‌های تلفن همراه فوق متراکم با چالش‌های بسیاری دررابطه‌با پیوندهای اصلی شبکه، تداخل و مدل‌سازی شبکه مواجه هستند.
به طور ویژه، ارتباط میلیمتری به ارتباط دید مستقیم محدود می‌شود تا به طور مؤثر قول سرعت بالا، ارتباطات بهتر را ارایه دهد \cite{Ashraf2024}.
با توسعه و استقرار پهپادهای ارتفاع پایین می‌توان یک روش مقرون‌به‌صرفه برای فراهم‌کردن اتصال بی‌سیم به مناطق جغرافیایی با زیرساخت سلولی محدود فراهم کرد. علاوه بر این، استفاده از ایستگاه‌های پایه پهپادی به هنگام استقرار هسته‌های کوچک برای اهداف خدمت‌رسانی رویدادهای موقت (رویدادهای ورزشی و فستیوال‌ها) باتوجه‌به دوره کوتاه زمانی که این رویدادها نیازمند دسترسی بی‌سیم را فراهم ساخت \cite{drones7090555}. همچنین، پهپادها می‌توانند یک راه‌حل پایدار بلندمدت برای پوشش در محیط‌های روستایی فراهم کنند. پهپادها می‌تواند بر اساس اتصال به تقاضا خدمات بی‌سیم نرخ اطلاعات بالا و فرصت تخلیه ترافیک \cite{bor2016new,bucaille2013rapidly,lyu2018uav} در نقاط حساس و در طول رویدادی موقتی مانند بازی‌های فوتبال یا رویدادهای دیگر را فراهم کند. در این زمینه  \lr{AT\&T} و \lr{Verizon} در حال حاضر برنامه‌های متعددی برای استفاده از پهپادها برای فراهم نمودن پوشش اینترنت برای مسابقات ملی فوتبال را اعلام کرده‌اند \cite{fuller2016t}؛ بنابراین، ایستگاه‌های پایه در حال حرکت می‌توانند مکمل مهمی برای شبکه‌های تلفن همراه فوق متراکم باشند.
علاوه بر این، ارتباطات امواج میلیمتری با پهپاد کاربرد دیگری از پهپادها است که می‌تواند پیوندهای ارتباطی دید مستقیم را برای کاربران ایجاد کند. این امر به نوبه خود می‌تواند یک راه‌حل جذاب برای ارایه قابلیت انتقال بی‌سیم با ظرفیت بالا باشد، درحالی‌که استفاده از مزایای هر دو ترکیب پهپادها با امواج میلیمتری و تکنیک‌های \gls{MIMO} می‌تواند یک نوع جدید از شبکه تلفن همراه را برای ارایه خدمات بی‌سیم با ظرفیت بالا، ایجاد نماید.
همچنین از طرفی می‌تواند به شبکه‌های وسایل نقلیه‌ای نیز کمک کند. به‌عنوان‌مثال، به‌خاطر تحرک‌پذیری و ارتباطات دید مستقیم، پهپادها می‌توانند انتشار اطلاعات سریع را در میان دستگاه‌های زمینی تسهیل کنند. علاوه بر این، پهپادها می‌توانند به طور بالقوه قابلیت اطمینان ارتباطات بی‌سیم را در ارتباطات \lr{D2D} و \gls{V2V} افزایش دهند. به طور خاص، پهپادها در حال پرواز می‌توانند به پخش اطلاعات مشترک به دستگاه‌های زمینی کمک کنند در نتیجه تداخل شبکه‌های زمین را با کاهش تعداد ارسال‌های بین دستگاه‌ها افرایش می‌دهد \cite{8638578}.
علاوه بر این، ایستگاه‌های پایه پهپادی می‌توانند از پیوندهای هوا به زمین برای خدمات دیگر پهپاد به هم متصل برای کاهش بار در شبکه زمینی استفاده کنند.
برای سناریوی شبکه‌های سلولی مرسوم، واضح است که استفاده از پهپادها کاملاً می‌تواند منطقی باشد، زیرا ویژگی‌های اصلی آن‌ها در جدول‌های \autoref{Table3} و \autoref{Table4} مانند چابکی، تحرک، انعطاف‌پذیری و ارتفاع قابل‌تنظیم می‌تواند انگیزه خوبی برای استفاده باشد. در حقیقت با بهره‌برداری از این ویژگی‌های منحصربه‌فرد و نیز ایجاد ارتباطات دید مستقیم، پهپادها می‌توانند عملکرد شبکه‌های بی‌سیم موجود در زمینه پوشش، ظرفیت، تأخیر و کیفیت کلی خدمات را افزایش دهند. این سناریوها به‌وضوح نویدبخش هستند و می‌توان پهپادها را به‌عنوان بخشی جدایی‌ناپذیر از شبکه‌های سلولی نسل پنجم و ششم ببینند، چرا که فناوری به بلوغ بیشتری می‌رسد و سناریوهای عملیاتی جدید پدیدار می‌شوند.
\begin{table}[]\caption{ایستگاه پایه زمینی در مقایسه با هوایی}\label{Table3}
	\begin{tabular}{ll}
		\hline
		ایستگاه پایه هوایی & ایستگاه پایه زمینی \\ \hline
		استقرار به طور طبیعی سه بعدی است & استقرار به طور معمول دو بعدی است \\ 
		کوتاه مدت، اغلب در حال تغییر & عمدتا استقرار طولانی مدت و دائمی \\ 
		مکان‌ها عمدتا بدون محدودیت & تعداد مکان‌های انتخاب شده کم \\ 
		دارای بعد متحرک & ثابت و غیر متحرک \\ \hline
		
	\end{tabular}
	
\end{table} 

\begin{table}[]\caption{شبکه هوای در مقایسه با شبکه زمینی}\label{Table4}
\begin{tabular}{ll}
	\hline
	شبکه های پهپادی & شبکه های زمینی \\ \hline
	کمبود طیف & کمبود طیف \\ 
	محدودیت شدید انرژی & محدودیت‌های انرژی کاملاً مشخص \\ 
	ارتباط سلول‌های مختلف & به طور عمده ارتباط ثابت \\ 
	محدودیت‌های زمان شناور و پرواز & بدون محدودیت زمانی  \\ \hline
	
\end{tabular}

\end{table} 

\subsection{پهپادها به‌عنوان ایستگاه پایه در امنیت عمومی} 

بلایای طبیعی مانند سیل، طوفان و برف شدید اغلب منجر به پیامدهای مخرب در بسیاری از کشورها می‌شوند. در طی بلایای طبیعی در مقیاس گسترده و وقایع غیرمنتظره، شبکه‌های ارتباط زمینی موجود ممکن است آسیب یا حتی به طور کامل تخریب شوند، به طور خاص، ایستگاه‌های پایه سلولی و زیرساخت ارتباطات زمینی را می‌توان در طی بلایای طبیعی در خطر دید. در چنین سناریوهایی، نیاز حیاتی به ارتباطات ایمنی عمومی بین نیروهای امداد و نجات در عملیات جستجو وجود دارد.
در نتیجه، یک سامانه ارتباطات اضطراری قوی، سریع و توانا برای فعال‌کردن ارتباطات مؤثر در طول عملیات ایمنی عمومی موردنیاز است. در سناریوهای امنیت عمومی، چنین سامانه ارتباطی قابل‌اطمینان نه‌تنها به بهبود اتصال بلکه برای نجات جان افراد کمک می‌کند. علاوه بر این بلایای غیرطبیعی که عامل انسانی دارند، به‌مانند جنگ و آتش‌سوزی نیز از ویژگی‌های پهپادها و استقرار پهپادها در این شرایط می‌توانند بهره‌مند شوند، زیر ممکن است در این شرایط ارتباطات رادیوی یا سلولی در دسترس نباشد یا از دسترس بنا بر شرایط ویژه خارج شده باش، در این شرایط پهپادهای می‌توانند برای برقراری ارتباط نقش مهمی داشته باشند. دراین‌بین می‌توانیم به خدمت‌رسانی به نیروهای امدادی اشاره کنیم.
در این راستا، در ایالات متحده آمریکا برای ایجاد شبکه بی‌سیم پهن باند پرسرعت برای ارتباطات \gls{Public safety} تأسیس شد. فنّاوری‌های بی‌سیم پهن باند برای سناریوهای امنیت عمومی عبارت‌اند از: 
\begin{itemize}
	\item 
	پایداری بلندمدت شبکه بی‌سیم، 
	\item 
	شبکه سلولی، 
	\item 
	ارتباطات ماهواره‌ای، 
	\item 
	سامانه‌های ایمنی عمومی.
\end{itemize}
بااین‌حال، این فنّاوری‌ها انعطاف‌پذیری، خدمات بی‌درنگ و سازگاری سریع با محیط را در طی بلایای طبیعی فراهم نمی‌کنند. در این راستا، استفاده از شبکه‌های هوایی مبتنی بر پهپاد به‌عنوان یک راه‌حل امیدبخش برای ارتباطات بی‌سیم قابل‌اطمینان در سناریوهای امنیت عمومی معرفی می‌شود. 
ازآنجایی‌که پهپادها نیازی به زیرساخت گران‌قیمت ندارند، به‌راحتی می‌توانند پرواز کنند و به طور پویا موقعیت خود را تغییر دهند، تا ارتباطات موردنیاز کاربران را در شرایط اضطراری فراهم آورند. در واقع، باتوجه‌به ویژگی‌های منحصربه‌فرد منطقه مانند تحرک، استقرار انعطاف‌پذیر و بازپیکربندی سریع آن‌ها می‌توانند به طور مؤثر شبکه‌های ارتباطی ایمنی عمومی را ایجاد نمایند. برای مثال، می‌توان از پهپاد به‌عنوان ایستگاه‌های هوایی متحرک جهت ارایه اتصال پهن باند در نواحی با زیرساخت مخابراتی زمینی آسیب‌دیده استفاده نمود. علاوه بر این، پهپادها می‌توانند به طور مداوم برای ارایه پوشش کامل به یک منطقه در حداقل زمان ممکن کارساز باشند، بنابراین استفاده از ایستگاه‌های پایه هوایی می‌تواند راه‌حلی مناسب برای ایجاد ارتباط سریع و فراگیر در سناریوهای امنیت عمومی باشد.
\subsection{شبکه اقتضایی هوایی}
یکی از موارد استفاده در \gls{FANET} است که در آن پهپاد به روش اقتضایی باهم ارتباط برقرار می‌کنند. با وجود تحرک آن‌ها، فقدان واپایش مرکزی و طبیعت سازمان‌دهی خود، شبکه اقتضایی می‌تواند اتصال و محدوده ارتباطی را در مناطق جغرافیایی با زیرساخت سلولی محدود گسترش دهد\cite{zafar2016flying}. در همین حال، \lr{FANET} نقش مهمی در کاربردهای مختلف مانند نظارت ترافیک، سنجش از راه دور، نظارت مرزی، مدیریت بلایای طبیعی، مدیریت کشاورزی، مدیریت حریق، ایفا می‌کند \cite{kundu2024trust}.
در همین حال، در مقایسه پهپادهای کوچک نسبت به پهپادهای بزرگ‌تر می‌توان به موارد زیر اشاره کرد.
\begin{itemize} 
	\item 
	مقیاس‌پذیری: پوشش عملیاتی می‌تواند به‌راحتی با افزودن پهپادهای جدید و اتخاذ طرح‌های مؤثر مسیریابی پویا افزایش یابد، 
	\item 
	هزینه: هزینه نگهداری پهپادهای کوچک بسیار کم‌تراز هزینه یک پهپاد با سخت‌افزار پیچیده است.
	\item 
	پایداری: در شبکه‌هایی که از پهپادهای کوچک‌تر استفاده می‌شود پایداری بیشتر خواهیم داشت.
\end{itemize}
\subsection{استفاده از پهپاد در شهر هوشمند}
دررابطه‌با \gls{Smart City} به طور مؤثر بسیاری از فناوری‌های مختلف را برای پیاده‌سازی در خود جای‌داده است، یک شبکه تلفن همراه قابل‌اعتماد و مقادیر عظیمی از داده‌ها برای یکپارچه‌سازی نیاز است \cite{ferdowsi2017colonel}. برای این منظور، پهپادها می‌توانند چندین مورد کاربرد بی‌سیم را در شهرهای هوشمند فراهم کنند. از یک سو آن‌ها می‌توانند به‌عنوان ابزارهای جمع‌آوری داده استفاده شوند که می‌توانند مقادیر زیادی از داده‌ها را در مناطق مختلف جغرافیایی در یک شهر جمع‌آوری کرده و آن‌ها را برای اهداف تجزیه‌وتحلیل داده‌های بزرگ به واحدهای \gls{Cloud} مرکزی تحویل دهند. از طرف دیگر، ایستگاه‌های پایه پهپاد می‌توانند به‌سادگی پوشش شبکه تلفن همراه در یک شهر را افزایش دهند یا به شرایط اضطراری خاص پاسخ دهند \cite{app13179881,drones7020079}.
برای کمک به پوشش کاربران شبکه و تلاش برای برنامه‌ریزی بَسامد، از پهپادها نیز می‌توان استفاده نمود. کاربرد کلیدی دیگر پهپادها در شهرهای هوشمند توانایی آن‌ها برای عمل به‌عنوان سامانه‌های \gls{Cloud Computing} متحرک است\cite{jeong2017mobile}. دراین‌رابطه، پهپادهای مشخصی می‌تواند برای دستگاه‌هایی که قادر به انجام کارهای سنگین محاسباتی نیستند، محاسبه \gls{Fog} و تخلیه بار را فراهم آورند. به‌طورکلی، پهپادها بخش مهمی از شهرهای هوشمند هستند، هم از نظر بی‌سیم و هم از نظر عملیاتی \cite{Siddiqi,8675178}.
\subsection{یادگیری ماشین}
به کمک \gls{Machine Learning}، سامانه‌ها قادرند تا عملکرد خود را به طور خودکار از محیط خود و تجربه گذشته خود بهبود بخشند. یادگیری ماشینی می‌تواند به طور بالقوه برای طراحی و بهینه‌سازی سامانه‌های ارتباطی بی‌سیم پهپاد استفاده شود. به‌عنوان‌مثال، با استفاده از الگوریتم‌های یادگیری تقویتی، پهپادها می‌توانند به طور پویا موقعیت خود، مسیرهای پرواز و واپایش حرکت را برای خدمت به کاربران زمینی خودتنظیم کنند. در این حالت، پهپادها قادر به انطباق سریع با محیط‌های پویا به روش خودسازماندهی و به طور خودکار بهینه‌کردن مسیر خود هستند. علاوه بر این، با استفاده از تکنیک‌های شبکه‌های عصبی و انجام تجزیه‌وتحلیل داده‌ها، می‌توان رفتار کاربران را پیش‌بینی کرد و به طور مؤثر مستقر و اجرا کرد. به‌عنوان‌مثال، ابزارهای یادگیری ماشین قادر به پیش‌بینی تحرک کاربران و توزیع بار آن‌ها هستند که می‌تواند برای انجام استقرار بهینه و برنامه‌ریزی مسیر پهپادها مورداستفاده قرار گیرد. چنین اطلاعاتی در مورد الگوی تحرک کاربران و توزیع ترافیک مشخصاً در طراحی سامانه‌های هوایی پهپادی حافظه نهان مفید است. یادگیری ماشین می‌تواند برای یادگیری نقشه‌های محیط رادیویی و ایجاد یک مدل کانال سه‌بعدی با استفاده از پهپادها بکار رود. متعاقباً می‌توان از چنین نقشه‌های محیط رادیویی برای استقرار بهینه و اجرای سامانه‌های ارتباطی پهپاد استفاده کرد \cite{chen2017machine,challita2019machine,s19235170,s19235170}.
\subsection{نظریه بهینه‌سازی متمرکز برای ارتباط پهپادی}
با وجود استقرار پذیری ذاتی پهپادها پیش‌بینی می‌شود که پهپادها در ابتدا بر روی واپایش متمرکز اتکا خواهند کرد. این امر به‌ویژه برای کاربردهایی مانند افزایش ظرفیت شبکه تلفن همراه مهم است که در آن اپراتورهای تلفن همراه ممکن است مایل به ترک واپایش شبکه خود در طول آزمایش اولیه یک فناوری مانند پهپادها نباشند. در چنین سناریوهایی، بسیاری از مسائل شناسایی شده به طور طبیعی مستلزم نیاز به فرموله‌کردن و حل مشکلات بهینه‌سازی متمرکز هستند. چنین مشکلاتی می‌توانند در سطح یک \gls{Cloud} اجرا شوند (همان‌طور که در شبکه دسترسی رادیویی به کمک ابر انجام می‌شود) \cite{peng2016recent}.
شایان‌ذکر است که مواردی که از مسئله بهینه‌سازی شبکه سلولی معمولی زمینی آموخته‌ایم، می‌تواند در ارتباطات پهپاد بسیار مفید واقع شود. برای مثال، رویکردهای کلاسیک مانند بهینه‌سازی سه‌بعدی را می‌توان برای بهینه‌سازی مکان سه‌بعدی و مسیر پهپاد بکار برد \cite{fiacco1990nonlinear}.
بااین‌حال، بسیاری از مشکلات شناسایی شده در این حوزه نیازمند روش‌های بهینه‌سازی پیشرفته‌تری هستند. به‌عنوان‌مثال، هنگام تحلیل مشکلات ارتباط کاربر، یکی به طور طبیعی به چالش کشیدن مسائل برنامه‌نویسی عدد صحیح مختلط ختم می‌شود که با استفاده از الگوریتم‌های سنتی، مانند آن‌هایی که برای بهینه‌سازی محدب به کار می‌روند، قابل‌حل نمی‌باشد. در این زمینه، ابزارهای ریاضی پیشرفته مثل نظریه انتقال بهینه می‌توانند راه‌حل‌های مناسب برای طیف وسیعی از مشکلات ارتباط سلول فراهم کنند که به دنبال بهینه‌سازی زمان پرواز، توان عملیاتی و انرژی کارایی شبکه‌های بی‌سیم پهپادی هستند.
\subsection{مدیریت منابع در شبکه‌های پهپاد}
مدیریت منابع یک مسئله تحقیق کلیدی دیگر در سامانه‌های ارتباطی مبتنی بر پهپاد است. به طور خاص، نیاز به یک چارچوب وجود دارد که می‌تواند به طور پویا منابع مختلفی از جمله پهنای باند، انرژی، انتقال قدرت، زمان پرواز پهپاد و تعداد پهپاد را در میان افراد دیگر مدیریت کند. برای مثال، چگونه به طور تطبیقی قدرت انتقال و خط سیر یک پهپاد را تنظیم کنید که به کاربران زمینی کمک کند. در این حالت، یک مسئله کلیدی ارایه مکانیزم‌های تخصیص پهنای باند بهینه است که می‌تواند تأثیر مکان‌ها، تحرک‌پذیری، تداخل دید مستقیم و توزیع ترافیک کاربران زمین را به دست آورد. همچنین، نیاز به طراحی تکنیک‌های برنامه‌ریزی کارآمد برای کاهش تداخل بین ایستگاه‌های پایه هوایی و زمینی در شبکه تلفن همراه با پهپاد وجود دارد. علاوه بر این، باید اشتراک طیفی پویا را در یک شبکه ناهمگن از ایستگاه‌های پایه را تحلیل کرد. در نهایت، به‌کارگیری باندهای بَسامد مناسب (\lr{WiFi}، باندهای \lr{LTE}) برای عملیات پهپاد مشکلات طراحی مهمی است \cite{ismail2024line,8807386}.