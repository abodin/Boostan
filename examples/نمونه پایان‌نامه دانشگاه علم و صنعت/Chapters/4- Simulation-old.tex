\chapter{مدلسازی و ارزیابی}\label{Chapter4}
در این فصل توضیحاتی پیرامون مدلسازی جایابی و تحرک بهینه پهپاد‌ها بر پایه الگوریتم فرا ابتکاری \lr{JAYA} داده می‌شود، قابل ذکر است که مدلسازی یا شبیه سازی این مدل با استفاده از زبان برنامه نویسی پایتون انجام گرفته شده است. 
\section{مقدمه}
در شبیه سازی سعی شده سناریو‌های مختلف ارزیابی شود، سناریو‌های مختلف را می‌توان برای محیط‌های مختلف در نظر گرفت به طور عمومی این سناریو‌ها می‌توانند در محیط‌‌های مختلفی در نظر گرفته شود، تعداد پهپادها و کاربران در سناریو برای شبیه سازی دقیق‌تر متفاوت می‌باشد، به عبارت دیگر با آزمایش تعداد مختلف به دنبال ارزیابی بهتر مدل ارایه شده هستیم، سناریوهای پایه شبیه سازی شامل ..... است. 
پارامتر‌‌های مورد هدف ارزیابی در سناریو‌های مختلف، با توجه به اینکه انرژی یکی از پارامترهای مهم استفاده پهپاد در سناریو‌های مختلف می‌باشد، انرژی یکی از پارامتر‌های اصلی ارزیابی مدل ارایه شده در شبیه سازی است، از طرفی میزان گذردهی، اتلاف بسته‌های ارسالی و تاخیر نیز شبیه سازی شده است.

\section{مراحل شبیه سازی}
در شبیه سازی مدل ارایه شده از زبان برنامه نویسی \lr{Python} با استفاده از کتابخانه‌های مورد نیاز، مدل‌سازی ریاضی انجام شده، سپس نتایج به دست آمده در قالب نمودار‌‌هایی برای ارایه نتایج ترسیم شده است، نمودار‌های شامل اتلاف انرژی پهپادها، میزان گذردهی، میزان اتلاف بسته و تاخیر نشان داده شده است.

\subsection{سناریو شبیه سازی}
سناریو‌های مورد ارزیابی در نظر گرفته شده برای بهره‌مندی از قابلیت تحرک در یک سناریو پایش و نظارت بر محیط را می‌توان با عملیات‌های مختلف مورد ارزیابی قرار گیرد. سناریویی را فرض می‌کنیم که در آن \lr{BS}های زمینی ممکن است به دلیل آسیب تجهیزات یا قطع برق غیرفعال شوند. در چنین شرایطی، \lr{UAV-BS} می‌توانند نقش حیاتی در بازیابی سریع و کارآمد خدمات ارتباطی داشته باشند. چندین \lr{BS} نصب شده بر روی پهپاد در مناطق هدف مستقر می‌شود و خدمات ارتباطی بی سیم موقت را ارائه می‌دهند. یکی از موارد استفاده زمانی به وجود می‌آید که نیاز اساسی به انعطاف پذیری، دسترسی مطمئن و تاخیر کم در چارچوب شبکه‌های \lr{6G} وجود داشته باشد. در پاسخ، پهپادها به عنوان زیرساخت \lr{5G} یا \lr{6G} در محیط‌های شهری یا غیر شهری به ارائه راه‌حلی بالقوه به عنوان سلول‌های کوچک و کم هزینه عمل ‌کنند، همانطور که در \cite{Bajracharya2022} مشخص شده است. این پهپادها در هنگام بلایای طبیعی، مانند طوفان‌های شدید، نقش حیاتی ایفا می‌کنند و در نتیجه تلاش‌های امدادی و قابلیت‌های ارتباط رادیویی را افزایش می‌دهند. به طور ویژه، پیامدهای طوفان کاترینا در ایالات متحده بر نیاز ضروری برای چنین فناوری‌ ارتباطی تاکید می‌کند همچنین در کشور ایران به توجه به وجود بلایای طبیعی یا غیرطبیعی که باعث مشکلات متعددی می‌شود، نیاز به چنین فناوری ارتباطی مورد توجه و تاکید است. علاوه بر این، پهپادها می‌توانند پیامدهای بلایای غیرطبیعی مانند جنگ یا آتش‌سوزی را که می‌توانند ارتباط رادیویی و دسترسی به اینترنت را مختل کنند، همانطور که در \cite{Parvaresh2023} و \cite{Liao2022} بررسی شده‌اند، بهبود بخشند. هدف این تحقیق بررسی این سناریوها از طریق شبیه‌سازی است. در شکل \ref{fig:scenario2} دو سناریو نشان دهنده بلایای طبیعی و غیرطبیعی ارائه شده است. \lr{UAV-BS} می‌تواند به طور موثر در این موقعیت‌های فاجعه بار کمک کند، در این شبیه سازی، این سناریوها برای نمایش موقعیت‌های دنیای واقعی که در آن \lr{UAV-BS}ها استفاده می‌شوند، مدل سازی می‌شوند.
%\input{Chapters/algorithm}

%\hl{ algorithm, including data, formulation, solution}
\begin{figure}%[H]
	\centering
	\includegraphics[width=0.83\textwidth]{Pictures/Scenario.pdf}
	\caption{سناریوهای بلایای غیر طبیعی و بلایای طبیعی.}
	\label{fig:scenario2}
\end{figure}


\subsection{تولید داده}








از یک اسکریپت پایتون برای تولید مجموعه داده اولیه با فیلدهای ذکر شده برای چهار آزمایش استفاده شده است. به روشی مشابه شبیه سازی مونت کارلو، ۱۰۰۰ تغییر مجزا از مجموعه داده اولیه برای هر آزمایش تولید می شود. این رویکرد مدل را قادر می‌سازد تا بر روی هر تغییر داده خروجی‌های حاصل مشاهده شود، در نتیجه میانگینی از این خروجی‌ها محاسبه خواهد شد، با توجه به نیازهای محاسباتی قابل توجه مرتبط با ابعاد بالا برای I، J و T، این مدل تنها بر روی ۱۰ تغییر مجموعه داده اول برای هر آزمایش اجرا  شده و خروجی‌های آن‌ را ارزیابی می‌کند. سپس میانگین این ۱۰ فایل محاسبه شده و در نتیجه یک فایل واحد که نمایانگر میانگین‌ داده‌ها است نشان داده می‌شود.  شبیه‌سازی فقط یک بار بر روی این میانگین داده اجرا می‌شود. مشاهده می‌شود که خروجی اجرای ۱۰ بار شبیه سازی و میانگین گیری نتایج معادل یک بار اجرای شبیه سازی بر روی داده‌های متوسط ​​است. این رویکرد امکان حفظ منابع محاسباتی را با حفظ دقت نتایج فراهم می کند. همچنین بر اساس تجزیه و تحلیل انجام شده، شبیه‌سازی با استفاده از میانگین داده‌های حاصل از ۱۰۰۰ تغییر انجام می‌شود و خروجی‌های حاصل مستند می‌شوند. این خروجی ها در ادامه به تفصیل آمده است.


\subsection{طراحی آزمایشات}
اجرای یک آزمایش فاکتوریل کامل و جامع نیاز به برنامه ریزی دقیق و ساختار سیستماتیک برای ارزیابی پویایی و اثرات  \lr{UE} و \lr{UAV-BS} در یک شبکه ارتباطی دارد. این تحقیق به بررسی این موضوع می‌پردازد که چگونه مقادیر مختلف UE (۱۵۰، ۷۵، ۵۰) و پهپادها (۲، ۳، ۵، ۱۰) بر معیارهای عملکرد شبکه، مانند از دست دادن بسته، میزان گذردهی، تاخیر، انرژی مصرفی تحرک و ارتباط پهپادی تاثیر گذار است.

طراحی آزمایشی شامل اجرای 12 آزمایش مجزا، کنترل دقیق و دستکاری متغیرهای مستقل برای مشاهده اثرات آنها بر متغیرهای وابسته است. ماتریس طراحی آزمایش شامل دو بعد است: اول که در آن تعداد \lr{UAV-BS} ها در حالی که \lr{UE}ها در نوسان هستند ثابت می‌ماند و دیگری که در آن \lr{UE}ها ثابت هستند در حالی که \lr{UAV-BS}ها تغییر می‌کند. هر آزمایش(کارآزمایی) ترکیب‌های خاصی از \lr{UE} و \lr{UAV-BS} را برای انجام آزمایش‌ها به طور موثر جدا می‌کند.

در مجموعه اولیه آزمایش‌ها، جایی که تعداد \lr{UAV-BS} ثابت نگه داشته می‌شود، مقادیر مختلفی از \lr{UE} برای ارزیابی تأثیر آنها بر معیارهای عملکرد استفاده می‌شود. برعکس، در مجموعه آزمایش‌های بعدی، تعداد \r{UE‌}ها ثابت می‌ماند در حالی که تعداد \lr{UAV-BS} برای درک تأثیر آنها بر معیارهای ذکر شده متفاوت است.
\begin{figure}%[!htb]
	\centering
	\includegraphics[width=0.9\textwidth]{Pictures/Postions-1.pdf}
	\caption{موقعیت‌های بهینه پهپادها در حین خدمت به تعداد مختلف کاربران}
	\label{fig:PostionsPlots}
\end{figure}

\section{نتیجه‌گیری}
این بخش نتایج شبیه‌سازی به‌دست‌آمده از توصیف مسئله بهینه‌سازی پیشنهادی و الگوریتم را مورد نتیجه‌گیری قرار می‌دهد، که شامل انرژی حرکتی و ارتباطی، نسبت از دست دادن بسته‌ها، تاخیر، گذردهی، و تعداد \lr{UE‌}های متصل به هر پهپاد می‌شود.

\begin{itemize}
	\item \textit{مکان‌یابی و تحرک بهینه پهپاد‌ها در سناریو‌های مختلف}
	مدل بهینه‌سازی ارائه ‌شده با هدف تعیین موقعیت‌های بهینه \lr{UAV-BS} برای یک طراحی آزمایشی جامع شامل ۵۰، ۷۵ و ۱۵۰ \lr{UE}، با تعداد متفاوتی از \lr{UAV-BS}ها در ۲، ۳، ۵ و ۱۰ واحد زمانی که نشان دهنده تغییرات تحرکی پهپاد‌ها به منظور تحرک پذیری بهینه برای خدمت‌رسانی بهتر به \lr{UE}ها است. این محاسبات در پیکربندی‌های مختلف داده ورودی که ۱۰ مرحله زمانی را شامل می‌شود، انجام می‌شود. شکل \ref{fig:PostionsPlots} تحرک بهینه \lr{UAV-BS} را در ۱۲ آزمایش و ارتباط آن‌ها با \lr{UE} را در مرحله زمانی اولیه برای هر سناریو نشان می‌دهد. مطابق شکل \ref{fig:PostionsPlots} هر \lr{UAV-BS} با یک مربع رنگی متفاوت نشان داده شده است. \lr{UE}ها با شناسه آن‌ها به صورت نقاط کوچک رنگی مطابق با \lr{UAV-BS} متصل آن‌ها نشان داده می‌شوند. \lr{UE}های قطع شده به رنگ خاکستری نشان داده شده‌اند. تصویر \ref{fig:PostionsPlots} نشان می‌دهد که وقتی تعداد \lr{UAV-BS} روی ۱۰ قرار می‌گیرد، احتمال پوشش ۱۰۰ درصد  به دست می‌آید که عملاً همه \lr{UE}ها را پوشش می‌دهد. در حالی که موقعیت‌های \lr{UAV-BS} و \lr{UE}ها در ۱۰ مرحله زمانی برای هر آزمایش نمایش داده می‌شود، به دلیل محدودیت فضا، تنها موقعیت‌ها در زمان ۰ در این پژوهش آورده شده است.
	\item \textit{نرخ از دست دادن بسته}
	در سناریوهای ارتباط بی سیم مانند ارتباطات مبتنی بر پهپاد، کاهش نرخ تلفات بسته برای افزایش عملکرد شبکه بسیار مهم است. همانطور که \lr{UAV-BS} شروع به جابجایی از موقعیت‌های اولیه خود می‌کند، \lr{UE}های مختلف با آن‌ها ارتباط برقرار می‌کنند. جدول زمانی برای \lr{UAV-BS} برای رسیدن به مجاورت \lr{UE} در تعیین اندازه گیری دقیق از دست دادن بسته حیاتی می‌شود. تخمین نرخ اتلاف بسته شامل یک رابطه خطی است که عوامل مختلفی مانند تعداد \lr{UEهای متصل، نرخ داده \lr{UE}ها و نزدیکی آن‌ها به \lr{UAV-BS} را در نظر می‌گیرد. در تصویر \ref{fig:PacketLossPlot}، نرخ‌های تلفات بسته برای ۲، ۳، ۵ و ۱۰ \lr{UAV-BS} در طول زمان در آزمایش‌های مختلف نشان داده شده‌اند که هر کدام شامل تعداد متفاوتی از \lr{UE} (۷۵ ،۵۰ و ۱۵۰) است. با مشاهده نمودارهای الف، ب، ث تصویر \ref{fig:PacketLossPlot}، آشکار است که میانگین تلفات بسته برای تعداد پهپاد ۲، ۳ و ۵ در مقایسه با UEها به ترتیب سناریوهای ۵۰ و ۷۵ بیشتر از سناریوهای ۵۰ و ۷۵ است. با این حال، هنگام استفاده از ۱۰ \lr{UAV-BS} برای پوشش این مجموعه از \lr{UE}ها، مقادیر تلفات بسته به دلیل پوشش افزایش یافته تقریباً به همان سطوح همگرا می‌شوند. با استقرار ۱۰ \lr{UAV-BS} پوشش کامل برای هر زیر مجموعه ۵۰، ۷۵ و ۱۵۰ \lr{UE} به دست می‌آید. به طور قابل‌توجه، واریانس مشاهده‌شده در نمودارهای مربوط به از دست دادن بسته‌ها به ماهیت تصادفی درخواست‌های داده از \lr{UE} در موارد زمانی مختلف نسبت داده می‌شود.
		\begin{figure}%[H]
			\centering
			\includegraphics[width=0.83\textwidth]{Pictures/pktloss0.pdf}
			\caption{میزان از دست رفتن بسته‌ها در طول آزمایش و تعداد مختلف کاربر}
			\label{fig:PacketLossPlot}
		\end{figure}
	\item \textit{تاخیر}
	تاخیر بین \lr{UE}ها و \lr{UAV-BS}ها در بسیاری از کاربردهای مدرن، به ویژه در حوزه سنجش از راه دور، نظارت و ارتباطات از اهمیت بالایی برخوردار است. تاخیر کم برای تصمیم گیری و کنترل در زمان واقعی بسیار مهم است، زیرا مستقیماً بر پاسخگویی \lr{UAV-BS} به دستورات \lr{UE} و تحویل به موقع داده ها یا خدمات تأثیر می‌گذارد. تاخیر کاهش یافته، کارایی \lr{UAV-BS} را در کارهایی مانند واکنش به بلایا، که در آن تصمیم‌گیری و ارتباطات سریع می‌تواند جان انسان‌ها را نجات دهد، یا در سرویس‌های تحویل مستقل \lr{UAV-BS}، که در آن ناوبری سریع و اجتناب از موانع ضروری است، افزایش می‌دهد. به طور خلاصه، به حداقل رساندن تاخیر بین \lr{UE} و \lr{UAV-BS} برای اطمینان از کارایی، ایمنی و قابلیت اطمینان عملیات پهپاد در طیف وسیعی از صنایع و کاربردها حیاتی است.
	جدول \ref{tab:avglatency} مقایسه میانگین تاخیر مدل بهینه‌سازی پیشنهادی را برای یک طرح فاکتوریل کامل نشان می‌دهد. همانطور که در جدول \ref{tab:avglatency} نشان داده شده است، متوسط ​​تاخیر برای هر مجموعه از \lr{UE}ها و \lr{UAV-BS} مربوطه آن‌ها تقریبا ۱ میلی ثانیه باقی می‌ماند.
	تاخیر در این زمینه به طور قابل توجهی تحت تأثیر جداسازی فضایی بین \lr{UE}ها و \lr{UAV-BS}ها است. یک همبستگی مستقیم وجود دارد که در آن افزایش فاصله بین \lr{UE}ها و \lr{UAV-BS}ها منجر به تاخیر مشاهده شده بالاتر می‌شود. مدل بهینه‌سازی پیشنهادی به طور استراتژیک فاصله بین \lr{UE}ها و \lr{UAV-BS}ها را به حداقل می‌رساند و در نتیجه به طور ذاتی تاخیر را کاهش می‌دهد. کاهش در جدایی فضایی منجر به کاهش قابل‌توجه در تاخیر می‌شود، که نمونه‌ای از کارآمدی مدل پیشنهادی در دستیابی به تاخیر ارتباطی کمتر است. این رویکرد بهینه‌سازی نقش مهمی در افزایش کارایی و پاسخگویی سیستم ارتباطی با کاهش نگرانی‌های تاخیر از طریق به حداقل رساندن فاصله بین \lr{UE} و \lr{UAV-BS} ایفا می‌کند.	
% Please add the following required packages to your document preamble:
% \usepackage{multirow}
% \usepackage[table,xcdraw]{xcolor}
% Beamer presentation requires \usepackage{colortbl} instead of \usepackage[table,xcdraw]{xcolor}
\begin{table}%[]
\centering
\caption{ متوسط ​​تاخیر هر پهپاد در هر سناریو در طول زمان}
\begin{tabular}{|c|c|c|c|}
\hline
\multicolumn{1}{|l|}{کارآزمایی} & \multicolumn{1}{l|}{تعداد کاربران} & \multicolumn{1}{l|}{تعداد پهپادها} & \multicolumn{1}{l|}{میانگین تاخیر (ms)} \\ \hline
1                                & \multirow{4}{*}{50}                & 2                                      & 0.60                                      \\ \cline{1-1} \cline{3-4} 
2                                &                                    & 3                                      & 0.40                                      \\ \cline{1-1} \cline{3-4} 
3                                &                                    & 5                                      & 0.24                                      \\ \cline{1-1} \cline{3-4} 
4                                &                                    & 10                                     & 0.20                                      \\ \hline
5                                & \multirow{4}{*}{75}                & 2                                      & 0.90                                      \\ \cline{1-1} \cline{3-4} 
6                                &                                    & 3                                      & 0.72                                      \\ \cline{1-1} \cline{3-4} 
7                                &                                    & 5                                      & 0.61                                      \\ \cline{1-1} \cline{3-4} 
8                                &                                    & 10                                     & 0.37                                      \\ \hline
9                                & \multirow{4}{*}{150}               & 2                                      & 1.82                                      \\ \cline{1-1} \cline{3-4} 
10                               &                                    & 3                                      & 1.22                                      \\ \cline{1-1} \cline{3-4} 
11                               &                                    & 5                                      & 1.66                                      \\ \cline{1-1} \cline{3-4} 
12                               &                                    & 10                                     & 0.67                                      \\ \hline
\end{tabular}
\label{tab:avglatency}
\end{table}\textbf{}
	شایان ذکر است که مقادیر پایین مشاهده شده در هر دو پارامتر، تاخیر و از دست دادن بسته، از ارتباط مستقیم تک گام بین \lr{UE}ها و \lr{UAV-BS}ها نشات می گیرد. برعکس، تضعیف کانال رادیویی در نظر گرفته نشده است. به عبارت دیگر، سناریوها به طور ایده آل ارتباط را با یک پرش واحد بین \lr{UE} و \lr{UAV-BS} آزمایش کرده‌اند. با این حال، بر اساس مدل ارائه شده، عوامل موثر بر این پارامترها شامل فاصله بین \lr{UE} و \lr{UAV-BS}، تعداد اتصالات همزمان به \lr{UAV-BS} و الزامات داده ای \lr{UE}ها می‌باشد.
	\item \textit{انرژی مصرفی}
	مصرف انرژی پهپادها به طور قابل توجهی تحت تأثیر حرکت آنها است. با این حال، توجه به این نکته مهم است که انتقال و دریافت داده در \lr{UAV-BS} انرژی مربوط به ارتباطات را نیز مصرف می کند، عاملی که اغلب در مقایسه با مصرف برق مربوط به حرکت نادیده گرفته می شود. قدرت حرکت جنبه های مختلف تحرک مانند حرکت افقی و عمودی، فرود، برخاستن و شناور شدن را در بر می گیرد. به حداقل رساندن حرکت \lr{UAV-BS} و در عین حال بهینه سازی استراتژیک قرارگیری آنها برای به حداکثر رساندن پوشش UE به ما امکان می دهد مصرف برق مربوط به حرکت را کاهش دهیم.
	\\
	شکل‌های  \ref{fig:movepower} و \ref{fig:commpowerfact} به ترتیب مقایسه قدرت حرکتی و مصرف توان ارتباطی پهپادهای \lr{UAV-BS} را در سناریوها و آزمایش‌های مختلف نشان می‌دهند.
	با توجه به تصویر \ref{fig:movepower}، مصرف انرژی حرکتی مهم‌ترین بخش توان است مصرف در مقایسه با توان ارتباطی همانطور که توسط مدل پیشنهادی تعیین می شود، قدرت حرکت تابعی از حرکت \lr{UAV-BS}، مشخصات روتور، موقعیت‌ \lr{UE}ها، موقعیت \lr{UAV-BS}ها، سرعت باد، جهت باد و سرعت پهپاد است. با توجه به تصویر \ref{fig:movepower} که در آن خطوط نشان دهنده قدرت حرکت متوسط ​​زمانی که حرکت \lr{UAV-BS} محدود است، کوچک می‌مانند.
	همانطور که می توان در تصویر \ref{fig:movepower} مشاهده کرد، مواردی وجود دارد که انرژی تحرک در طی مراحل زمانی خاص افزایش می یابد. این اتفاق زمانی می افتد که \lr{UAV-BS} برای دستیابی به مکان‌های بهینه برای \lr{UE} جامع نیاز به تغییر موقعیت خود را به طور گسترده‌تری دارند پوشش.
	در تصویر \ref{fig:movepower}، به‌ویژه در نمودار (a)، مقادیر پیک در مرحله زمانی ۲ برای آزمایش‌ها با ۵۰ و ۱۵۰ \lr{UE} وجود دارد که می‌تواند به \lr{UAV-BS}‌هایی نسبت داده شود که مسافتی نزدیک به ۱۰۰ درصد را پوشش می‌دهند. واحدهای بین مراحل زمانی ۱ و ۴. این فاصله بسیار مهم است زیرا با داده های مورد نیاز \lr{UE}ها هماهنگ است و تحت تأثیر عوامل مختلفی مانند الگوهای تحرک\lr{UAV-BS}، حرک پویا، تغییرات موقعیت، ظرفیت \lr{UAV-BS}، و محدوده پوشش قرار می‌گیرد. علاوه بر این، شبیه‌سازی شرایط باد را به‌طور تصادفی انتخاب می‌کند، این بازه مقداری از سرعت ۱ تا ۲۱ کیلومتر در ساعت و جهت‌گیری ۰ تا ۳۶۰ درجه است، در مرحله زمانی ۲، مقادیر ویژه سرعت باد ($V_{t}^{Wind}$) با سرعت 17.23 کیلومتر در ساعت و جهت باد ($\theta_{t}^{Wind}$) در 251.06 درجه ثبت می‌شوند. ایم مقدار نشان دهنده وزش باد شدید مخالف جهت حرکت \lr{UAV-BS} است. در نتیجه، این مخالفت منجر به افزایش مصرف انرژی توسط \lr{UAV-BS} به دلیل اثر مقاومت باد  بر روی پهپاد می‌شود. تنظیمات تصادفی مشابه‌ای ممکن است در نمودارهای دیگر مشاهده شود، که می‌تواند تحت تأثیر این عوامل محیطی و تحرک‌های متنوع باشد.
\\	
	\begin{figure}%[H]
		\centering
		\includegraphics[width=0.83\textwidth]{Pictures/movementpower.pdf}
		\caption{انرژی مصرفی حرکتی پهپاد در آزمایش‌های و تعداد کاربران مختلف}
		\label{fig:movepower}
	\end{figure}
\\
 تصویر \ref{fig:commpowerfact} توان ارتباطی مصرف شده توسط هر \lr{UAV-BS} را در طول زمان در آزمایش‌های مختلف با تعداد متغیر \lr{UE} نشان می‌دهد. یکی از عوامل مهمی که بر این قدرت ارتباطی تأثیر می‌گذارد، تعداد \lr{UE}هایی است که توسط هر \lr{UAV-BS} پوشش داده می‌شود. به طور کلی، هنگامی که یک \lr{UAV-BS} تعداد \lr{UE}های بیشتری را پوشش می‌دهد، تمایل به مصرف انرژی بیشتری برای ارتباط دارد، زیرا نیاز به مدیریت انتقال و دریافت داده برای یک مجموعه بزرگتر \lr{UE} را دارد. برعکس، زمانی که \lr{UE}های کمتری متصل می‌شوند، توان ارتباطی کمتر می‌شود. این رابطه در تمام نمودارهای نشان داده شده صادق است.
\begin{figure}%[H]
	\centering
	\includegraphics[width=0.83\textwidth]{Pictures/commpowerfactorial.pdf}
	\caption{انرژی مصرفی ارتباطی پهپاد در آزمایش و تعداد کاربر مختلف}
	\label{fig:commpowerfact}
\end{figure}
علاوه بر این، جدول \ref{tab:connectdiseduser} بینش‌های ارزشمندی را در مورد میانگین درصد \lr{UE}های متصل و غیر متصل در طول مدت هر تعامل \lr{UAV-BS} را ارائه می‌کند، که به طور مستقیم  بر روند مشاهده ‌شده در انرژی ارتباطی نمایش‌داده ‌شده در نمودار انرژی مرتبط با ارتباطات تاثیر می‌گذارد. تجزیه و تحلیل جدول، یک روند واضح را نشان می دهد: با افزایش تعداد \lr{UAV-BS}، درصد اتصال مشاهده شده در بین \lr{UE}ها افزایش می یابد. علاوه بر این، قابل توجه است که در هر آزمایش، نرخ پوشش ۱۰۰ درصد با استفاده از ۱۰ \lr{UAV-BS} به دست آمده است، که بر همبستگی بین کمیت \lr{UAV-BS} و اتصال افزایش یافته مشاهده شده در سراسر تعامل با \lr{UE}ها تاکید می‌کند.
	\item \textit{نرخ گذردهی}
	تصویر \ref{fig:Throughput} میزان گذردهی (اندازه‌گیری شده در میلی‌ثانیه) \lr{UAV-BS} را در طول زمان نشان می‌دهد، و بینش‌هایی را در مورد نرخ انتقال داده به دست آمده توسط این \lr{UAV-BS} ارا رائه می‌دهد. عوامل متعددی در تغییر گذردهی در این نمودار نقش دارند. یکی از عوامل مهم تعداد \lr{UE}های متصل به هر \lr{UAV-BS} در یک زمان معین است. هنگامی که یک \lr{UAV-BS} به تعداد بیشتری از \lr{UE}ها به طور همزمان سرویس دهد، معمولاً منجر به توان عملیاتی بالاتری می‌شود، زیرا داده‌های بیشتری ارسال و دریافت خواهد شد. برعکس، زمانی که \lr{UE}های کمتری متصل شوند، به دلیل کاهش تقاضای تبادل داده، توان عملیاتی کمتر خواهد شد. علاوه بر این، کیفیت پیوند بی‌سیم، تراکم شبکه و کارایی پروتکل‌های انتقال داده نیز می‌تواند بر توان عملیاتی تأثیر بگذارد. گذردهی بالا زمانی به دست می‌آید که این عوامل به طور مطلوب در یک راستا قرار گیرند و امکان انتقال کارآمد و سریع داده‌ها را فراهم کنند، در حالی که گذردهی پایین اغلب نتیجه شرایط نامطلوب یا تراکم شبکه است که مانع از جریان داده می‌شود. بازده نیز تحت تأثیر از دست دادن بسته، ظرفیت \lr{UAV-BS} که به طور تصادفی در محدوده ۴۰۰-۵۰۰ متر تولید می‌شود و نرخ داده \lr{UE}ها می‌باشد. نرخ داده‌ای که به‌طور تصادفی تولید می‌شود، تأثیر مستقیم و قابل‌توجهی بر نرخ گذردهی دارد و به عدم تعادل مشاهده ‌شده در نمودارهای گذردهی کمک می‌کند.	
\end{itemize}
% Please add the following required packages to your document preamble:
% \usepackage{multirow}
% \usepackage[table,xcdraw]{xcolor}
% Beamer presentation requires \usepackage{colortbl} instead of \usepackage[table,xcdraw]{xcolor}
\begin{table}%[]
\centering
\caption{میانگین درصدی متصل یا غیر متصل بودن کاربران به ازای هر کار‌آزمایی در زمان}
\begin{tabular}{|c|c|c|c|c|}
\hline
کار‌آزمایی & تعداد کاربران        & تعداد پهپاد‌ها & میانگین متصل & میانگین‌ غیر متصل‌ \\ \hline
1          & \multirow{4}{*}{50}  & 2                 & 51                   & 49                      \\ \cline{1-1} \cline{3-5} 
2          &                      & 3                 & 66                   & 34                      \\ \cline{1-1} \cline{3-5} 
3          &                      & 5                 & 91.80                & 8.2                     \\ \cline{1-1} \cline{3-5} 
4          &                      & 10                & 100                  & 0                       \\ \hline
5          & \multirow{4}{*}{75}  & 2                 & 51.99                & 48.01                   \\ \cline{1-1} \cline{3-5} 
6          &                      & 3                 & 60.94                & 39.06                   \\ \cline{1-1} \cline{3-5} 
7          &                      & 5                 & 78.93                & 21.07                   \\ \cline{1-1} \cline{3-5} 
8          &                      & 10                & 100                  & 0                       \\ \hline
9          & \multirow{4}{*}{150} & 2                 & 39.07                & 60.93                   \\ \cline{1-1} \cline{3-5} 
10         &                      & 3                 & 62                   & 38                      \\ \cline{1-1} \cline{3-5} 
11         &                      & 5                 & 88.81                & 11.19                   \\ \cline{1-1} \cline{3-5} 
12         &                      & 10                 & 100                  & 0                       \\ \hline
\end{tabular}
\label{tab:connectdiseduser}
\end{table}
\begin{figure}%[!htbp]
	\centering
	\includegraphics[width=0.9\textwidth]{Pictures/throughput.pdf}
	\caption{میزان گذردهی پهپاد‌ها در طول آزمایش‌ها و تعداد کاربران مختلف}
	\label{fig:Throughput}
\end{figure}
\subsection{تجزیه تحلیل}\label{subsection:referencecompare}
برای اعتبارسنجی کارایی مدل بهینه‌سازی پیشنهادی در این پژوهش، الگوریتم‌های موجود در ادبیات، از جمله \cite{Pasandideh2023} و \cite{8892933} برای مقایسه و تحلیل انتخاب شده‌اند.
در \cite{Pasandideh2023}، مشاهده شد که \lr{UE}های خاص نرخ از دست دادن بسته های قابل توجهی را تجربه می‌کنند. این مشکل عمدتا از موقعیت دور از دسترس این \lr{UE}ها از منطقه پوشش‌دهی پهپادها مربوطه ناشی می‌شود و باعث می‌شود داده‌ها از این \lr{UE}ها نتوانند به \lr{UAV-BS}ها دسترسی پیدا کنند. با این حال، آزمایش‌های جامع نشان داده شده در شکل \ref{fig:PacketLossPlot}، نشان‌دهنده نرخ اتلاف بسته‌های پهپاد یک بهبود مشهود آشکار است. نرخ کلی اتلاف بسته به دست آمده از طریق روش پیشنهادی به طور قابل توجهی کمتر از میانگین نرخ اتلاف بسته است که در \cite{Pasandideh2023} بیان شده است. این پیشرفت را می توان به افزایش دقت مدل فعلی در تعیین موقعیت‌ها و مدیریت تحرک بهینه پهپادها نسبت داد. در نتیجه، \lr{UE}ها به طور موثرتری به پهپاد های نزدیک تر تخصیص داده می‌شوند که منجر به کاهش قابل توجه نرخ اتلاف بسته می‌شود.
//
در \cite{Pasandideh2023} متوسط ​​تأخیر ثبت شده برای هر مجموعه \lr{UE} و پهپاد مربوط به آنها تقریباً 15$ms$ بود. این مقدار تأخیر در مقایسه با نرخ تأخیر به‌دست‌آمده در روش پیشنهادی، همانطور که در جدول \ref{tab:avglatency} ارائه شده است، به‌طور قابل‌توجهی بالاتر است، جایی که میانگین تأخیر تقریباً 1$ms$ است. تأخیر به طور پیچیده با فاصله بین \lr{UE} و \lr{UAV-BS} گره خورده است، که نشان می دهد فاصله فیزیکی بیشتر بین این موجودیت‌ها منجر به تأخیر بالاتر می‌شود. روش پیشنهادی بر به حداقل رساندن فاصله بین \lr{UE}‌ها و \lr{UAV-BS‌‌}ها نیز تمرکز دارد، در نتیجه تأخیر را در مقایسه با تأخیر گزارش شده در \cite{Pasandideh2023} کاهش می‌دهد. علاوه بر این، یک ارتباط مستقیم بین از دست دادن بسته و تأخیر وجود دارد. روش پیشنهادی به طور قابل توجهی از دست دادن بسته را در مقایسه با کار قبلی کاهش داده است. هنگامی که بسته‌ها در طول انتقال از طریق شبکه گم می‌شوند، زمان بیشتری برای شناسایی و بازیابی این بسته‌های از دست رفته توسط فرستنده لازم است. بنابراین، کاهش تلفات بسته در روش پیشنهادی به طور مستقیم به تاخیر کمتر مشاهده شده کمک می‌کند.
//
مقایسه مصرف انرژی نشان ‌داده ‌شده در شکل \ref{fig:powercomp} یک روند صعودی را در خط قرمز نشان می‌دهد، که نشان‌دهنده مصرف انرژی بر اساس روش‌ مشخص شده در \cite{8892933} است. این روند حاکی از همبستگی مستقیم بین افزایش تعداد \lr{UE}ها و افزایش مصرف انرژی است. با این حال، مدل پیشنهادی، که با خط آبی نشان داده می‌شود، تشخیص می‌دهد که مصرف انرژی صرفاً به تعداد کاربرانی که خدمات ارائه می‌دهند وابسته نیست. در مدل پیشنهادی، عوامل اضافی مختلف به طور قابل توجهی در مصرف انرژی نقش دارند. عناصری مانند سرعت و جهت باد، سرعت \lr{UE}ها، نرخ داده مورد نیاز آن‌ها، ظرفیت پهپاد و سایر عوامل کمک کننده اجزای جدایی ناپذیر در مدل مصرف انرژی هستند. این عوامل مجموعاً پویایی مصرف انرژی را شکل می‌دهند و بر آن تأثیر می‌گذارند. این دیدگاه گسترده‌تر منجر به الگوی مصرف انرژی پیچیده‌تر می‌شود. به عنوان مثال، تشدید سرعت باد و جهت مخالف می‌تواند به طور قابل توجهی بر استفاده از توان تأثیر بگذارد، حتی در سناریوهایی با تعداد ثابتی از \lr{UE}.
با ادغام این عوامل متنوع، مدل پیشنهادی درک جامعی از دینامیک مصرف انرژی ارائه می‌کند، که فراتر از تمرکز منحصر به فرد بر روی کمیت \lr{UE‌}هایی است که برای به تصویر کشیدن تأثیر متقابل پیچیده متغیرهای متعدد مؤثر بر مصرف انرژی عمل می‌کنند.
\begin{figure}%[!htbp]
	\centering
	\includegraphics[width=0.6\textwidth]{Pictures/moveComparison_5UAVs.pdf}
	\caption{مقایسه توان مصرفی بین روش پیشنهادی و \cite{8892933}}
	\label{fig:powercomp}
\end{figure}
\\