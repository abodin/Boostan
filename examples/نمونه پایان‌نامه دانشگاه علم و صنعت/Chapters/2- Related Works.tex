\chapter{کارهای مرتبط}\label{Chapter2-2}
در این فصل، پژوهش‌های انجام شده، در زمینه پایش و نظارت با استفاده از پهپادها همچنین تحرک و جانمایی پهپاد مورد بررسی قرار می‌گیرد. از طرفی لازم به ذکر است جانمایی اولیه پهپادها و تحرک آن‌ها به‌منظور پوشش بهبودیافته کاربران متحرک یا گره‌های ثابت می‌تواند بیانگر اهمیت این موضوع باشد که سعی شده براین‌اساس پارامترهایی را برای سنجش در نظر بگیریم، از جمله انرژی، تأخیر ارسال بسته و انسداد بسته‌ها مورد تحقیق و شبیه‌سازی قرار گرفته شده است.
\section{مقدمه}
در سال‌های اخیر، ایستگاه‌های پایه مبتنی بر پهپاد به دلیل استقرار پذیری و قابلیت مانور در شبکه‌های حسگر بی‌سیم و شبکه‌های سلولی اهمیت یافته‌اند. بااین‌حال، مسئله مکان‌یابی و تحرک‌پذیری همچنان یک موضوع تحقیقاتی باز باقی‌مانده است. چالش این است که مکان‌های بهینه برای استقرار و تحرک بهینه \lr{UAV-BS}ها را تعیین کنیم تا به اهداف خاصی مانند حداکثرکردن پوشش منطقه و تعداد کاربران نهایی خدمات‌دهی شده، بهبود اتصال بی‌سیم بین \lr{UAV-BS}ها و گره‌های زمینی، کاهش زمان عملیات بین نقاط، افزایش بهره‌وری شبکه، و یافتن راه‌حل‌های کارآمد از نظر انرژی برای حل مسئله مکان‌یابی و تحرک \lr{UAV-BS} دست یابیم.
یک خلاصه سطح بالا از رویکردهای مختلف بهینه‌سازی انرژی در سامانه‌های ارتباطی مبتنی بر پهپاد در \autoref{fig:solutions} ارایه شده است که شامل روش‌های سنتی، یادگیری ماشین، الگوریتم‌های بهینه‌سازی، و راهبُردهای ترکیبی است.
بر اساس \autoref{fig:solutions} رویکردهای سنتی مانند آزمون‌وخطا از راهبُردهای مبتنی بر روش‌های اکتشافی یا جستجوهای شبکه‌ای برای مکان‌یابی \lr{UAV-BS}ها به‌گونه‌ای استفاده می‌شود که به طور شهودی مناطق هدف را پوشش داده و مصرف انرژی را کاهش دهند. روش‌های تحلیلی مانند مدل‌های بهینه‌سازی و نظریه بازی‌ها از فرمول‌بندی‌های ریاضی برای تعیین دقیق مکان‌های بهینه \lr{UAV-BS} با درنظرگرفتن عواملی مانند پوشش، تداخل، و توازن انرژی استفاده می‌کنند. این روش‌ها تلاش می‌کنند تا مصرف انرژی را به حداقل رسانده و درعین‌حال پوشش ارتباطی مؤثر را تضمین کنند.
روش‌های مبتنی بر یادگیری ماشین، شامل یادگیری نظارت شده (استفاده از مدل‌های رگرسیون و طبقه‌بندی)، یادگیری بدون نظارت (استفاده از خوشه‌بندی و تشخیص ناهنجاری)، و یادگیری عمیق (استفاده از شبکه‌های عصبی بازگشتی (\gls{RNN}) و شبکه‌های عصبی پیچشی (\gls{CNN}))، به طور قابل‌توجهی کمک می‌کنند. این رویکردها از الگوهای داده‌ها برای پیش‌بینی مصرف انرژی، شناسایی مکان‌های کارآمد، یا کشف الگوهای پیچیده فضایی و زمانی برای استقرار بهینه \lr{UAV-BS} استفاده می‌کنند و بدین ترتیب کارایی انرژی را افزایش می‌دهند. یادگیری تقویتی، به‌ویژه از طریق فرایندهای تصمیم‌گیری مارکوف (\gls{MDP})، به \lr{UAV-BS}ها اجازه می‌دهد تا سیاست‌های مکان‌یابی بهینه را از طریق تعامل با محیط یاد بگیرند و هدف آنها کاهش مصرف انرژی در عین حفظ پوشش مؤثر است. الگوریتم‌های بهینه‌سازی مانند الگوریتم‌های حریصانه و الگوریتم‌های فراابتکاری (مانند الگوریتم ژنتیک (\gls{GA})، بهینه‌سازی ازدحام ذرات (\gls{PSO}) و \lr{JAYA}) فضای جستجو را برای یافتن راه‌حل‌های نزدیک به بهینه که کارایی انرژی را در مکان‌یابی \lr{UAV-BS} در نظر می‌گیرند، کاوش می‌کنند. رویکردهای ترکیبی، روش‌های سنتی را با یادگیری ماشین ترکیب می‌کنند و ترکیبی از دقت ریاضی و تشخیص الگوها را برای مکان‌یابی بهینه و تحرک‌پذیری بهینه پهپادها ارائه می‌دهند.
در این پژوهش، روش‌های فراابتکاری برای بهینه‌سازی انرژی در سامانه‌های ارتباطی مبتنی بر پهپادها ترجیح داده می‌شوند؛ زیرا توانایی مؤثری در پیمایش فضاهای جستجوی پیچیده و بزرگ را دارند. درحالی‌که روش‌های سنتی ممکن است بینش‌هایی در مورد سناریوهای خاص ارایه دهند، آنها اغلب در مدیریت پیچیدگی‌ها و ماهیت پویای محیط‌های واقعی مشکل دارند. از سوی دیگر، رویکردهای یادگیری ماشین ممکن است به‌شدت به دسترسی به داده‌ها متکی باشند و ممکن است همیشه به‌خوبی به شرایط متنوع و در حال تغییر تعمیم نیابند.

%\begin{landscape}
\begin{figure}
\includegraphics[width=0.93\linewidth]{Solution}
\caption[\lofimage{Solution}راه‌کار‌هایی برای بهینه سازی انرژی در پهپاد‌ها]{رویکرد بهینه‌سازی انرژی در پهپاد}
\label{fig:solutions}
\end{figure}
%\end{landscape}
\section{راه‌کار‌های پایش با پهپاد}
در مقاله \cite{Ananthi2021} که توسط \lr{Ananthi} و همکاران در مورد جمع‌آوری علائم حیاتی با استفاده از پهپاد روشی را ارایه کرده‌اند. به دلیل نیاز این نوع اطلاعات تولیدی به امنیت، انتقال سریع و مصرف کم، از این دید پهپاد موردتوجه قرار گرفته شده است، پنج نوع حسگر در این طرح در نظر گرفته شده است که عبارت‌اند از: حسگر ضربان قلب، دمای بدن، حرکتی، فشارخون و اکسیژن سنج. راه‌حل استفاده از پهپاد به موارد اضطراری کمک می‌کند تا اطلاعات را زودتر به دست پزشک برسانند.
در مقاله \cite{Tan2021}، \lr{Shihan} و همکاران در مورد اهمیت وجود پهپاد در شبکه‌های شهر هوشمند آینده و یکپارچه‌سازی شبکه هوایی با زمینی باهدف حل مشکلات واپایش پهپاد در شهر هوشمند موردتوجه قرار گرفته است، دو نوع کنترل‌کننده معرفی شده، کنترل‌کننده \gls{SDN} سطح بالا وظیفه برنامه‌ریزی پیکربندی پهپاد را بر عهده دارد از طرفی کنترل‌کننده \lr{SDN} سطح پایین وظیفه هماهنگی منطقه‌ای پهپاد را عهده‌دار است. با شبیه‌سازی وضعیت شبکه در بازه‌های زمانی راهبُرد بهینه زمان‌بندی پهپاد با استفاده از الگوریتم تکرار راهبُرد به دست می‌آید. در نهایت، یک مثال برای تأیید اینکه معماری یادگیری تقریباً بی‌درنگ می‌تواند، به‌دقت نیاز پهپاد را پیش‌بینی کند و توان عملیاتی سامانه شبکه را در مقایسه با رویکرد سنتی افزایش دهد، ارائه شده است.
در پژوهش \cite{Ma2021}، \lr{Ruofei} و همکاران در مورد پایش اقیانوس با کمک پهپاد از راه دور را مورد بررسی قرار دادند، به‌طوری‌که داده‌ها از طریق حسگرهایی که در زیر آب قرار دارند به گره‌هایی در روی آب فرستاده شده و پهپاد وظیفه جمع‌آوری و ارسال به ایستگاه‌های زمینی پایه را دارند، مسئله بهینه‌سازی در این مقاله بر اساس تخصیص منابع دسترسی حسگر به گره سطحی و گره سطح به پهپاد است.
در این \cite{Caruso2021}، پژوهش \lr{Caruso} و همکاران، از پهپادها به‌عنوان یک شیوه برای جمع‌آوری داده در کشاورزی دقیق استفاده کرده‌اند. این پهپادها داده‌ها را از شبکه‌های حسگر بی‌سیم روی زمین جمع‌آوری می‌کنند، این امر در سناریوهایی که فاقد هرگونه زیرساخت ارتباطی ثابت هستند یا زیرساخت‌های موجود با الزامات سناریو مطابقت ندارد، از اهمیت بالایی برخوردار است. حسگرهای روی زمین می‌توانند داده‌های سنجش را ذخیره کنند، و در سناریوهایی که نیازی به مشاهده و تحلیل آنی داده‌ها ندارند، مانند کشاورزی هوشمند، می‌توان از پهپاد یک یا دو بار در روز برای جمع‌آوری و گزارش داده‌ها استفاده کرد. به‌طورکلی در این پژوهش به‌صورت تحلیلی، با استفاده از رادیو \gls{LoRa}، برای جمع‌آوری داده مطالعه شده است.
در سناریو مطرح شده در این پژوهش که از پهپاد به‌عنوان یک فناوری برای جمع‌آوری داده استفاده می‌شود، از طرفی حسگرها به طور منظم بر روی زمین قرار گرفته‌اند، پهپادها طی زمان‌های مشخص شده‌ای به جمع‌آوری داده مورد سنجش این حسگرهای می‌پردازند، در این پژوهش با استفاده از یک مدل تحلیل به حل مشکلات پوشش و قرارگرفتن در میدان دید انجام داده‌اند، این به نوع فناوری رادیویی مورداستفاده نیز بستگی دارد. همچنین خروجی این مدل ارایه شده، فاصله بهینه بین گره‌ها، زمانی که یک حسگر در محدوده پهپاد برای برقراری ارتباط موفق قرار می‌گیرد، و سرعتی که باید پهپاد برای تکمیل برنامه پرواز خود حفظ کند، است.
در مقاله \cite{Behjati2021}، بهجتی و همکاران، یک سامانه جمع‌آوری داده هوایی توسعه داده‌اند که مبتنی بر ادغام \lr{LoRa}، \lr{IoT} و پهپادها است. سامانه توسعه‌یافته از سه بخش اصلی تشکیل شده است: 1- گره‌های حسگر که در سراسر مزرعه توزیع شده‌اند که می‌تواند هر نوع حسگری باشند، ۲- یک شبکه ارتباطی مبتنی بر \lr{LoRaWAN} که داده‌ها را از حسگرها جمع‌آوری و آن را به فضای ذخیره‌سازی ابری منتقل می‌کند، ۳- یک تکنیک بهینه‌سازی برنامه‌ریزی مسیر که مسیر پهپاد را برای پوشش مؤثر مزرعه و جمع‌آوری داده‌ها از همه حسگرهای مستقر شده بهینه می‌کند. بر اساس نتایج، \lr{LoRaWAN} یک سامانه قابل‌اعتماد است که قادر به ایجاد یک \gls{WSN} در یک مزرعه در مقیاس بزرگ است. علاوه بر این، با ادغام \lr{LoRAWAN} با یک پهپاد، می‌توان به پوشش بیشتری دست‌یافت و عملکرد قوی با حداکثر سرعت پهپاد ۹۵ کیلومتر در ساعت انجام داد. همچنین نشان داده شد که با افزایش ارتفاع پرواز (به‌عنوان‌مثال، ۱۰۰ متر تا ۱۵۰ متر)، می‌توان پوشش \lr{LoRa} بهتری را فراهم کرد. در نهایت، استفاده از پهپاد برای جمع‌آوری داده‌های هوایی نه‌تنها می‌تواند دردسترس‌بودن داده‌ها را بهبود بخشد، بلکه می‌تواند فرایندهای نظارت مزرعه را تسریع کند و بهره‌وری مزارع را افزایش دهد، به‌عنوان‌مثال، با افزایش تولید و بهینه‌سازی استفاده از منابع. در این راستا، نظام توسعه‌یافته یک توسعه موفق در زمینه سامانه‌های جمع‌آوری داده‌های هوایی مبتنی بر پهپاد است که راه‌حلی برای مقابله با مشکلات حیاتی در مدیریت مزرعه، نظارت بر دام‌ها و جمع‌آوری داده‌ها از حسگرهای مختلف اینترنت اشیا ارائه می‌دهد.
پژوهش انجام شده در \cite{Tehseen2021}، \lr{Tehseen} و همکاران، در مورد تشخیص آتش مطالعه داشته‌اند، متأسفانه تقریباً هر ساله آتش‌سوزی به دلیل تشخیص دیرهنگام و ناکارآمد آتش به میلیون‌ها هکتار از اراضی جنگلی خسارت وارد می‌کند. بااین‌حال، شناسایی آتش‌سوزی جنگل در سطح اولیه قبل از گسترش آن به مناطق وسیع و تخریب منابع طبیعی مهم است. در این پژوهش، یک مدل از اینترنت اشیا و نظام تشخیص و مقابله با آتش‌سوزی جنگل مبتنی بر پهپاد ارائه شده است. سامانه پیشنهادی شامل تعمیر و نگهداری شبکه است. استقرار حسگر بر روی درختان، زمین و حیوانات به شکل زیرشبکه برای انتقال داده‌های به‌دست‌آمده برای انتقال به اتاق واپایش است. همه زیرشبکه‌ها از طریق گره‌های دروازه به اتاق واپایش متصل می‌شوند. از زنگ هشدار برای هشداردادن به انسان‌ها و حیوانات برای نجات استفاده می‌شود که به محافظت اولیه آنها در برابر آتش کمک می‌کند. حسگرهای تعبیه شده اطلاعات را جمع‌آوری کرده و به دروازه‌ها منتقل می‌کنند. پهپادها برای عملیات بی‌درنگ در مناطق متأثر از آتش‌سوزی و انجام اقداماتی برای واپایش آتش استفاده می‌شوند، زیرا آنها نقش حیاتی در بلایا دارند. نظریه گراف برای ساخت یک مدل کارآمد و نشان‌دادن اتصال شبکه استفاده می‌شود. برای شناسایی خرابی‌ها و توسعه روش‌های بازیابی، الگوریتم از طریق مدل مبتنی بر نمودار طراحی شده است. این مدل توسط زبان \lr{Vienna} توسعه‌یافته است.
در مقاله \cite{Samarakkody2021}، \lr{Samarakkody} و همکاران در مورداستفاده پهپاد در شبکه‌های حس‌گر بی‌سیم پژوهش داشته‌اند، گسترش دستگاه‌های اینترنت اشیا کم‌مصرف در شبکه‌های حسگر بی‌سیم به دلیل کاهش مصرف انرژی، استفاده از اینترنت اشیا را گسترده‌تر کرده است. در طول مقاله، به دسترسی گسترده و فرایند جمع‌آوری داده یک \gls{WSN} از راه دور مبتنی بر اینترنت اشیا با یک پهپاد پرداخته شده است. تأثیر پارامترهای مختلف مربوط به عملکرد جمع‌آوری داده‌ها و نظارت بر سامانه \lr{UAV-WSN} ارزیابی شده و بهبودهایی در این زمینه پیشنهاد شده. شبیه‌سازی‌ها نشان می‌دهند که فاصله بین حسگرها، پارامتر غالب در عملکرد جمع‌آوری داده‌های \lr{WSN} است. این سامانه پیشنهادی منجر به افزایش ۲۳ درصدی عملکرد جمع‌آوری داده‌ها شده است. یک نمونه اولیه برای سامانه پیشنهاد شده با استفاده از ماژول \lr{Raspberry Pi} تعبیه شده در پهپاد \lr{Phantom 3 Standard} و سکوهای \lr{CC2650 SensorTag} با استفاده از ارتباط \lr{BLE} اجرا و آزمایش شده است.
در مرجع \cite{Singh2022}، \lr{Singh} و \lr{Sharma}، بستری را برای مدیریت اطلاعات محصولات کشاورزی جمع‌آوری‌شده از طریق پهپاد ارائه می‌دهد. چارچوب \lr{Django} برای طراحی سامانه خدمات اطلاعاتی برای جمع‌آوری اطلاعات محصول و داده موقعیت استفاده می‌شود، یک ارزیابی عملکرد گسترده برای سنجش و پایش محصولاتی که از معماری پیشنهادی بهره می‌برند انجام شده است. معماری پیشنهادی بازده پوششی ۹۶.۳ درصد را ارائه می‌کند و پتانسیل بالایی در کاربردهای کشاورزی مانند نظارت بر سلامت محصول، سم‌پاشی، کودها و آفت‌کش‌ها دارد.
در این پژوهش با استفاده از راهکار \lr{WSN} و اینترنت اشیا مبتنی بر پهپاد را برای پشتیبانی از راه‌حل مقرون‌به‌صرفه نظارت بر محصول، نقشه‌برداری و سم‌پاشی در کشاورزی دقیق ارائه می‌کند. دستگاه‌های هوشمند اینترنت اشیا برای جمع‌آوری داده‌های زمین کشاورزی و پهپادها برای نظارت هوایی و وظایف پایش استفاده می‌شوند. برای ارزیابی عملکرد معماری پیشنهادی، ۳۰ کاربرد مختلف پهپاد اختصاص‌داده‌شده به نظارت بر محصول و حوزه کشاورزی مورد تجزیه‌وتحلیل قرار گرفته شده است. سهم کلیدی این مطالعه طراحی مسیرهای بهینه شده برای پایش محصول مبتنی بر پهپاد و تکنیک پردازش داده است که امکان استفاده کارآمد از منابع شبکه حسگر مستقر را فراهم می‌کند. به طور خاص، مدل پیشنهادی تعادلی بین طول مسیر و زمان صرف شده توسط گره همسایه سرخوشه ارائه می‌کند. نتایج عملی و قوی برای داده‌های جمع‌آوری‌شده با استفاده از گره‌های \lr{WSN} و سامانه یکپارچه‌سازی \lr{WSN-UAV} این پیشنهاد به دست می‌آید. روش پیشنهادی برای همه مکان‌های جغرافیایی به‌خوبی عمل می‌کند و نتایج پیش‌بینی بهتری را برای پایش کشاورزی فراهم می‌کند.
در پژوهش \cite{Tan2021a}، \lr{Eng Choon Tan} و همکاران در مورد سنجش عمق برف بر روی سطح یخ‌زده آب پژوهش داشته‌اند، به این صورت که با درنظرگرفتن یک پهپاد به همراه تجهیزات نصب شده بر روی آن عمق برف را مورد سنجش قرار می‌دهد این کار در تابستان سال ۲۰۱۷ تا ۲۰۱۸ در قطب جنوب مورد ارزیابی قرار گرفته شده است.
در مقاله \cite{Kannadaguli2020}، \lr{Kannadaguli} در مورد تشخیص انسان با کمک پهپاد و الگوریتم تشخیص \lr{Yolo} تحقیق شده است این مورد شامل تشخیص انسان از روی حرارت بدن است، به این صورت که با تجهیزکردن پهپاد به دوربین حرارتی و استفاده از الگوریتم \lr{Yolo} بتوانیم به این مورد دست‌یافت، از طرفی طبق گفته نویسنده این روش امیدوارکننده‌ای برای تشخیص خودکار انسان نیز است و می‌تواند مورداستفاده قرار بگیرد.
در مقاله \cite{Yilmaz2020}، \lr{Yilmaz} و \lr{Denizer}، یکی از اهداف مهم شهرهای هوشمند، مشکل ترافیک، را موردبحث و بررسی قرار داده‌اند و راه‌حلی با استفاده از پهپاد ارائه شده است که هدف آن شناسایی گره‌های ترافیکی در شهر با ثبت تصاویر مستمر از مکان‌های مختلف است. برای این راه‌حل نیاز به برنامه‌ریزی مسیر پهپادهایی است که بر فراز شهر به پرواز در می‌آیند و نقاط بحرانی یک شهر هوشمند را بررسی می‌کنند که باتوجه‌به نیازهای مختلف مانند ترافیک، گزارش تصادفات، واپایش وضعیت فیزیکی برخی جاده‌ها، ردیابی برخی وسایل نقلیه و غیره؛ بنابراین، در سامانه پیشنهادی این نقاط توسط مدیر سامانه بررسی یا به‌صورت یک فایل متنی ذخیره می‌شود و سپس مسیر امکان‌پذیر برای تعدادی پهپاد با استفاده از یک الگوریتم تکاملی به نام الگوریتم ژنتیک محاسبه می‌شود. نتیجه‌گیری تجربی نویسنده نشان داد که نظام پیشنهادی راه‌حل‌های ترافیکی مبتنی بر پهپاد عملی و کارآمد را تولید می‌کنند.
\section{جاگیری و مکان‌یابی پهپادها}
در مقالهٔ \cite{Pasandideh2023}، نویسندگان یک الگوریتم بهبودیافتهٔ \lr{PSO} برای حل مسئلهٔ مکان‌یابی پیشنهاد می‌دهند. آن‌ها مسئله‌ای را با فرموله‌سازی \lr{MINLP} ارایه کرده‌اند که در آن مکان پهپادها و تعداد بهینهٔ پهپادها به طور مشترک با استفاده از روش پیشنهادی که بر اساس ادغام الگوریتم‌های \lr{PSO} و \lr{K-means} است، به دست می‌آید. یک پروتکل ارتباطی سفارشی نیز برای تبادل داده بین \lr{UE} و کنترل‌کنندهٔ شبکه پیشنهاد شده است. الگوریتم پیشنهادی تأخیر کم و نرخ از اتلاف بستهٔ پایین برای \lr{UE} را درحالی‌که حداکثر پوشش \lr{UE} را فراهم می‌کند، ارایه داده‌اند. بااین‌حال، مدل ریاضی ارایه شده، جنبهٔ مصرف انرژی \lr{UAV-BS}ها را نادیده گرفته و سناریوهای بررسی شده به تعداد محدودی از \lr{UE} محدود هستند.
در پژوهش \cite{10004755}، شاهد ‌آن هستیم که نویسندگان الگوریتم تخصیص پهنای باند و مکان‌یابی پهپادهای آگاه به بک‌هاول را معرفی می‌کنند که از الگوریتم ژنتیک (\lr{GA}) برای تخصیص پهنای باند موجود به \lr{UE} و تعیین مکان \lr{UAV-BS}ها با درنظرگرفتن ارتباط دید مستقیم (\lr{LOS}) با ایستگاه پایه ماکرو (\gls{MBS}) استفاده می‌کند. نویسندگان این مقاله پیشنهاد می‌کرده‌اند که از ارتباط فضای آزاد نوری (\lr{FSO}) به‌عنوان راه‌حل بک‌هاول بین \lr{UAV-BS} و \gls{MBS} استفاده شود. بر اساس \cite{10004755}، استفاده از \lr{UAV-BS} با بک‌هاول معمولی در منطقهٔ حادثه‌دیده که \gls{MBS} بیش از ۱۰ کیلومتر فاصله دارد، منجر به ظرفیت پایین پیوند می‌شود. \lr{FSO} نرخ‌های داده‌ای از گیگابیت بر ثانیه تا ترابیت بر ثانیه را بر چندین کیلومتر ارایه می‌دهد. بااین‌حال، حفظ تراز نوری دقیق بین فرستنده \lr{FSO} در \gls{MBS} و گیرنده \gls{FSO} در \lr{UAV-BS} به دلیل احتمال حرکت و لرزش \lr{UAV-BS} در هوا چالش‌برانگیز است. درحالی‌که سامانه‌های دقت بالا، ردیابی و نشانه‌گذاری (\gls{ATP}) تراز را تضمین می‌کنند، وزن آن‌ها تخلیه سریع باتری \lr{UAV-BS} را تسریع می‌کند. برای حل این مسئله، نویسندگان پیشنهاد می‌دهند که از فناوری انتقال هم‌زمان اطلاعات و توان بی‌سیم برای \lr{FSO} استفاده شود.
در مقالهٔ \cite{8642333}، نویسندگان مسئله مکان‌یابی \lr{UAV-BS} را به‌عنوان مسئله کوله‌پشتی در نظر گرفته‌اند که باتوجه‌به نیازهای ترافیک شبکه و تراکم کاربران نهایی (\lr{UE}) در منطقه مدل‌سازی شده است، برخلاف روش‌های دیگر که سعی در پوشش حداکثر منطقه با استفاده از \lr{UAV-BS} دارند. در الگوریتم پیشنهادی که بر اساس مفهوم الگوریتم ژنتیک (\gls{GA}) طراحی شده است، نویسندگان محدوده‌های شبکه متفاوتی را برای پهپادها باتوجه‌به تراکم \lr{UE} پیشنهاد داده‌اند که تضمین می‌کند نرخ داده‌ای و کاهش توان انتقال را فراهم می‌کند. الگوریتم مکان‌یابی سه‌بعدی \lr{UAV-BS} آگاه به تراکم، سه \lr{UE} تصادفی را انتخاب کرده و سپس محیطی را که محدوده \lr{UAV-BS} است، به دست می‌آورد. هرچه منطقه کوچک‌تر باشد، نرخ بیت بالاتری به \lr{UE} ارایه می‌شود.
یافته‌های مقالهٔ \cite{Islam2022} نشان می‌دهد که راه‌حل پیشنهادی به طور مؤثری الگوهای ترافیک خودروها را برای بهینه‌سازی تنظیمات ارتفاع پهپاد پیش‌بینی می‌کند. این مقاله از الگوریتم \lr{PSO} برای تعیین مکان‌های بهینهٔ استقرار پهپاد در کل شبکه استفاده می‌کند و عواملی مانند تراکم، جهت حرکت، و داده‌های پوشش قبلی را در نظر می‌گیرد. سپس الگوریتم \lr{PSO} به‌صورت تکراری به کار گرفته می‌شود تا تعداد بهینهٔ پهپادها را برای رسیدن به یک آستانهٔ پوشش شبکه ازپیش‌تعیین‌شده تعیین کند. نتایج شبیه‌سازی به‌وضوح نشان می‌دهد که طرح پیشنهادی که به نام طرح افزایش پوشش شبکه همکارانه شناخته می‌شود، عملکرد شبکه موقت وسایل نقلیه (\lr{VANET}) را از نظر معیارهای کلیدی شامل نسبت تحویل بسته، تعداد گام‌ها، تأخیر انتها به انتها و گذردهی به طور قابل‌توجهی بهبود می‌بخشد.
نویسندگان در مقاله \cite{Shakhatreh2021} مسئله مکان‌یابی بهینه \lr{UAV}ها به‌عنوان ایستگاه‌های پایه هوایی را برای حداکثرکردن توان عملیاتی کل دستگاه‌های بی‌سیم مدل‌سازی کرده‌اند. نویسندگان مکان پهپادها را با استفاده از الگوریتم \lr{PSO} پیدا می‌کنند و سپس توان عملیاتی کل پهپادها را با استفاده از سه رویکرد مختلف به دست می‌آورند: (1) رویکرد تخصیص توان مساوی، (2) رویکرد پر کردن آب، و (3) رویکرد اصلاح شده پر کردن آب. 
نویسندگان در مقاله \cite{Abu-Baker2023} نیز از الگوریتم \lr{PSO} و الگوریتم ژنتیک (\lr{GA}) برای خوشه‌بندی شبکه‌های حسگر بی‌سیم (\lr{WSN}) و برای جستجوی بهینه باهدف افزایش عمر باتری استفاده کرده‌اند.
در مقاله \cite{Ouamri2022}، نویسندگان چالش مکان‌یابی پهپاد را با استفاده از الگوریتم بهینه‌سازی گرگ خاکستری (\gls{GWO}) مورد بررسی قرار می‌دهند. هدف اصلی تحقیق آن‌ها بهینه‌سازی پوشش است. این مطالعه نتایج شبیه‌سازی برای سناریویی با ۱۰ پهپاد و ۲۰۰ کاربر را ارایه می‌دهد که نرخ پوشش گزارش‌شده ۸۵ درصد را به دست می‌آورد. بااین‌حال، نویسندگان عوامل مهمی همچون اثر بلوکه‌کننده و همپوشانی بین پهپادها را که برای مکانیزم‌های تحویل مؤثر بسیار حیاتی هستند، نادیده گرفته‌اند.
در مقاله \cite{Mandloi2023}، نویسندگان به بررسی استفاده بهینه از پهپادها در سناریوهایی می‌پردازند که با محیط‌های آسیب‌دیده یا کمبود زیرساخت‌های ارتباطی مشخص می‌شوند. درحالی‌که تمرکز اصلی بر تعیین تعداد ایده‌آل پهپادها است، مدل شامل چندین مرحله مجزا است و استقرار و مکان‌یابی بهینه این پهپادها به‌عنوان اهداف کلیدی در این تحقیق مطرح شده‌اند. این مقاله سه مرحله مهم را شرح می‌دهد: کاربرد تکنیک \lr{K-Means}، روش‌های فرموله‌سازی، و استفاده از یک الگوریتم ژنتیک مبتنی بر فازی.
در \autoref{tab:related-work} مروری بر مطالعاتی ارایه می‌دهد که از روش‌های فراابتکاری برای بهینه‌سازی مصرف توان در سامانه‌های ارتباطی مبتنی بر پهپاد استفاده کرده‌اند. روش پیشنهادی که در \autoref{tab:related-work} به تصویر کشیده شده است، تمام اهداف اصلی مشخص‌شده را در نظر می‌گیرد.
\begin{table*}%[ht]
\centering
\caption{خلاصه ویژگی های اصلی مرتبط ترین مطالعات}
%\begin{tabular}{|c|c|c|c|c|c|}
\begin{tabular}{llllll}
\hline
مرجع & روش‌ها & پوشش‌دهی & اتصال & انرژی & تاخیر \\ \hline
\cite{Pasandideh2023}       &     PSO      &       \checkmark   &       \checkmark       &        &     \checkmark    \\
\cite{10004755}       &     BROAD (Heuristic)     &    \checkmark      &        \checkmark      &        &      \\   
\cite{8642333}      &       GA     &      \checkmark    &       \checkmark       &   \checkmark     &         \\
%\cite{Aloqaily2022}        &     Machine learning technique      &    \checkmark      &             &   \checkmark     &       \\  
\cite{Islam2022}       &     PSO      &    \checkmark      &              &        &     \checkmark    \\
%\cite{Guo2022}       &     Numerical search (NS)       &    \checkmark      &              &     \checkmark   &         \\
\cite{Shakhatreh2021}       &      PSO      &     \checkmark     &     \checkmark         &        &         \\
\cite{Abu-Baker2023}  &     PSO + GA       &          &              &    \checkmark    &         \\
%\cite{Wang2023}      &      Heuristic Search      &          &         \checkmark     &        &     \\    
%\cite{Saadi2023}     &    IMRFO-TS        &      \checkmark    &     \checkmark         &     \checkmark   &      \\   
\cite{Ouamri2022}      &       GWO     &      \checkmark    &             &       &         \\
پژوهش جاری      &       JAYA     &     \checkmark     &      \checkmark        &     \checkmark   &    \checkmark     \\ \hline
\end{tabular}
\label{tab:related-work}
\end{table*}
\section{تحرک پهپادها}
در مقاله \cite{9078878} نویسندگان به یکی از چالش‌های اصلی در شبکه‌های پهپادی، یعنی مدل‌سازی و مدیریت تحرک پهپادها پرداخته است. این مقاله به ارزیابی عملکرد مدل‌های تحرک مختلف در شبکه‌های پهپادی را در نظر گفته است.
این مقاله به بررسی چندین مدل تحرک رایج، می‌پردازد. این مدل‌ها شامل: 
\begin{itemize}
	\item \gls{RW}
	\item \gls{RWP}
	\item \gls{GMD}
	\item \gls{BMM}
\end{itemize}

برای ارزیابی عملکرد هر یک از مدل‌های تحرک، از شبیه‌سازی‌های گسترده‌ای استفاده شده است. معیارهای ارزیابی شامل: پوشش‌دهی شبکه: میزان ناحیه تحت پوشش پهپادها، میزان گذردهی: مقدار داده‌ای که به طور موثر از طریق شبکه منتقل می‌شود، تأخیر انتها به انتها: زمان موردنیاز برای انتقال داده از مبدأ به مقصد و مصرف انرژی: مقدار انرژی مصرفی توسط پهپادها است. این مقاله به این نتیجه می‌رسد که انتخاب مدل تحرک مناسب برای پهپادها در شبکه‌های سلولی بستگی به نیازهای خاص کاربردی و محدودیت‌های سامانه دارد. مدل‌های پیچیده‌تر معمولاً عملکرد بهتری دارند اما نیاز به منابع محاسباتی بیشتری دارند؛ بنابراین، تعادل بین عملکرد و پیچیدگی باید در نظر گرفته شود.
در پژوهش \cite{Meer2024} نویسندگان به بررسی یک روش مبتنی بر یادگیری برای بازپیکربندی پویا خوشه‌ها و مدیریت تحرک پهپادها می‌پردازد. این مقاله نشان می‌دهد که استفاده از روش‌های یادگیری مبتنی بر تقویتی و بازپیکربندی پویا خوشه‌ها می‌تواند به طور موثر مدیریت تحرک پهپادها را بهبود بخشد. شکل‌دهی پرتو سه‌بعدی نیز به‌عنوان یک روش برای بهبود پوشش‌دهی و کاهش تداخل در شبکه‌های ارتباطی مبتنی بر پهپاد معرفی شده است.
نتایج شبیه‌سازی‌ها این مقاله نشان می‌دهند که روش پیشنهادی به طور قابل‌توجهی عملکرد شبکه را بهبود می‌بخشد. معیارهای ارزیابی شامل پوشش‌دهی شبکه، کیفیت خدمات و مصرف انرژی است.
همان‌طور که در مقاله \cite{Alshaibani2022} شبکه‌های ناهمگن فوق متراکم به دلیل توانایی پشتیبانی از تعداد زیادی از دستگاه‌ها و ارایه پوشش گسترده، به‌عنوان یک راه‌حل کلیدی برای نسل‌های آینده شبکه‌های بی‌سیم مطرح شده‌اند. پهپادها نیز می‌توانند به‌عنوان ایستگاه‌های پایه هوایی در این شبکه‌ها استفاده شوند تا پوشش و ظرفیت شبکه را افزایش دهند. این مقاله نشان می‌دهد که مدیریت تحرک پهپادها در شبکه‌های ناهمگن متراکم می‌تواند تأثیر بسزایی در بهبود عملکرد شبکه داشته باشد. استفاده از الگوریتم‌های بهینه‌سازی و یادگیری ماشین می‌تواند به طور قابل‌توجهی مصرف انرژی را کاهش داده و کیفیت خدمات را بهبود بخشد.
در مقاله \cite{Meer2023} نویسندگان به بررسی واپایش پویا و مدیریت تحرک پهپادها با استفاده از یادگیری تقویتی می‌پردازند. نتایج شبیه‌سازی‌ها نشان می‌دهند که روش پیشنهادی مبتنی بر یادگیری تقویتی می‌تواند به طور قابل‌توجهی عملکرد شبکه را بهبود بخشد. معیارهای ارزیابی شامل: پوشش‌دهی شبکه، مصرف انرژی و کیفیت خدمات است. همچنین نویسندگان در پژوهشی \cite{Chowdhury2020} به بررسی یک رویکرد مبتنی بر یادگیری برای مدیریت تحرک پهپادهای متصل به شبکه‌های سلولی پرداخته‌اند.
در پژوهش \cite{10398469} نویسندگان به دنبال هدف حفظ دسترسی به خدمات با استفاده از مدیریت تحرک پهپادها بوده‌اند، در این مقاله با معرفی یک روش به دنبال در دسترس‌پذیر بودن خدمات برای کاربران زمینی بوده است، یکی از ایراداتی که در این مقاله دیده می‌شود، در نظر گرفته نشدن مصرف انرژی پهپادها است، چرا که می‌دانیم یکی از اصلی‌ترین محدودیت‌های استفاده از پهپادها استفاده بهینه از مصرف انرژی است که می‌توانیم با درنظرگرفتن این جنبه سناریوهای پروازی و عملیات نجات را نزدیک‌تر به واقعیت مدل‌سازی کنیم، یکی از اهداف اصلی این پژوهش ما رسیدن به این هدف است.

















