\documentclass{Boostan-UserManual}





\begin{document}

سلام این یک مثال است:

\begin{note}
شهر مردگان، شهر انسان های «بی دفاع» است. این تعبیر اقتباس از قرآن کریم است که «غیبت» را خوردن گوشت «مرده» خوانده است. 
\begin{equation}
A = B + \sin (x)
\end{equation}
در تفاسیر آمده است که خداوند «انسان بی دفاع» را که به دلیل عدم حضور در مجلس بدگویی نمی تواند از خود دفاع کند، «مرده» دانسته است. پس آنجا که نسبت های ناروا دادن مباح، و دفاع کردن ممنوع است، در حقیقت «شهر مردگان» است.
\end{note}

\begin{problem}
شهر مردگان، شهر انسان های «بی دفاع» است. این تعبیر اقتباس از قرآن کریم است که «غیبت» را خوردن گوشت «مرده» خوانده است . در تفاسیر آمده است که خداوند «انسان بی دفاع» را که به دلیل عدم حضور در مجلس بدگویی نمی تواند از خود دفاع کند، «مرده» دانسته است. پس آنجا که نسبت های ناروا دادن مباح، و دفاع کردن ممنوع است، در حقیقت «شهر مردگان» است.


\end{problem}


\begin{refer}
شهر مردگان، شهر انسان های «بی دفاع» است. این تعبیر اقتباس از قرآن کریم است که «غیبت» را خوردن گوشت «مرده» خوانده است.
\begin{latin}
\lstset{numbers=none,frame=none}
\begin{lstlisting}
for i:=maxint to 0 do
begin
{ do nothing }
end;
\end{lstlisting}
\end{latin}
  در تفاسیر آمده است که خداوند «انسان بی دفاع» را که به دلیل عدم حضور در مجلس بدگویی نمی تواند از خود دفاع کند، «مرده» دانسته است. پس آنجا که نسبت های ناروا دادن مباح، و دفاع کردن ممنوع است، در حقیقت «شهر مردگان» است.
\end{refer}

\begin{info}
شهر مردگان، شهر انسان های «بی دفاع» است. این تعبیر اقتباس از قرآن کریم است که «غیبت» را خوردن گوشت «مرده» خوانده است . 
در تفاسیر آمده است که خداوند «انسان بی دفاع» را که به دلیل عدم حضور در مجلس بدگویی نمی تواند از خود دفاع کند، «مرده» دانسته است. پس آنجا که نسبت های ناروا دادن مباح، و دفاع کردن ممنوع است، در حقیقت «شهر مردگان» است.
\begin{equation}
A = B + \sin (x)
\end{equation}
\end{info}

\begin{warning}{نکات مهم}
شهر مردگان، شهر انسان های «بی دفاع» است. این تعبیر اقتباس از قرآن کریم است که «غیبت» را خوردن گوشت «مرده» خوانده است . در تفاسیر آمده است که خداوند «انسان بی دفاع» را که به دلیل عدم حضور در مجلس بدگویی نمی تواند از خود دفاع کند، «مرده» دانسته است. پس آنجا که نسبت های ناروا دادن مباح، و دفاع کردن ممنوع است، در حقیقت «شهر مردگان» است.
\end{warning}


\begin{goal}{نکات مهم}
شهر مردگان، شهر انسان های «بی دفاع» است. این تعبیر اقتباس از قرآن کریم است که «غیبت» را خوردن گوشت «مرده» خوانده است . 

در تفاسیر آمده است که خداوند «انسان بی دفاع» را که به دلیل عدم حضور در مجلس بدگویی نمی تواند از خود دفاع کند، «مرده» دانسته است.

  پس آنجا که نسبت های ناروا دادن مباح، و دفاع کردن ممنوع است، در حقیقت «شهر مردگان» است.
\end{goal}

\end{document}


