\section{اثبات \autoref{cor:verdu}}
\label{proof:verdu}
پارامتر
$\varrho^*$
را به صورت زیر تعریف می‌کنیم. 
\begin{equation}
\varrho^* \stackrel{\Delta}{=} \max_{i,j} \frac{\pi_i p_{j|i}}{\sum_{k} \pi_{k}p_{j|k}}.
\label{eq:alphastart}
\end{equation}
با استفاده از این تعریف،
\eqref{sdsafdsafsdflps}
به صورت زیر بازنویسی می‌گردد.
\begin{equation}
\label{eq:verdustar}
P_e \ge  (1-\varrho^*) \Pr[P_{I|J}\le \varrho^*]
\end{equation}
در ضمن می‌دانیم که
$\Pr(X)=\mathbb{E}\{\mathbf{1}(X)\}$
که در آن 
$\mathbf{1}(X)$
نشانگر
\gls{IndicatorFunction} $X$
است که به صورت زیر تعریف می‌شود.
\begin{equation}
\mathbf{1}(X) = \left\{
		\begin{array}{lr}
		1 \quad & \text{\rl{اگر $ A $ برقرار باشد.}} \\
		0 \quad & \text{سایر} \\
		\end{array}
		\right.
\end{equation}
بدین‌سان خواهیم داشت:
\begin{align}
P_e \ge &  (1-\varrho^*) \sum_{i,j} \pi_i p_{j|i} \mathbf{1}[\Pr(f^I=f^i|f^J=f^j)\le \varrho^*] \nonumber \\
\stackrel{\text{(a)}}{=} & (1-\varrho^*) \sum_{i,j} \pi_i p_{j|i} \mathbf{1}[\frac{\pi_{i}p_{j|i}}{\sum_{k} \pi_{k} p_{j|k}}\le \varrho^*] \nonumber \\
\stackrel{\text{(b)}}{=} & (1-\varrho^*) \sum_{i,j} \pi_i p_{j|i} \nonumber \\
= & (1-\varrho^*)=1-\max_{i,j} \frac{\pi_i p_{j|i}}{\sum_{k} \pi_{k}p_{j|k}},
\label{sdwsdwdw}
\end{align}
که در روابط فوق،
\lr{(a)} 
با استفاده از \gls{BayesianRule} بدست می‌آید؛ چراکه بر طبق این قانون داریم:
\begin{equation*}
\Pr(f^I=f^i|f^J=f^j)=\frac{\Pr(f^J=f^j|f^I=f^i) \Pr(f^I=f^i)}{\sum_{k} \Pr(f^J=f^j|f^I=f^k) \Pr(f^I=f^k)} = \frac{p_{i,j}p_{i}}{\sum_{k} p_{k,j}p_{k}}.
\end{equation*}
\lr{(b)}
نیز با قرار دادن 
$\varrho^*$
در
\eqref{eq:alphastart}
حاصل می‌گردد. 

\section{اثبات \autoref{cor:medard}}
\label{proof:medard}
با استفاده از 
\cite[نتیجه 2]{Christiansen2013Bounds} و 
برای 
\gls{RandomVariable} $f^I$
که از 
\gls{UniformDistribution}
تبعیت می‌کند، 
\gls{MarkovChain}
به صورت
$ f^I\longrightarrow f^J \longrightarrow \hat{f^I} $
در نظر می‌گیریم، که در آن
$ \hat{f^I} $ \gls{Estimator} \gls{ML} برای $f^I$
است. در این صورت 
\gls{LowerBound} $P_e$
به صورت زیر حاصل می‌گردد.
\begin{align}
P_e &\ge 1-\frac{1}{N} - \frac{\sqrt{(N-1)(\mathbb{E}[\frac{P_{I,J}}{\pi_I P^J}]-1)}}{N}\nonumber\\
& \ge 1-\frac{1}{N} - \frac{\sqrt{(N-1)(\sum_{i,j} \frac{p_{j|i}\pi_i}{\pi_i \sum_{k}p_{j|k}\pi_k} p_{j|i}\pi_i -1)}}{N}\nonumber\\
&
\ge 1-\frac{1}{N} - \frac{\sqrt{(N-1)(\sum_{i,j}\frac{p_{j|i}^2}{ \sum_{k}p_{j|k}}-1)}}{N}.
\end{align}







