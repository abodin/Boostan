\clearpage\newpage
\section*{چکیده}
\baselineskip=.90cm
امروزه نشان داده شده که حتی در صورت استفاده از سازوکارهای تامین امنیت محتوای داده، نظیر رمزنگاری داده‌ها، یک مهاجم باهوش می‌تواند اطلاعات مفیدی را با تحلیل رفتار تولید داده‌ها، بدست آورد. این بدان علت است   که با استفاده از سازوکارهای سنتی رمزنگاری، حریم‌خصوصی بسیاری از جنبه‌های تولید داده نظیر میانگین و پراش زمان تولید داده‌ها،  در برابر یک مهاجم باهوش قابل حفظ نخواهد بود. بدین‌سان علاوه بر استفاده از سازکارهای تامین امنیت محتوای داده و به عنوان مکمل آن، نیاز به ارایه راه کارهایی به منظور حفظ حریم خصوصی این جوانب نیز وجود دارد. به عبارت‌بهتر و از دیدگاه علم
\glspl*{ComputerNetwork}، \gls*{Adversary}
بدون داشتن اطلاعاتی در مورد 
\gls*{ApplicationLayer} و \gls*{TransportLayer}،
و فقط با داشتن اطلاعات زمانی ارسال 
\glspl*{Packet} در \gls*{NetworkLayer}، \gls*{Privacy} \gls*{User}
را به خطر بیاندازد. 

در این رساله، با استفاده از 
‎\gls*{QueuingTheory}‎ و \gls*{InformationTheory}
این جنبه از
\gls*{Privacy}
را مورد بررسی قرار خواهیم داد. ما در ابتدا به سراغ نرخ تولید داده در \gls*{SourceNode} می‌رویم. خواهیم گفت که یک
\gls*{Adversary}
باهوش می‌تواند با بهره‌گرفتن از
\gls*{FlowConservationLaw} و با استفاده از \gls*{Rate}، \gls*{TemporalAndStatisticalPrivacy} \gls*{User}
را به خطر بیاندازد. البته در ادامه خود را محدود به 
\gls*{Rate}
نخواهیم کرد و از یک
\gls*{Feature}
به صورت کلی سخن به میان خواهد آمد. برای
\gls*{Feature} \gls*{Rate}
راه‌کاری نیز مبتنی بر اضافه نمودن
\glspl*{DummyPacket}
ارایه می‌گردد. در ضمن گذری نیز بر 
\gls*{TemporalAndStatisticalPrivacy} در \glspl*{CachingSystem}
خواهیم داشت.
\gls*{Adversary} 
باشنود کانال در مرحله
\gls*{Delivery} در \glspl*{CachingSystem}،
می‌تواند 
\gls*{Privacy} \glspl*{User}
را به خطر بیاندازد، بدین‌سان که می‌تواند دریابد که با احتمال زیادی یک
\gls*{User}،
چه فایلی را درخواست کرده‌است. 

روش‌های پیشنهادی در کل این رساله، می‌بایست به نحوی باشد که  ضمن حفظ
\gls*{TemporalAndStatisticalPrivacy}، ‎\gls*{QoS}‎ \gls*{User}
را نیز حفظ نماید. تمامی روش‌های پیشنهادی ارایه شده در این رساله، بر پایه تعدادی
\gls*{OptimizationProblem}،
استوار است که 
\gls*{TradeOff} بین \gls*{Privacy} و \gls*{Cost}
را مدیریت می‌نماید. از سوی دیگر به منظور مدل‌سازی هرچه بهتر مفهوم
\gls*{Privacy}، 
به سراغ یافتن یک‌سری
\gls*{LowerBound} برای \gls*{AdversarysBestEstimationErrorProbability}
خواهیم رفت. در این راه از باندهای پیشنهاد شده در مبحث 
\glspl*{CommunicationChannel} که در علم \gls*{InformationTheory}
مورد بحث قرار می‌گیرد، استفاده کرده‌ایم.
در نهایت نیز یک مدل ریاضیاتی برای
\glspl*{CachingSystem}
و بالابردن
\gls*{Privacy}
در این نوع از سامانه‌ها با درنظر گرفتن میزان ترافیک مبادله شده پیشنهاد خواهیم داد. 

\textbf{کلمات کلیدی:}
\glsentrytext{Privacy} \glsentrytext{ContextOriented}، 
\glsentrytext{DummyPacket}، 
\glsentrytext{AdversarysBestEstimationErrorProbability}، 
\glsentrytext{InformationTheory}، 
\glsentrytext{CachingSystem}.




