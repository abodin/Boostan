\thispagestyle{empty} %prevents tex from numbering of this page
\begin{latin} % xepersian enviorment

\centerline{\textbf{\large{Abstract}}}
\vskip 1cm
 
Recent  investigations have clarified that not only insensitive data with no encryption methods, but amazingly also encrypted sensitive data may translate into invaluable information by an intelligent adversary. In the latter case, in spite of the protection that data encryption might
provide, there are many aspects related to the creation and delivery of messages that remain
unprotected by conventional security mechanisms. Complement to the data encryption methods,
other techniques are required to protect such contextual information to preserve the privacy of the
sources and have been the focus of attention of many research studies during the past few years. The former is known as data-oriented privacy and employs encryption methods to
protect data, while the latter is known as context-oriented privacy, which focuses on preservation
of the contextual information such as the location and the time when a message is generated
i.e., location and temporal privacy, respectively. 
 Inhibiting the adversary of being able to extract information from the traffic rate of source nodes is a complicated task unless taking into consideration the \emph{flow conservation law} effect of the transmitter queue. A reliable method of preserving the privacy that copes with the \emph{flow conservation law}. Augmenting dummy packets, however, bears redundancy and hence requires extra resources in terms of bandwidth and buffer requirements and more importantly suggests higher transmitting energy consumption. Grounded on the queueing and information theories, in this paper we present an efficient method that minimally augments dummy packets to preserve the source rate privacy at a given degree while preserving the delay distribution of the original packets intact, and thus does not affect the QoS parameters of the transmitted data in terms of delay and jitter. Then we extend our proposed approach to preserve privacy of a general feature. We present an approach that mixes the features of applications in the source node such
that maximizes the ambiguity of adversary. Finally, we formulate a mathematical model for privacy preserving of a caching system and then present a method so as to cache
files in an efficient manner such that maximizes the degree of
privacy preservation while maintains the average delivery load at
a given level.

\end{latin}
