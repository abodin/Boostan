\documentclass{report}
\usepackage{algorithm}
\usepackage{algcompatible}

%زی‌پرشین (به انگلیسی: XePersian) یک بسته حروف‌چینی رایگان و متن‌باز برای نگارش مستندات پارسی/انگلیسی با زی‌لاتک است.
% در واقع، زی‌پرشین، کمک می‌کند تا به آسانی، مستندات را به پارسی، حروف‌چینی کرد. این بسته را وفا خلیقی نوشته است،
% و به طور منظم، آن را بروز‌رسانی کرده و باگ‌های آن را رفع می‌کند.
% نکته مهم این جا است که بسته Xepersian برای پشتیبانی از زبان فارسی آورده شده است، و 
% می بایست آخرین بسته ای باشد که شما وارد می کنید، دقت کنید: آخرین بسته
\usepackage{xepersian}

%%  با دستور زیر می توانید فونتی مخصوص عبارات فارسی تعیین کنید:
\settextfont[Scale=1.2]{XB Niloofar}
%% شما با دستور زیر بعد از فراخوانی بسته xepersian می توانید فونت انگلیسی را تعیین کنید
%% دقت کنید که عبارات انگلیسی شما باید در دستور \lr{} قرار گیرد تا xepersian بتواند بفمهد که این عبارات انگلیسی است
\setlatintextfont[Scale=1]{Times New Roman}

\begin{document}
	
\begin{algorithm}
	\caption{
		خارج کردن صف از حالت پایدار
	}
	\label{alg1}
	\begin{latin}
		\begin{algorithmic}[1]
			\REQUIRE Privacy level, delay condition, $\lambda$. 
			 \STATEx\textbf{Input:}  Privacy level, delay condition
			\STATEx  \textbf{Output:}  Privacy level, delay condition
			\STATE\textbf{Input:}  Privacy level, delay condition
			\STATE  \textbf{Output:}  Privacy level, delay condition
			\STATE \textbf{Compute} $L_{\max}  , L_{\min} , \mu_{0} , \mu_{Max}, \mu_{Min}, \xi_{\uparrow}, \xi_{\downarrow}$
			\WHILE {$!$ Packet arrive} 
			\STATE Wait
			\ENDWHILE
			\STATE Compute buffer length ($L$).
			\IF {$L<L_{\min}$}
			\STATE $\mu = \min (\mu - \xi_{\downarrow},\mu_{Max})$
			\ELSIF {$L>L_{\max}$}
			\STATE $\mu = \min (\mu + \xi_{\uparrow},\mu_{Min})$
			\ENDIF
			\IF {$\lambda$ changes}
			\STATE Go to 1
			\ELSE
			\STATE Go to 2
			\ENDIF
		\end{algorithmic}
	\end{latin}
\end{algorithm}
\end{document}



