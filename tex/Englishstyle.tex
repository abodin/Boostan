\usepackage[usenames,dvipsnames,table]{xcolor}
\usepackage[top=2cm,bottom=2cm,left=1.5cm,right=1.5cm]{geometry}
\usepackage{graphicx}
\usepackage{pifont}
\usepackage{subcaption}
\usepackage{amsmath,amsthm,amssymb}
\usepackage{float}
\usepackage{tikz}
\usetikzlibrary{arrows,calc}


\usepackage{xparse}
\usepackage{ifthen}
\usepackage{tikz}
\usepackage[explicit]{titlesec}
\usepackage{tocloft,calc}
\usepackage{listings}
\usepackage{hyperref}
\hypersetup{citecolor={Green}}

 \usepackage[acronym]{glossaries}

\newcommand{\hand}{\item[\Large\color{blue}\ding{43}]}

\definecolor{quotationcolour}{HTML}{F0F0F0}
\definecolor{quotationmarkcolour}{HTML}{1F3F81}

% Double-line for start and end of epigraph.
\newcommand{\epiline}{\color{magenta}\hrule \vskip -.2em \hrule}
% Massively humongous opening quotation mark.
\newcommand{\hugequote}{%
  \fontsize{42}{48}\selectfont \color{quotationmarkcolour} \textbf{``}
  \vskip -.5em
}

% Beautify quotations.
%\newcommand{\epigraphs}[2]{%
%
%  \begin{flushright}
%  \colorbox{quotationcolour}{%
%    \parbox{\textwidth}{%
%    {\epiline} \vskip 1em {\hugequote} \vskip -.5em
%    \parindent 2.2em \baselineskip=.8cm
%    #1\begin{flushright}\textsc{#2}\end{flushright}
%    {\epiline}
%    }
%  }
%  \end{flushright}
%}

\lstset{
	backgroundcolor=\color{Plum!5},
	frameround=fttt,
	frame=trBL,
	aboveskip=5mm,
	belowskip=4mm,
	language=sh,
	%numbers=left,
	basicstyle=\ttfamily\footnotesize,
	keywordstyle=\color{blue}\bfseries,
	keywords={set,show,read,add,sudo,service,ping,unlock},
	stringstyle=\ttfamily\color{red}, 
	commentstyle=\color{Green},
	lineskip = .4pt,
	captionpos=b, 
	breaklines=true,
	showstringspaces=true,
	emph={RAN},emphstyle={\color{Plum}}
}% 


\hypersetup{
	colorlinks=true, 
} % M


\newcommand{\bigqm}[1][1]{\textcolor{red}{\text{\larger[#1]{\textbf{?}}}}}


\newcommand{\handBS}{\noindent\textcolor{ForestGreen}{\Huge\ding{45}}}
\NewDocumentEnvironment{note}{g g}{
\tikzstyle{mybox1} = [draw=YellowGreen, fill=green!15,very thick, rectangle, rounded corners, inner sep=10pt, inner ysep=20pt]
\tikzstyle{fancytitle1} =[fill=YellowGreen, text=white]
\tikzstyle{fancytitle2} =[fill=YellowGreen!5, text=white]
\tikzstyle{fancytitle3} =[fill=white, text=white]
	\begin{center}
	\vskip 2mm
		\begin{tikzpicture}
			\node [mybox1] (box)\bgroup
				\IfFileExists{warining.png}{\begin{minipage}{.85\textwidth}}{\begin{minipage}{.93\textwidth}}
			\baselineskip=.6cm
}{%
			\end{minipage}
			\egroup;
			\IfValueTF{#1}{\node[fancytitle1, left=10pt] at (box.north east) {\large \textbf{#1}};}{}%
				\IfFileExists{warining.png}
				{\node[fancytitle3, left=3pt,   rounded corners] at (box.west) {\includegraphics[width=.08\textwidth]{warining}}; }
				{\node[fancytitle2,  rounded corners] at (box.east) {\handBS};}
		\end{tikzpicture}
	\end{center}
}%


\NewDocumentEnvironment{problem}{g g}{
\tikzstyle{mybox1} = [draw=red, fill=red!15,very thick, rectangle, rounded corners, inner sep=10pt, inner ysep=20pt]
\tikzstyle{fancytitle1} =[fill=red, text=white]
\tikzstyle{fancytitle2} =[fill=red!5, text=white]
\tikzstyle{fancytitle3} =[fill=white, text=white]
	\begin{center}
		\begin{tikzpicture}
			\node [mybox1] (box)\bgroup
				\IfFileExists{infoRR.png}{\begin{minipage}{.85\textwidth}}{\begin{minipage}{.93\textwidth}}
			\baselineskip=.6cm
}{%
			\end{minipage}
			\egroup;
			\IfValueTF{#1}{\node[fancytitle1, left=10pt] at (box.north east) {\large \textbf{#1}};}{\node[fancytitle1, left=10pt] at (box.north east) {\large \textbf{Problem}};}%
				\IfFileExists{infoRR.png}
				{\node[fancytitle3, left=3pt,   rounded corners] at (box.west) {\includegraphics[width=.08\textwidth]{infoRR}}; }
				{\node[fancytitle2,  rounded corners] at (box.east) {\handBS};}
		\end{tikzpicture}
	\end{center}
}%

\renewcommand{\figurename}{Fig.}

\defglsentryfmt[acronym]{\glsentryname{\glslabel}\ifglsused{\glslabel}{}{\footnote{\glsentrydesc{\glslabel}}}}

\newcounter{pointNum}
\newcommand{\newP}{%
\addtocounter{pointNum}{1}%
\noindent\textcolor{Green}{\textbf{\arabic{pointNum})~~}}%
}

\newcommand{\idx}[1]{\index{#1}#1}
%% می تواند برای ارایه نکات در محیط itemize به کار رود، روند این کار به این صورت است،  (شکل یک تیر)
\newcommand{\arcm}{\item[\Large\color{red}\ding{247}]}
\newcommand{\arcmO}{\noindent\textcolor{red}{\Large\ding{247}}\;}
%% این شکل می‌تواند برای بیان مزایای یک قضیه بکار رود (شکل تیک)
\newcommand{\tick}{\item[\large\color{green}\ding{52}]}
\newcommand{\tickO}{\noindent\textcolor{green}{\Large\ding{52}}\;}
%% برای  بیان معایب و یا نکات منفی (شکل یک ضربدر)
\newcommand{\X}{\item[\Large\color{red}\ding{56}]}
\newcommand{\XO}{\noindent\textcolor{red}{\LARGE\ding{56}}\;}
%% بیان موارد یک قضیه (شکل یک دست)

\newcommand{\handO}{\noindent\textcolor{blue}{\LARGE\ding{45}}\;}
%% برای مواردی که: این موارد شامل .... می شود، توسط عناصر زیر مشخص می شود (شکل یک درخت)
%% برای نوشتن  پارامتر‌ها، 
\newcommand{\tree}{\item[\Large\color{ForestGreen}\ding{171}]}
\newcommand{\treeO}{\noindent\textcolor{ForestGreen}{\Large\ding{171}}\;}
%% برای این که چند مورد را تعریف کنیم (علامت دست که دو گرفته)
\newcommand{\two}{\item[\LARGE\color{blue}\ding{44}]}
\newcommand{\twoO}{\noindent\textcolor{blue}{\LARGE\ding{44}}\;}
%% (شکل یک قیچی)
\newcommand{\sci}{\item[\footnotesize\color{OrangeRed}\ding{108}]}
\newcommand{\sciO}{\noindent\textcolor{OrangeRed}{\footnotesize\ding{108}}\;}

\newcommand{\starE}{\item[\Large\color{Plum}\ding{97}]}
\newcommand{\starEO}{\noindent\textcolor{Plum}{\Large\ding{97}}\;}
%% برای حالت‌هایی که فرضیات داریم. 
\newcommand{\music}{\item[\Large\color{Green}\ding{161}]}
\newcommand{\musicO}{\noindent\textcolor{Green}{\Large\ding{161}}\;}

\newcommand{\gol}{\item[\Huge\color{RubineRed}\ding{96}]}
\newcommand{\golO}{\noindent\textcolor{RubineRed}{\Huge\ding{96}}\;}


\makeatletter
\patchcmd{\hyper@makecurrent}{%
    \ifx\Hy@param\Hy@chapterstring
        \let\Hy@param\Hy@chapapp
    \fi
}{%
    \iftoggle{inappendix}{%true-branch
        % list the names of all sectioning counters here
        \@checkappendixparam{chapter}%
        \@checkappendixparam{section}%
        \@checkappendixparam{subsection}%
        \@checkappendixparam{subsubsection}%
        \@checkappendixparam{paragraph}%
        \@checkappendixparam{subparagraph}%
    }{}%
}{}{\errmessage{failed to patch}}

\newcommand*{\@checkappendixparam}[1]{%
    \def\@checkappendixparamtmp{#1}%
    \ifx\Hy@param\@checkappendixparamtmp
        \let\Hy@param\Hy@appendixstring
    \fi
}
\makeatletter

\newtoggle{inappendix}
\togglefalse{inappendix}

\apptocmd{\appendix}{\toggletrue{inappendix}}{}{\errmessage{failed to patch}}

\makeatletter

%%% تعریف یکسری متغیرها که کاربر می‌تواند بعدا آن ها را مقداردهی کند. 

\gdef\@type{}
\def\type#1{\gdef\@type{#1}}

%% عنوان محصول را تعیین می‌کند. این عنواند در ایجاد عنوان در مستند استفاده
%% می‌شود این عنوان در هر مستند باید ایجاد شود در غیر این صورت از عنوان
%% پیشفرض استفاده خواهد شد.
\gdef\@title{}
\def\title#1{\gdef\@title{#1}}

%% زیر عنوان یک متن ساده را تعیین می‌کند که یک هدف مهم محصول را تعیین می‌کند
%% این عنوان می تواند برای یک محصول در نظر گرفته نشود. از این داده برای 
%% ایجاد عنوان و سایر مکان های محصول استفاده می‌شود.
\gdef\@subtitle{}
\def\subtitle#1{\gdef\@subtitle{#1}}
%% افراد و گروه های که در تهیه این مستند و محصول همکاری داشته اند را تعیین
%% می کند این داده همواره باید بیان شود. این داده در نوشتن عنوان و دیگر قسمت
%% های مستند مورد استفاده قرار می‌گیرد.
\gdef\@author{}
\def\author#1{\gdef\@author{#1}}
\newcommand{\authorText}{\@author\,}
%% تاریخ نهایی نوشتن مستند را تعیین می‌کند این تاریخ در نوشتن عنوان استفاده
%% می‌شود این تارخ باید تعیین شود در غیر این صورت به صورت پیش فرض یک تاریخ
%% برای آن استفاده می شود.
\gdef\@date{\today}
\def\date#1{\gdef\@date{#1}}

\gdef\@supervisor{} 
\def\supervisor#1{\gdef\@supervisor{#1}}

\gdef\@adviser{}
\def\adviser#1{\gdef\@adviser{#1}}

\gdef\@session{}
\def\session#1{\gdef\@session{#1}}

\gdef\@institute{}
\def\institute#1{\gdef\@institute{#1}}

\gdef\@faculity{}
\def\faculity#1{\gdef\@faculity{#1}}

\gdef\@group{}
\def\group#1{\gdef\@group{#1}}

\gdef\@community{}
\def\community#1{\gdef\@community{#1}}

\gdef\@forwhat{}
\def\forwhat#1{\gdef\@forwhat{#1}}

\gdef\@field{}
\def\field#1{\gdef\@field{#1}}


\gdef\@version{}
\def\version#1{\gdef\@version{#1}}

%%% نام فایلی لوگوی مورد استفاده در نوشتار توسط این پارامتر مشخص می‌شود. 
\gdef\@logofile{logonotfound}
\def\logofile#1{\gdef\@logofile{#1}}
%%% اندازه فایل لوگوی موجود در متن توسط این پارامتر مشخص می‌شود.
\gdef\@logoScale{.3\textwidth}
\def\logoScale#1{\gdef\@logoScale{#1}}

\newcommand{\lshortStyle}{
	%% استایلی شبیه به استایل شروع کتاب "مقدمه‌ای نه چندان کوتاه بر Latex"
	\definecolor{authorcol}{rgb}{.51,0,.51}
	\begin{flushleft}
	\vspace*{\stretch{.1}}
	\begin{flushright}\includegraphics[width=\@logoScale]{\@logofile}\end{flushright}
	\newlength{\centeroffset}
	\setlength{\centeroffset}{-0.5\oddsidemargin}
	\addtolength{\centeroffset}{0.5\evensidemargin}
	\addtolength{\textwidth}{-\centeroffset}
	%% توسط این دستور تمامی footer و header های صفحه را حذف می‌کنیم. 
	\thispagestyle{empty}
	
	\vspace*{\stretch{2}}
	
	\noindent\hspace*{\centeroffset}\makebox[0pt][l]{
		\begin{minipage}{\textwidth}
			\flushleft
			{\fontfamily{pzc}\selectfont \Huge\textcolor{magenta}\@title \\*[10pt]}
			\noindent\color{gray}{\rule[-1ex]{\textwidth}{5pt}\\*[4mm]
			\hfill{\large \bfseries\@type }}
		\end{minipage}
	}%
	
	\vspace{\stretch{2}}
	
	\noindent\hspace*{\centeroffset}\makebox[0pt][l]{
		\begin{minipage}{\textwidth}
			{\flushleft\textcolor{authorcol}{\bfseries\@author\\*[5pt]}}
			{\flushleft\textcolor{authorcol}{\bfseries\@supervisor\\*[5pt]}}
			{\flushleft\textcolor{authorcol}{\bfseries\@date}\\}
		\end{minipage}
	}%
		
	\vspace{\stretch{1.5}}

	\end{flushleft}
	\clearpage
}%

\makeatother


\newcommand*\chapterlabel{}
\titleformat{\chapter}
  {\gdef\chapterlabel{}
   \normalfont\Huge\bfseries}
  {\gdef\chapterlabel{\thechapter\ }}{0pt}
  {\begin{tikzpicture}[remember picture,overlay]
    \node[yshift=-3cm] at (current page.north west)
      {\begin{tikzpicture}[remember picture, overlay]
        \draw[fill=BlueGreen] (0,0) rectangle
          (\paperwidth,3cm);
        \node[anchor=east,xshift=0.95\paperwidth,rectangle,
              rounded corners=20pt,inner sep=11pt,
              fill=MidnightBlue,text width=\dimexpr0.9\paperwidth-22pt\relax,align=justify]
             {\color{white}\chapterlabel #1};
       \end{tikzpicture}
      };
   \end{tikzpicture}
  }
%\titlespacing*{\chapter}{0pt}{50pt}{-60pt}


\newcommand{\setupname}[1]{%
  \addtocontents{toc}{%
    \unexpanded{\unexpanded{%
      \renewcommand{\cftchappresnum}{#1 }%
      \setlength\cftchapnumwidth{\widthof{\bfseries #1 }}%
      \addtolength\cftchapnumwidth{\fixedchapnumwidth}%
    }}%
  }%
}
\AtBeginDocument{\edef\fixedchapnumwidth{\the\cftchapnumwidth}}

