\usepackage{amsmath,amssymb,amsthm}
\usepackage[usenames,dvipsnames,table]{xcolor}
\usepackage[top=2cm,bottom=2cm,left=1.5cm,right=1.5cm]{geometry}
\usepackage{graphicx}
\usepackage{pifont}
\usepackage{subcaption}
\usepackage{float}

\usepackage{tikz}
\usetikzlibrary{arrows,calc}


\usepackage{xparse}
\usepackage{ifthen}

\usepackage{listings}
\usepackage{hyperref}

 \usepackage[acronym]{glossaries}

\newcommand{\arcm}{\item[\Large\color{red}\ding{247}]}
\newcommand{\arcmO}{\noindent\textcolor{red}{\Large\ding{247}}\;}
%% این شکل می‌تواند برای بیان مزایای یک قضیه بکار رود (شکل تیک)
\newcommand{\tick}{\item[\large\color{green}\ding{52}]}
\newcommand{\tickO}{\noindent\textcolor{green}{\Large\ding{52}}\;}
%% برای  بیان معایب و یا نکات منفی (شکل یک ضربدر)
\newcommand{\X}{\item[\Large\color{red}\ding{56}]}
\newcommand{\XO}{\noindent\textcolor{red}{\LARGE\ding{56}}\;}
%% بیان موارد یک قضیه (شکل یک دست)
\newcommand{\hand}{\item[\Large\color{blue}\ding{45}]}
\newcommand{\handO}{\noindent\textcolor{blue}{\LARGE\ding{45}}\;}
%% برای مواردی که: این موارد شامل .... می شود، توسط عناصر زیر مشخص می شود (شکل یک درخت)
%% برای نوشتن  پارامتر‌ها، 
\newcommand{\tree}{\item[\Large\color{ForestGreen}\ding{171}]}
\newcommand{\treeO}{\noindent\textcolor{ForestGreen}{\Large\ding{171}}\;}
%% برای این که چند مورد را تعریف کنیم (علامت دست که دو گرفته)
\newcommand{\two}{\item[\LARGE\color{blue}\ding{44}]}
\newcommand{\twoO}{\noindent\textcolor{blue}{\LARGE\ding{44}}\;}
%% (شکل یک قیچی)
\newcommand{\sci}{\item[\footnotesize\color{OrangeRed}\ding{108}]}
\newcommand{\sciO}{\noindent\textcolor{OrangeRed}{\footnotesize\ding{108}}\;}

\newcommand{\starE}{\item[\Large\color{Plum}\ding{97}]}
\newcommand{\starEO}{\noindent\textcolor{Plum}{\Large\ding{97}}\;}
%% برای حالت‌هایی که فرضیات داریم. 
\newcommand{\music}{\item[\Large\color{Green}\ding{161}]}
\newcommand{\musicO}{\noindent\textcolor{Green}{\Large\ding{161}}\;}

\newcommand{\gol}{\item[\Huge\color{RubineRed}\ding{96}]}
\newcommand{\golO}{\noindent\textcolor{RubineRed}{\Huge\ding{96}}\;}

\definecolor{quotationcolour}{HTML}{F0F0F0}
\definecolor{quotationmarkcolour}{HTML}{1F3F81}

% Double-line for start and end of epigraph.
\newcommand{\epiline}{\color{magenta}\hrule \vskip -.2em \hrule}
% Massively humongous opening quotation mark.
\newcommand{\hugequote}{%
  \fontsize{42}{48}\selectfont \color{quotationmarkcolour} \textbf{``}
  \vskip -.5em
}

% Beautify quotations.
\newcommand{\epigraph}[2]{%

  \begin{flushright}
  \colorbox{quotationcolour}{%
    \parbox{\textwidth}{%
    {\epiline} \vskip 1em {\hugequote} \vskip -1.7em
    \parindent 1.8em \baselineskip=.8cm
    #1\begin{flushright}\textsc{#2}\end{flushright}
    {\epiline}
    }
  }
  \end{flushright}
}

\lstset{
	backgroundcolor=\color{Plum!5},
	frameround=fttt,
	frame=trBL,
	aboveskip=5mm,
	belowskip=4mm,
	language=sh,
	%numbers=left,
	basicstyle=\ttfamily\footnotesize,
	keywordstyle=\color{blue}\bfseries,
	keywords={set,show,read,add,sudo,service,ping,unlock},
	stringstyle=\ttfamily\color{red}, 
	commentstyle=\color{Green},
	lineskip = .4pt,
	captionpos=b, 
	breaklines=true,
	showstringspaces=true,
	emph={RAN},emphstyle={\color{Plum}}
}% 


\hypersetup{
	colorlinks=true, citecolor=Green
} % M

\def\thesection{\arabic{section}}
\renewcommand{\thesection}{\arabic{section}} 
\renewcommand{\thefigure}{\arabic{figure}}
\renewcommand{\thetable}{\arabic{table}}


\newcommand{\bigqm}[1][1]{\textcolor{red}{\text{\larger[#1]{\textbf{?}}}}}


\newcommand{\handBS}{\noindent\textcolor{ForestGreen}{\Huge\ding{45}}}
\NewDocumentEnvironment{note}{g g}{
\tikzstyle{mybox1} = [draw=YellowGreen, fill=green!15,very thick, rectangle, rounded corners, inner sep=10pt, inner ysep=20pt]
\tikzstyle{fancytitle1} =[fill=YellowGreen, text=white]
\tikzstyle{fancytitle2} =[fill=YellowGreen!5, text=white]
\tikzstyle{fancytitle3} =[fill=white, text=white]
	\begin{center}
		\begin{tikzpicture}
			\node [mybox1] (box)\bgroup
				\IfFileExists{warining.png}{\begin{minipage}{.85\textwidth}}{\begin{minipage}{.93\textwidth}}
			\baselineskip=.6cm
}{%
			\end{minipage}
			\egroup;
			\IfValueTF{#1}{\node[fancytitle1, left=10pt] at (box.north east) {\large \textbf{#1}};}{\node[fancytitle1, left=10pt] at (box.north west) {\large \textbf{Point}};}%
				\IfFileExists{warining.png}
				{\node[fancytitle3, left=3pt,   rounded corners] at (box.west) {\includegraphics[width=.08\textwidth]{warining}}; }
				{\node[fancytitle2,  rounded corners] at (box.east) {\handBS};}
		\end{tikzpicture}
	\end{center}
}%


\NewDocumentEnvironment{problem}{g g}{
\tikzstyle{mybox1} = [draw=red, fill=red!15,very thick, rectangle, rounded corners, inner sep=10pt, inner ysep=20pt]
\tikzstyle{fancytitle1} =[fill=red, text=white]
\tikzstyle{fancytitle2} =[fill=red!5, text=white]
\tikzstyle{fancytitle3} =[fill=white, text=white]
	\begin{center}
		\begin{tikzpicture}
			\node [mybox1] (box)\bgroup
				\IfFileExists{infoRR.png}{\begin{minipage}{.85\textwidth}}{\begin{minipage}{.93\textwidth}}
			\baselineskip=.6cm
}{%
			\end{minipage}
			\egroup;
			\IfValueTF{#1}{\node[fancytitle1, left=10pt] at (box.north east) {\large \textbf{#1}};}{\node[fancytitle1, left=10pt] at (box.north east) {\large \textbf{Problem}};}%
				\IfFileExists{infoRR.png}
				{\node[fancytitle3, left=3pt,   rounded corners] at (box.west) {\includegraphics[width=.08\textwidth]{infoRR}}; }
				{\node[fancytitle2,  rounded corners] at (box.east) {\handBS};}
		\end{tikzpicture}
	\end{center}
}%

\renewcommand{\figurename}{Fig.}

\defglsentryfmt[acronym]{\glsentryname{\glslabel}\ifglsused{\glslabel}{}{\footnote{\glsentrydesc{\glslabel}}}}


