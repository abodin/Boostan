\documentclass{article}

% بسته ای برای تنظیم حاشیه صفحات
\usepackage[top=3cm, bottom=2.5cm, left=2cm, right=2.5cm]{geometry}
% بسته ای برای آوردن فهرست اختصارات در متن
\usepackage[intoc]{nomencl}
% بسته ای برای رنگی کردن لینک ها و فعال سازی لینک ها در یک نوشتار، بسته hyperref باید جزو آخرین بسته‌هایی باشد که فراخوانی می‌شود. 
\usepackage{hyperref}
\hypersetup{ colorlinks=true,linkcolor=blue, anchorcolor=green, citecolor=magenta, urlcolor=cyan, filecolor=magenta, pdftoolbar=true, pdfpagemode=UseOutlines}

% نکته مهم این جا است که بسته Xepersian برای پشتیبانی از زبان فارسی آورده شده است، و 
% می بایست آخرین بسته ای باشد که شما وارد می کنید، دقت کنید: آخرین بسته 
\usepackage{xepersian}

%%  با دستور زیر می توانید فونتی مخصوص عبارات فارسی تعیین کنید:
\settextfont[Scale=1.2]{XB Niloofar}
%% شما با دستور زیر بعد از فراخوانی بسته xepersian می توانید فونت انگلیسی را تعیین کنید
%% دقت کنید که عبارات انگلیسی شما باید در دستور \lr{} قرار گیرد تا xepersian بتواند بفمهد که این عبارات انگلیسی است
\setlatintextfont[Scale=1]{Times New Roman}

%%%%%%%%%%%%%%%%%%%%%%%%%%%%%%%%%%%%%%%%%%%%%%%%%%%%%%%%%%%%%%%%%%%%%%%%%%%
%% دستوراتی برای قرار دادن فهرست اختصارات 
%% برای وارد کردن فهرست اختصارات  نیازمند بسته nomencl هستیم که در قسمت usepackage وارد شد. 
%% توسط این دستور عنوان را تعیین می کنید مثلا: (فهرست اختصارات)
\renewcommand{\nomname}{فهرست اختصارات}
% همانند ساخت واژه نامه و ساخت نمایه این دستور برای شروع ساخت فهرست اختصارات الزامی است.
\makenomenclature
%% دستور \nomitemsep برای تنظیم فاصله بین item های فهرست اختصارات 
\setlength{\nomitemsep}{-7pt}
%% این دستور هم برای این است که محیط فهرست اختصارات یک محیط لاتین است ولی عنوان آن فارسی است.
%% محیط لاتین باید در سمت چپ و عنوان باید در سمت راست وجود داشته باشد
\renewcommand{\nompreamble}{\begin{latin}}
\renewcommand{\nompostamble}{\end{latin}}
%% تعریف یک دستور برای قرار دادن در پاورقی و اضافه کردن به لیست فهرست اختصارات 
\def\inpabr #1#2{\lr{#1\LTRfootnote{#2}}\nomenclature{#1}{#2}}
%% همانند دستور بالا تنها تفاوت در این است که دیگر در پاورقی آورده نمی‌شود. 
\def\inabr #1#2{\lr{#1}\nomenclature{#1}{#2}}

%%%%%%%%%%%%%%%%%%%%%%%%%%%%%%%%%%%%%%%%%%%%%%%%%%%%%%%%%%%%%%%%%%%%%%%%%%%


\begin{document}
\baselineskip=.90cm

برای تولید فهرست اختصارات از بسته {\lr{nomencl}} استفاده کنید. مراحل تولید فهرست اختصارات بسیار ساده و راحت است. این مراحل به شرح زیر می باشد. 
\begin{itemize}
\item
وارد کردن بسته {\lr{nomencl}} به صورت زیر:
\begin{LTR}
\verb+\usepackage[intoc]{nomencl}+
\end{LTR}
\item
قرار دادن دستور {\verb+ \makenomenclature +} قبل از {\verb+ \begin{document}+}.
\item
دستور {\lr{inpabr}} اختصار را در داخل متن قرار داده و باز شده آن را در پاورقی می آورد و در ضمن آن را به فهرست اختصارات نیز اضافه می کند، مثل {\inpabr{UMTS}{Universal Mobile Telecommunications System}}.

 دستور {\lr{inabr}} دیگر باز شده در پاورقی نمی گذارد، فقط اختصار را در متن ظاهر و در فهرست واژگان اضافه می کند، مثل {\inabr{GSM}{Global System for Mobile}}.
 
  دستور {\lr{nomenclature}} تنها اختصار و باز شده آن را در فهرست واژگان می آورد و اصلا در متن ظاهر نمی شود، مثل {\nomenclature{CDMA}{Code division multiple access}}
\item
 در هر جایی که می خواهید فهرست اختصارات وارد شود دستور {\lr{printnomenclature}} را تایپ کنید. این دستور باید به صورت زیر باشد:
\begin{LTR}
\verb+ \markboth{\nomname}{\nomname}+

\verb+ \printnomenclature[70pt]+
\end{LTR}
آرگومان این دستور فاصله بین اختصار و بازشده آن را تعیین می کند. در این جا یعنی فاصله بین اختصار و بازشده آن در فهرست اختصارات ۷۰ پوینت است. 
\item
در انتهای نیز دنباله زیر را برای تولید فهرست اختصارات انجام دهید:
\begin{LTR}
\begin{itemize}
\item
\verb+ xelatex -interaction=nonstopmode -synctex=-1 %.tex+
\item
\verb+ makeindex %.nlo -s nomencl.ist -o %.nls +
\item
\verb+ xelatex -interaction=nonstopmode -synctex=-1 %.tex+
\item
\verb+ xelatex -interaction=nonstopmode -synctex=-1 %.tex+
\end{itemize}
\end{LTR}
\end{itemize}

برای تولید فهرست اختصارات کافی است از همین فایل مثال استفاده کنید، و خطوط قبل از {\lr{begin{document}} و خطوط انتهایی را در متن خود وارد کنید و به روش یاد شده کامپایل را انجام دهید.  در ضمن با آوردن بسته {\lr{hyperref}} فهرست اختصارات به صورت مستقیم وارد {\lr{bookmark}} و فهرست مطالب می شود و نیازی به وارد کردن دستی آن نیست.

مثال های دیگر:

نوع اول:
\inpabr{OVSF}{Orthogonal Variable Spreading Factor},
\inpabr{QoS}{Quality of Service}

نوع دوم:
\inabr{FM}{Frequency Modulation},
\inabr{QoE}{Quality of end-user Experience}

نوع سوم: 
\nomenclature{WLAN}{Wireless Local Area Network},
\nomenclature{PTT}{Push To Talk}


%%%%%%%%%%%%%%%%%%%%%%%%%%%%%%%%%%%%%%%%%%%%%%%%%%%%%%%%%%%%%%%%%%%%%%%%%%%
\clearpage
\markboth{\nomname}{\nomname}% maybe with \MakeUppercase
%	%  معمولا فهرست اختصارات  در گزارشات تخصصی بعد از فهرست ها قرار می دهند. ما هم از همین رویه استفاده کردیم	
%	% دستور \printnomenclature برای وارد کردن واژه نامه. option بعد از آن یعنی [70pt] نشان‌دهنده میزان فاصله بین کلمات و توضیح آن ها می‌باشد. 
\printnomenclature[70pt]\clearpage

\end{document}


