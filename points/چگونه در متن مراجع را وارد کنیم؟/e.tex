\documentclass{report}
% برای تنظیم حاشیه صفحات
\usepackage[top=3cm, bottom=2.5cm, left=2cm, right=2.5cm]{geometry}
% بسته ای برای رنگی کردن لینک ها و فعال سازی لینک ها در یک نوشتار، بسته hyperref باید جزو آخرین بسته‌هایی باشد که فراخوانی می‌شود. 

\usepackage{hyperref}


\usepackage{xepersian}

%% در این قسمت تنظیمات بسته hyperref را قرار می دهیم.
%% این تنظیمات شامل موارد زیر است.
\hypersetup{ 
% به جای استفاده از مربع قرمز دور موارد ارجاعی از لینک های رنگی استفاده کند.
colorlinks=true}


%%  با دستور زیر می توانید فونتی مخصوص عبارات فارسی تعیین کنید:
\settextfont[Scale=1.2]{XB Niloofar}
%% شما با دستور زیر بعد از فراخوانی بسته xepersian می توانید فونت انگلیسی را تعیین کنید
%% دقت کنید که عبارات انگلیسی شما باید در دستور \lr{} قرار گیرد تا xepersian بتواند بفمهد که این عبارات انگلیسی است
\setlatintextfont[Scale=1]{Times New Roman}


\begin{document}
\baselineskip = .85cm


برای مثال مرجع {\cite{Beasley}} در مورد شبکه .... . و این هم مرجع دوم {\cite{Meyer2004}} 

و سپس مرجع سوم {\cite{Rafsanjani2010}}

برای آوردن مراجع باید مراحل زیر را انجام دهید.
\begin{itemize}
\begin{LTRitems}
\item
\verb+ xelatex -interaction=nonstopmode -synctex=-1 %.tex+
\item
\verb+ bibtex8 -W -c cp1256fa % +
\item
\verb+ xelatex -interaction=nonstopmode -synctex=-1 %.tex+
\item
\verb+ xelatex -interaction=nonstopmode -synctex=-1 %.tex+
\end{LTRitems}
\end{itemize}
اگر از ویرایشگر {\lr{Texmaker}} استفاده می‌کنید، دستور اولی، سومی و چهارمی همان {\lr{Quick Build}} است. یعنی اگر دکمه {\lr{Quick Build}} را بزنید، انگار دستور مورد اشاره را اجرا کرده اید. در مورد دستور دوم، در {\lr{Texmaker}} همان دستور {\lr{bibtex}} است. در اکثر ویرایشگرها چنین چیزی وجود دارد.  

% انواع دیگر استایل ها در راهنمای persian-bib آمده است. 

\cite{R1}
\bibliographystyle{ieeetr-fa}
\bibliography{myref}

\end{document}

