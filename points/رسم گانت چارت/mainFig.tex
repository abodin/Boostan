\documentclass{article}

%%% برای رنگی کردن متن و استفاده از رنگ در متن این دو بسته مورد نیاز است.
\usepackage[usenames,dvipsnames]{color,xcolor}

%%% بسته‌ای برای رسم گانت چارت
\usepackage{pgfgantt}

\usetikzlibrary{shadows}
\usetikzlibrary{shadings}

%%% برای تنظیم حاشیه صفحات
\usepackage{geometry}
\geometry{top=.05cm, bottom=0cm, left=-.4cm, right=0cm,paperwidth=9in, paperheight=9.4in}

%زی‌پرشین (به انگلیسی: XePersian) یک بسته حروف‌چینی رایگان و متن‌باز برای نگارش مستندات پارسی/انگلیسی با زی‌لاتک است.
% در واقع، زی‌پرشین، کمک می‌کند تا به آسانی، مستندات را به پارسی، حروف‌چینی کرد. این بسته را وفا خلیقی نوشته است،
% و به طور منظم، آن را بروز‌رسانی کرده و باگ‌های آن را رفع می‌کند.
% نکته مهم این جا است که بسته Xepersian برای پشتیبانی از زبان فارسی آورده شده است، و 
% می بایست آخرین بسته ای باشد که شما وارد می کنید، دقت کنید: آخرین بسته
\usepackage{xepersian}

%%  با دستور زیر می توانید فونتی مخصوص عبارات فارسی تعیین کنید:
\settextfont[Scale=1.2]{XB Niloofar}
%% شما با دستور زیر بعد از فراخوانی بسته xepersian می توانید فونت انگلیسی را تعیین کنید
%% دقت کنید که عبارات انگلیسی شما باید در دستور \lr{} قرار گیرد تا xepersian بتواند بفمهد که این عبارات انگلیسی است
\setlatintextfont[Scale=1]{Times New Roman}

\begin{document}
\pagestyle{empty}
\begin{tikzpicture}

\begin{ganttchart}[
vgrid,canvas/.style={draw=none}, title/.append style={fill=blue!20, rounded corners=2mm, drop shadow},
group label font=\color{Green},group/.append style={draw=none, fill=Green!40},
progress label text={\lr{#1 \%}},
bar/.append style={fill=red},
bar incomplete/.append style={fill=gray!30},
milestone/.append style={fill=blue, draw=blue}
]{1}{31}
\gantttitle{1392}{12}
\gantttitle{1393}{12} 
\gantttitle{1394}{6}\\
\gantttitle{بهار}{3}
\gantttitle{تابستان}{3}
\gantttitle{پاییز}{3}
\gantttitle{زمستان}{3}
\gantttitle{بهار}{3}
\gantttitle{تابستان}{3}
\gantttitle{پاییز}{3}
\gantttitle{زمستان}{3}
\gantttitle{بهار}{3}
\gantttitle{تابستان}{3} \\
\gantttitlelist[title/.style={draw=none, inner color=Magenta!50},title label font=\color{black!60}]{1,...,30}{1} \\

%%% ======================================================================================================

\ganttgroup[]{\rl{
انتخاب و پیش‌نیازها
}}{1}{8} \\
\ganttbar[name=Queuing,progress=100]{\rl{
انتخاب موضوع
}}{1}{3} \\
\ganttbar[name=Capacity,progress=20]{\rl{
مباحث مقدماتی در موضوع
}}{4}{5} \\
\ganttbar[name=Efficiency,progress=100]{\rl{
بررسی خروجی صف‌های مختلف
}}{6}{7} \\
\ganttmilestone{\textcolor{Blue}{\rl{
گزارش
}}}{8} \\


\ganttgroup{\rl{
صف با اولویت
}}{9}{12} \\
\ganttbar[name=History,progress=100]{\rl{
صف‌ با اولویت و صف‌ با تغییر اولویت در زمان
}}{9}{9} \\
\ganttbar[name=Requirement,progress=100]{\rl{
تحلیل خروجی صف اولویت‌دار
}}{10}{10} \\
\ganttbar[name=Status,progress=70]{\rl{
پیشنهاد چارچوب صف‌های اولویت‌دار
}}{10}{10} \\
\ganttbar[name=Status2,progress=30]{\rl{
پارامترهای کارایی برای چارچوب یاد شده
}}{10}{11} \\
\ganttbar[name=Status3,progress=60]{\rl{
شبیه‌سازی
}}{11}{12} \\
\ganttmilestone{\textcolor{Blue}{\rl{
گزارش
}}}{12} \\


\ganttgroup{\rl{
خارج کردن صف از حالت پایدار
}}{13}{26} \\
\ganttbar[name=HistoryRT,progress=100]{\rl{
چارچوب پیشنهادی برای حالت پایدار
}}{13}{16} \\
\ganttbar[name=Requi,progress=0]{\rl{
نوشتن معادلات دیفرانسیل در حالت غیر پایدار
}}{17}{19} \\
\ganttbar[name=StaE,progress=34]{\rl{
ارایه یک معیار برای میزان فواصل دو توزیع
}}{19}{21} \\
\ganttbar[name=StaY,progress=10]{\rl{
پارامترهای کارایی برای چارچوب پیشنهادی
}}{21}{23} \\
\ganttbar[name=StaO,progress=30]{\rl{
شبیه‌سازی
}}{24}{26} \\
\ganttmilestone{\textcolor{Blue}{\rl{
گزارش
}}}{26} \\


\ganttlink{HistoryRT}{Requi}
\ganttlink{Requi}{StaE}
\ganttlink{StaE}{StaY}
\ganttlink{StaY}{StaO}

\end{ganttchart}

\end{tikzpicture}

\end{document}
